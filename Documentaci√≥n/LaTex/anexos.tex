\documentclass[a4paper,12pt,twoside]{memoir}

% Castellano
\usepackage[spanish,es-tabla]{babel}
\selectlanguage{spanish}
\usepackage[utf8]{inputenc}
\usepackage[T1]{fontenc}
\usepackage{lmodern} % scalable font
\usepackage{microtype}
\usepackage{placeins}
\usepackage[inline]{enumitem}
\usepackage{listings}
\usepackage{lscape}

\RequirePackage{booktabs}
\RequirePackage[table]{xcolor}
\RequirePackage{xtab}
\RequirePackage{multirow}

% Links
\usepackage[colorlinks]{hyperref}
\hypersetup{
	allcolors = {red}
}

% Ecuaciones
\usepackage{amsmath}

%Para las ecuaciones en formato SI
\usepackage{eurosym}
\usepackage{amstext} % for \text
\DeclareRobustCommand{\officialeuro}{%
	\ifmmode\expandafter\text\fi
	{\fontencoding{U}\fontfamily{eurosym}\selectfont e}}

% Rutas de fichero / paquete
\newcommand{\ruta}[1]{{\sffamily #1}}

%Codigo para listado
\renewcommand{\lstlistingname}{Código}

% Párrafos
\nonzeroparskip


% Imagenes
\usepackage{graphicx}
\newcommand{\imagen}[3]{
	\begin{figure}[htbp]
		\centering
		\includegraphics[width=#3\textwidth]{#1}
		\caption{#2}\label{fig:#1}
	\end{figure}
	\FloatBarrier
}

\newcommand{\imagenflotante}[2]{
	\begin{figure}%[!h]
		\centering
		\includegraphics[width=0.9\textwidth]{#1}
		\caption{#2}\label{fig:#1}
	\end{figure}
}

%Código JSON
\colorlet{punct}{red!60!black}
\definecolor{background}{HTML}{EEEEEE}
\definecolor{delim}{RGB}{20,105,176}
\colorlet{numb}{magenta!60!black}

\lstdefinelanguage{json}{
	basicstyle=\normalfont\ttfamily,
	showstringspaces=false,
	breaklines=true,
	frame=lines,
	backgroundcolor=\color{background},
	literate=
	*{0}{{{\color{numb}0}}}{1}
	{1}{{{\color{numb}1}}}{1}
	{2}{{{\color{numb}2}}}{1}
	{3}{{{\color{numb}3}}}{1}
	{4}{{{\color{numb}4}}}{1}
	{5}{{{\color{numb}5}}}{1}
	{6}{{{\color{numb}6}}}{1}
	{7}{{{\color{numb}7}}}{1}
	{8}{{{\color{numb}8}}}{1}
	{9}{{{\color{numb}9}}}{1}
	{:}{{{\color{punct}{:}}}}{1}
	{,}{{{\color{punct}{,}}}}{1}
	{\{}{{{\color{delim}{\{}}}}{1}
	{\}}{{{\color{delim}{\}}}}}{1}
	{[}{{{\color{delim}{[}}}}{1}
	{]}{{{\color{delim}{]}}}}{1},
}


% El comando \figura nos permite insertar figuras comodamente, y utilizando
% siempre el mismo formato. Los parametros son:
% 1 -> Porcentaje del ancho de página que ocupará la figura (de 0 a 1)
% 2 --> Fichero de la imagen
% 3 --> Texto a pie de imagen
% 4 --> Etiqueta (label) para referencias
% 5 --> Opciones que queramos pasarle al \includegraphics
% 6 --> Opciones de posicionamiento a pasarle a \begin{figure}
\newcommand{\figuraConPosicion}[6]{%
  \setlength{\anchoFloat}{#1\textwidth}%
  \addtolength{\anchoFloat}{-4\fboxsep}%
  \setlength{\anchoFigura}{\anchoFloat}%
  \begin{figure}[#6]
    \begin{center}%
      \Ovalbox{%
        \begin{minipage}{\anchoFloat}%
          \begin{center}%
            \includegraphics[width=\anchoFigura,#5]{#2}%
            \caption{#3}%
            \label{#4}%
          \end{center}%
        \end{minipage}
      }%
    \end{center}%
  \end{figure}%
}

%
% Comando para incluir imágenes en formato apaisado (sin marco).
\newcommand{\figuraApaisadaSinMarco}[5]{%
  \begin{figure}%
    \begin{center}%
    \includegraphics[angle=90,height=#1\textheight,#5]{#2}%
    \caption{#3}%
    \label{#4}%
    \end{center}%
  \end{figure}%
}
% Para las tablas
\newcommand{\otoprule}{\midrule [\heavyrulewidth]}
%
% Nuevo comando para tablas pequeñas (menos de una página).
\newcommand{\tablaSmall}[5]{%
 \begin{table}
  \begin{center}
   \rowcolors {2}{gray!35}{}
   \begin{tabular}{#2}
    \toprule
    #4
    \otoprule
    #5
    \bottomrule
   \end{tabular}
   \caption{#1}
   \label{tabla:#3}
  \end{center}
 \end{table}
}

%
%Para el float H de tablaSmallSinColores
\usepackage{float}

%
% Nuevo comando para tablas pequeñas (menos de una página).
\newcommand{\tablaSmallSinColores}[5]{%
 \begin{table}[H]
  \begin{center}
   \begin{tabular}{#2}
    \toprule
    #4
    \otoprule
    #5
    \bottomrule
   \end{tabular}
   \caption{#1}
   \label{tabla:#3}
  \end{center}
 \end{table}
}

\newcommand{\tablaApaisadaSmall}[5]{%
\begin{landscape}
  \begin{table}
   \begin{center}
    \rowcolors {2}{gray!35}{}
    \begin{tabular}{#2}
     \toprule
     #4
     \otoprule
     #5
     \bottomrule
    \end{tabular}
    \caption{#1}
    \label{tabla:#3}
   \end{center}
  \end{table}
\end{landscape}
}

%
% Nuevo comando para tablas grandes con cabecera y filas alternas coloreadas en gris.
\newcommand{\tabla}[6]{%
  \begin{center}
    \tablefirsthead{
      \toprule
      #5
      \otoprule
    }
    \tablehead{
      \multicolumn{#3}{l}{\small\sl continúa desde la página anterior}\\
      \toprule
      #5
      \otoprule
    }
    \tabletail{
      \hline
      \multicolumn{#3}{r}{\small\sl continúa en la página siguiente}\\
    }
    \tablelasttail{
      \hline
    }
    \bottomcaption{#1}
    \rowcolors {2}{gray!35}{}
    \begin{xtabular}{#2}
      #6
      \bottomrule
    \end{xtabular}
    \label{tabla:#4}
  \end{center}
}

%
% Nuevo comando para tablas grandes con cabecera.
\newcommand{\tablaSinColores}[6]{%
  \begin{center}
    \tablefirsthead{
      \toprule
      #5
      \otoprule
    }
    \tablehead{
      \multicolumn{#3}{l}{\small\sl continúa desde la página anterior}\\
      \toprule
      #5
      \otoprule
    }
    \tabletail{
      \hline
      \multicolumn{#3}{r}{\small\sl continúa en la página siguiente}\\
    }
    \tablelasttail{
      \hline
    }
    \bottomcaption{#1}
    \begin{xtabular}{#2}
      #6
      \bottomrule
    \end{xtabular}
    \label{tabla:#4}
  \end{center}
}

%
% Nuevo comando para tablas grandes sin cabecera.
\newcommand{\tablaSinCabecera}[5]{%
  \begin{center}
    \tablefirsthead{
      \toprule
    }
    \tablehead{
      \multicolumn{#3}{l}{\small\sl continúa desde la página anterior}\\
      \hline
    }
    \tabletail{
      \hline
      \multicolumn{#3}{r}{\small\sl continúa en la página siguiente}\\
    }
    \tablelasttail{
      \hline
    }
    \bottomcaption{#1}
  \begin{xtabular}{#2}
    #5
   \bottomrule
  \end{xtabular}
  \label{tabla:#4}
  \end{center}
}



\definecolor{cgoLight}{HTML}{EEEEEE}
\definecolor{cgoExtralight}{HTML}{FFFFFF}

%
% Nuevo comando para tablas grandes sin cabecera.
\newcommand{\tablaSinCabeceraConBandas}[5]{%
  \begin{center}
    \tablefirsthead{
      \toprule
    }
    \tablehead{
      \multicolumn{#3}{l}{\small\sl continúa desde la página anterior}\\
      \hline
    }
    \tabletail{
      \hline
      \multicolumn{#3}{r}{\small\sl continúa en la página siguiente}\\
    }
    \tablelasttail{
      \hline
    }
    \bottomcaption{#1}
    \rowcolors[]{1}{cgoExtralight}{cgoLight}

  \begin{xtabular}{#2}
    #5
   \bottomrule
  \end{xtabular}
  \label{tabla:#4}
  \end{center}
}


\newcommand{\enumeratecompacto}[1]{
	\begin{enumerate*}[label=\arabic*. ,itemjoin={\newline}]
		#1
	\end{enumerate*}
}

\newcommand{\tabitem}{~~\llap{\textbullet}~~}

\newcommand{\tablaAncho}[4]{%
	\begin{table}[]
		\centering
		\begin{tabularx}{\textwidth}{#2}
			\toprule
			#4
			\bottomrule
		\end{tabularx}
		\caption{#1}
		\label{tabla:#3}
	\end{table}
}

\graphicspath{ {./img/} }

% Capítulos
\chapterstyle{bianchi}
\newcommand{\capitulo}[2]{
	\setcounter{chapter}{#1}
	\setcounter{section}{0}
	\chapter*{#2}
	\addcontentsline{toc}{chapter}{#2}
	\markboth{#2}{#2}
}

% Apéndices
\renewcommand{\appendixname}{Apéndice}
\renewcommand*\cftappendixname{\appendixname}

\newcommand{\apendice}[1]{
	%\renewcommand{\thechapter}{A}
	\chapter{#1}
}

\renewcommand*\cftappendixname{\appendixname\ }

% Formato de portada
\makeatletter
\usepackage{xcolor}
\newcommand{\tutor}[1]{\def\@tutor{#1}}
\newcommand{\course}[1]{\def\@course{#1}}
\definecolor{cpardoBox}{HTML}{E6E6FF}
\def\maketitle{
  \null
  \thispagestyle{empty}
  % Cabecera ----------------
\noindent\includegraphics[width=\textwidth]{cabecera}\vspace{1cm}%
  \vfill
  % Título proyecto y escudo informática ----------------
  \colorbox{cpardoBox}{%
    \begin{minipage}{.8\textwidth}
      \vspace{.5cm}\Large
      \begin{center}
      \textbf{TFG del Grado en Ingeniería Informática}\vspace{.6cm}\\
      \textbf{\LARGE\@title{}}
      \end{center}
      \vspace{.2cm}
    \end{minipage}

  }%
  \hfill\begin{minipage}{.20\textwidth}
    \includegraphics[width=\textwidth]{escudoInfor}
  \end{minipage}
  \vfill
  % Datos de alumno, curso y tutores ------------------
  \begin{center}%
  {%
    \noindent\LARGE
    Presentado por \@author{}\\ 
    en Universidad de Burgos --- \@date{}\\
    Tutores: \@tutor{}\\
  }%
  \end{center}%
  \null
  \cleardoublepage
  }
\makeatother


% Datos de portada
\title{Visor 3D 2.0}
\author{Jose Manuel Moral Garrido}
\tutor{José Francisco Díez Pastor, Álvar Arnaiz González}
\date{\today}

\begin{document}

\maketitle



\cleardoublepage



%%%%%%%%%%%%%%%%%%%%%%%%%%%%%%%%%%%%%%%%%%%%%%%%%%%%%%%%%%%%%%%%%%%%%%%%%%%%%%%%%%%%%%%%



\frontmatter


\clearpage

% Indices
\tableofcontents

\clearpage

\listoffigures

\clearpage

\listoftables

\clearpage

\mainmatter

\appendix

\apendice{Plan de Proyecto Software}

\section{Introducción}
Cabe mencionar que no he realizado una dedicación a tiempo completo al trabajo de final de grado, sino que ha sido una dedicación parcial. Esto es debido a que me encuentro trabajando al mismo tiempo, por lo tanto, en la mayoría de los \textit{sprint} se han cerrado la mayor parte de las tareas el mismo día ya que mi horario laboral cambia cada semana. Por lo tanto, el día o los días que tengo libres los aprovecho al completo avanzando en el proyecto.

\section{Planificación temporal}
A continuación, detallaremos los diferentes objetivos que se han establecido para cada \textit{sprint} y, a su vez, el progreso obtenido en los mismos. Para ello hemos utilizado una metodología ágil denominada \textit{SCRUM}~\cite{schwaber2002agile}. Emplearemos a su vez el gestor de tareas provisto por GitHub y generaremos gráficos \textit{burndown} para el seguimiento de los \textit{sprint}, los cuales son provistos por la extensión \textit{ZenHub}.

\subsection{Sprint 1 - 18--92017/24-09-2017}
En este \textit{sprint} se pretenden instalar las herramientas necesarias para la realización del proyecto, así como la lectura y comprensión de la \textit{memoria} y \textit{anexos} del proyecto de partida~\cite{github:alberto-viewer}. A su vez, se pretende probar los métodos creados de la aplicación de partida para comprobar su correcto funcionamiento.

Podemos observar el progreso del \textit{sprint} en la figura~\ref{fig:sprint-1}
\imagen{sprint-1}{Progreso en el \textit{sprint} 1.}{0.9}

\subsection{Sprint 2 - 25-09-2017/01-10-2017}
En este \textit{sprint} se pretende subir a mi repositorio propio toda la información necesaria para continuar el proyecto y solucionar los fallos encontrados la semana anterior en los distintos métodos de la aplicación. Además se prentende conocer la estructura completa del proyecto con el fin de agilizar las tareas en el momento de la búsqueda de los puntos de la aplicación a corregir o modificar.

Podemos observar el progreso del \textit{sprint} en la figura~\ref{fig:sprint-2}
\imagen{sprint-2}{Progreso en el \textit{sprint} 2.}{0.9}

\subsection{Sprint 3 - 02-10-2017/08-10-2017}
En este \textit{sprint} se pretende cambiar la manera que tiene la aplicación de cotejar los roles de los usuarios (mediante base de datos) a ser cotejados y asignados mediante la \textit{API} de UBUVirtual e intentar interar la aplicación en un servidor público como es \textit{Heroku}~\cite{wiki:heroku}.

Podemos observar el progreso del \textit{sprint} en la figura~\ref{fig:sprint-3}
\imagen{sprint-3}{Progreso en el \textit{sprint} 3.}{0.9}

\subsection{Sprint 4 - 09-10-2017/15-10-2017}
En este \textit{sprint} se pretende continuar con el intento de integración de la aplicación en \textit{Heroku}. A su vez, se pretende albergar los modelos de manera privada para que los alumnos no puedan acceder a ellos nada más que mediante la plataforma. También instalaremos \textit{Moodle} de manera local para poder realizar pruebas sin depender de un tutor que pueda facilitarnos asignaturas, asignación de roles, subida de recursos, etc.

Podemos observar el progreso del \textit{sprint} en la figura~\ref{fig:sprint-4}
\imagen{sprint-4}{Progreso en el \textit{sprint} 4.}{0.9}

\subsection{Sprint 5 - 16-10-2017/22-10-2017}
En este \textit{sprint} se pretende instalar de nuevo \textit{Moodle}, ya que el instalado en el \textit{sprint} anterior no funcionaba correctamente. A su vez continuamos con la integración de la aplicación en \textit{Heroku} (tarea que se está alargando por dos motivos: errores en la estructura de la aplicación y que nos encontramos a la espera de que la UBU nos proporcione un servidor que cumpla ciertos requisitos).

Podemos observar el progreso del \textit{sprint} en la figura~\ref{fig:sprint-5}
\imagen{sprint-5}{Progreso en el \textit{sprint} 5.}{0.9}

\subsection{Sprint 6 - 24-10-2017/29-10-2017}
En este \textit{sprint} se pretende revertir la aplicación a un punto anterior ya que podemos decir que no esperábamos que la UBU nos proporcionara un servidor para la ejecución de nuestra aplicación. Esta vuelta atrás la realizaremos de manera manual ya que si la realizamos mediante los \textit{commit}, perderemos unos cambios que no queremos perder. Este cambio manual también involucra cambiar las dependencias que eran necesarias para el lanzamiento de la aplicación en Heroku (\textit{sprint 5}), ya que en este \textit{sprint} hemos sustituido un servidor público como es Heroku por el proporcionado por la Universidad de Burgos. Por otra parte, tendremos que ejecutar la aplicación en el servidor provisto por la \textit{UBU} (\textit{arquimedes}), realizando a su vez una comparativa de los distintos servidores posibles para desplegar nuestra \textit{API}

Podemos observar el progreso del \textit{sprint} en la figura~\ref{fig:sprint-6}
\imagen{sprint-6}{Progreso en el \textit{sprint} 6.}{0.9}

\subsection{Sprint 7 - 30-10-2017/06-11-2017}
En este \textit{sprint} se pretende trabajar los aspectos relacionados con la seguridad de los modelos subidos al servidor. Con esto queremos decir que en este \textit{sprint} nos dedicaremos a realizar una encriptación de los modelos para que únicamente los usuarios autorizados sean capaces de visualizar el modelo tal y como es. Esto se realiza con el fin de conservar la privacidad de los modelos ya que estos son únicos.
La encriptación se realizará en el momento de la subida del modelo a la aplicación alojada en el servidor, y justo en el momento de visualizar el modelo se realizará su desencriptación. La encriptación se hará para los modelos \textit{PLY} tanto en formato \textbf{binario} como en \textbf{ascii}, mientras que la desencriptación se hará únicamente desde formato \textit{ascii} para así ahorrar tiempo, complejidad y la programación de dos desencriptadores diferentes. A su vez, se realizará un estudio de los tiempos de carga de los modelos debido a las operaciones realizadas para proceder con su encriptación y desencriptación.

En este \textit{sprint} no se ha conseguido integrar el \textit{script} de desencriptación en el cargador de modelos \textit{PLY}, por lo que será una tarea prioritaria en el siguiente \textit{sprint}.

Podemos observar el progreso del \textit{sprint} en la figura~\ref{fig:sprint-7}
\imagen{sprint-7}{Progreso en el \textit{sprint} 7.}{0.9}

\subsection{Sprint 8 - 06-11-2017/12-11-2017}
En este \textit{sprint} trataremos de terminar de hacer funcionar la funcionalidad de encriptación y desencriptación de nuestro visor, ya que en el \textit{sprint} anterior tuvimos problemas con el tema de los números en coma flotante, lo cual será mencionado en el \textit{Manual del Programador}~\ref{sec:manual-programador}. A su vez, dedicaremos este \textit{sprint} a documentar al completo los pasos avanzados hasta el momento, así como solucionar los errores pertinentes a la hora de compilar \LaTeX.
Con el fin de mejorar los tiempos de carga del visor así como la precisión de los modelos a la hora de encriptarlos y desencriptarlos, se estudiarán diferentes métodos de encriptación (vértices, caras, etc). También realizaremos la configuración \textit{VPN} correspondiente para poder conectarnos al servidor proporcionado por la Universidad de Burgos desde otra red diferente a la de la misma.

No hemos conseguido comprobar el funcionamiento de la aplicación en el servidor <<Arquímedes>> debido a que no se han instalado correctamente las herramientas requeridas para el funcionamiento de nuestra \textit{API}, lo cual será objetivo para el siguiente \textit{sprint}.

Podemos observar el progreso del \textit{sprint} en la figura~\ref{fig:sprint-8}
\imagen{sprint-8}{Progreso en el \textit{sprint} 8.}{0.9}

\subsection{Sprint 9 - 13-11-2017/19-11-2017}
En este \textit{sprint} realizaremos la prueba no realizada en el \textit{sprint} anterior (prueba de ejecución de la aplicación en el servidor <<Arquímedes>>). También manipularemos la interfaz gráfica de la aplicación cambiando el estilo de ciertas ventanas al mismo tiempo que modificando y ampliando su funcionalidad. A su vez crearemos nuevas pantallas en nuestra aplicación con el fin de aumentar su funcionalidad y facilitar el uso de la misma al usuario (introducción de migas de pan, icono sugerentes y fáciles de comprender, etc). Añadiremos un apartado completo de ejercicios en el que está pensado que interaccione únicamente el profesor y consista en añadir diferentes soluciones a ejercicios propuesto por el mismo, pudiendo albergar una plantilla de cada uno de los ejercicios de cada modelo disponible con el fin de facilitar la enseñanza y corrección. 

Para este \textit{sprint} no hemos conseguido realizar  los siguientes puntos:
\begin{itemize}
	\item Autocarga de datos (anotaciones y medidas) en el inicio del visor de ejercicios.
	\item Documentación del Manual de Usuario (ya que no hemos completado las pantallas que se corresponden con la interfaz de usuario).
\end{itemize}

Podemos observar el progreso del \textit{sprint} en la figura~\ref{fig:sprint-9}
\imagen{sprint-9}{Progreso en el \textit{sprint} 9.}{0.9}

\subsection{Sprint 10 - 20-11-2017/26-11-2017}
En este \textit{sprint} realizaremos la prueba no realizada en el \textit{sprint} anterior (autocarga de anotacione sy medidas, así como la documentación del Manual de Usuario, pero sin incluir las imágenes de las vistas ya que no tenemos aún las versiones finales de las mismas). Además, trataremos de realizar la configuración del servidor de la Universidad de Burgos con el fin de ejecutar nuestra aplicación en el. No hemos podido avanzar mucho en este \textit{sprint} debido a mi dedicación parcial al proyecto debido s estar trabajando al mismo tiempo.

Para este \textit{sprint} no hemos conseguido realizar  los siguientes puntos:
\begin{itemize}
	\item Llevar a cabo la configuración correspondiente que nos permita ejecutar nuestra \textit{API} en el servidor proporcionado por la Universidad de Burgos.
\end{itemize}

Podemos observar el progreso del \textit{sprint} en la figura~\ref{fig:sprint-10}
\imagen{sprint-10}{Progreso en el \textit{sprint} 10.}{0.9}

\subsection{Sprint 11 - 27-11-2017/03-12-2017}
En este \textit{sprint} realizaremos la configuración del servidor <<Arquímedes>> para posibilitar la ejecución de nuestra aplicación. A su vez, corregiremos errores encontrados en los diferentes botones de la aplicación (carga de puntos, confirmación de eliminación de ejercicio, cancelación de ejercicio ya empezado, etc). También corregiremos los errores acaecidos en la memoria y anexos del proyecto.

Aunque lo hemos intentando fehacientemente, no hemos conseguido realizar la configuración de <<Arquímedes>> debido a la aparición de diversos fallos dicha configuración, como es la pérdida de la imágenes en la ejecución. de nuestras aplicación.

Podemos observar el progreso del \textit{sprint} en la figura~\ref{fig:sprint-11}
\imagen{sprint-11}{Progreso en el \textit{sprint} 11.}{0.9}

\subsection{Sprint 12 - 05-12-2017/10-12-2017}
En este \textit{sprint} daremos prioridad a la configuración de nuestro servidor, aunque nos encontramos con dificultades las cuales no es seguro que puedan ser resueltas. A su vez, realizaremos correcciones en el código (refactorización, cambio de nombres, etc), así como cambios en la interfaz con el fin de que el usuario se sienta cómodo con la aplicación y faciliten su entendimiento. Dentro de estos cambios cabe resaltar el cambio de los colores en las esferas importadas pertenecientes a ejercicios de alumnos, así como cambios en los nombre de las etiquetas de las mismas.

Para este \textit{sprint} no hemos conseguido realizar  los siguientes puntos:
\begin{itemize}
	\item Crear botón de \textit{reset} para restablecer los datos de partida del profesor.
	\item Llevar a cabo la configuración correspondiente que nos permita ejecutar nuestra \textit{API} en el visor <<Arquímedes>> (será único objetivo del siguiente \textit{sprint}).
\end{itemize}

Podemos observar el progreso del \textit{sprint} en la figura~\ref{fig:sprint-12}
\imagen{sprint-12}{Progreso en el \textit{sprint} 12.}{0.9}

\subsection{Sprint 13 - 13-12-2017/17-12-2017}
En este \textit{sprint} nos centraremos únicamente en informarnos a fondo acerca de la <<librería>> de Apache (\textit{mod wsgi}) con la que deseamos conseguir ejecutar nuestra aplicación en el servidor <<Arquímedes>>. A su vez, deshabilitaremos el botón de \textit{logueo} para evitar errores, así como realizaremos los cambios necesarios para solucionar los errores acaecidos en la importación de los mismos datos repetidas veces. Tambié dejaremos solucionada la tarea pendiente de \textit{resetear} los datos del profesor en el visor de ejercicios.

Podemos observar el progreso del \textit{sprint} en la figura~\ref{fig:sprint-13}
\imagen{sprint-13}{Progreso en el \textit{sprint} 13.}{0.9}

\subsection{Sprint 14 - 18-12-2017/24-12-2017}
En este \textit{sprint} no centraremos en la ejecución de nuestra \textit{API} en el servidor <<Arquímedes>> para lo cual generaremos un \textit{script} que facilite el trabajo de configuración del servidor a nuestro operador. Además, incluiremos expresiones regulares con el fin de limitar los caracteres introducidos por lo alumnos y profesores a la etiquetas de anotaciones y medidas. A su vez, dedicaremos tiempo a la búsqueda de imágenes con las que podamos decorar nuestra interfaz gráfica y así dejar pulida la misma.

Podemos observar el progreso del \textit{sprint} en la figura~\ref{fig:sprint-14}
\imagen{sprint-14}{Progreso en el \textit{sprint} 14.}{0.9}


\section{Estudio de viabilidad}

\subsection{Viabilidad económica}

\subsection{Viabilidad legal}



\apendice{Especificación de Requisitos}

\section{Introducción}
Se expondrán en esta sección los distintos objetivos que la aplicación debe lograr, así como los requisitos que debe cumplir.

\section{Objetivos generales}
Como objetivos generales de nuestro proyecto tendremos:
\begin{itemize}
	\item Diferenciar en nuestra aplicación entre los diferentes roles de usuario, cada uno con sus distintas funcionalidades.
	\item Añadir restricciones en el uso de caracteres para anotaciones, medidas y ejercicios.
	\item Facilitar la navegabilidad durante el uso de la aplicación.
	\item Creación de una interfaz que permita al profesor facilidad para corregir ejercicios de alumnos.
	\item Proporcionar seguridad para los modelos debido a su unicidad a la hora de ser alojados en el servidor web.
	\item Desplegar nuestra aplicación en un servidor web.
\end{itemize}

\section{Catalogo de requisitos}
A continuación, listaremos el conjunto de requisitos funcionales extraídos a partir de los objetivos generales del proyecto.

\subsection{Requisitos funcionales}
Los requisitos funcionales se han sacado a partir de un diagrama de casos de uso hecho desde cero aunque ciertos requisitos coinciden con los de dicha versión anterior, por lo que serán iguales.

\begin{itemize}
	\item \textbf{RF-1 Visor de modelos}: la aplicación debe ser capaz de visualizar un modelo.
		\begin{itemize}
		\item \textbf{RF-1.1 Gestión de anotaciones}: la aplicación debe ser capaz de manejar anotaciones sobre el modelo.
			\begin{itemize}
				\item \textbf{RF-1.1.1 Añadir anotación}: el usuario debe poder añadir una anotación.
				\item \textbf{RF-1.1.2 Eliminar anotación}: el usuario debe poder eliminar una anotación.
				\item \textbf{RF-1.1.3 Editar anotación}: el usuario debe poder editar la etiqueta de una anotación.
				\item \textbf{RF-1.1.4 Seleccionar anotación}: el usuario debe poder seleccionar una anotación.
				\item \textbf{RF-1.1.5 Deseleccionar anotación}: el usuario debe poder deseleccionar una anotación.
			\end{itemize}
		\end{itemize}
		\begin{itemize}
			\item \textbf{RF-1.2 Gestión de medidas}: la aplicación debe ser capaz de manejar medidas sobre el modelo.
			\begin{itemize}
				\item \textbf{RF-1.2.1 Añadir medida}: el usuario debe poder añadir una medida.
				\item \textbf{RF-1.2.2 Eliminar medida}: el usuario debe poder eliminar una medida.
				\item \textbf{RF-1.2.3 Editar medida}: el usuario debe poder editar la etiqueta de una medida.
				\item \textbf{RF-1.2.4 Seleccionar medida}: el usuario debe poder seleccionar una medida.
				\item \textbf{RF-1.2.5 Deseleccionar medida}: el usuario debe poder deseleccionar una medida.
			\end{itemize}
		\end{itemize}
		\item \textbf{RF-1.3 Exportar puntos}: la aplicación debe ser capaz de exportar las anotaciones y medidas que se encuentren actualmente en el visor.
		\item \textbf{RF-1.4 Importar puntos}: la aplicación debe ser capaz de importar las anotaciones y medidas que se encuentren actualmente en el visor.

	\item \textbf{RF-2 Visor de ejercicio}: la aplicación debe ser capaz de visualizar un ejercicio guardado.
	\begin{itemize}
		\item \textbf{RF-2.1 Gestión de anotaciones}: la aplicación debe ser capaz de manejar anotaciones sobre el modelo.
		\begin{itemize}
			\item \textbf{RF-2.1.1 Añadir anotación}: el usuario debe poder añadir una anotación.
			\item \textbf{RF-2.1.2 Eliminar anotación}: el usuario debe poder eliminar una anotación.
			\item \textbf{RF-2.1.3 Editar anotación}: el usuario debe poder editar la etiqueta de una anotación.
			\item \textbf{RF-2.1.4 Seleccionar anotación}: el usuario debe poder seleccionar una anotación.
			\item \textbf{RF-2.1.5 Deseleccionar anotación}: el usuario debe poder deseleccionar una anotación.
		\end{itemize}
	\end{itemize}
	\begin{itemize}
		\item \textbf{RF-2.2 Gestión de medidas}: la aplicación debe ser capaz de manejar medidas sobre el modelo.
		\begin{itemize}
			\item \textbf{RF-2.2.1 Añadir medida}: el usuario debe poder añadir una medida.
			\item \textbf{RF-2.2.2 Eliminar medida}: el usuario debe poder eliminar una medida.
			\item \textbf{RF-2.2.3 Editar medida}: el usuario debe poder editar la etiqueta de una medida.
			\item \textbf{RF-2.2.4 Seleccionar medida}: el usuario debe poder seleccionar una medida.
			\item \textbf{RF-2.2.5 Deseleccionar medida}: el usuario debe poder deseleccionar una medida.
		\end{itemize}
	\end{itemize}
	\item \textbf{RF-2.3 Restaurar datos}: la aplicación debe ser capaz de restaurar los datos iniciales del ejercicio previo a la carga del ejercicio del alumno.
	\item \textbf{RF-2.4 Cancelar ejercicio}: la aplicación debe ser capaz de cancelar y salir de un ejercicio en curso.
	\item \textbf{RF-2.5 Exportar puntos}: la aplicación debe ser capaz de exportar las anotaciones y medidas que se encuentren actualmente en el visor.
	\item \textbf{RF-2.6 Importar puntos}: la aplicación debe ser capaz de importar las anotaciones y medidas que se encuentren actualmente en el visor.

	\item \textbf{RF-3 Listar modelos}: la aplicación debe ser capaz de mostrar de una pasada los diferentes modelos disponibles.
	\begin{itemize}
		\item \textbf{RF-3.1 Eliminar modelos}: el usuario debe ser capaz de eliminar modelos.
	\end{itemize}
	\item \textbf{RF-4 Listar ejercicios}: la aplicación debe ser capaz de manejar un listado de ejercicios de cada modelo.
	\begin{itemize}
		\item \textbf{RF-4.1 Gestión de medidas}: la aplicación debe ser capaz de manejar medidas sobre el modelo.
		\begin{itemize}
			\item \textbf{RF-4.1.1 Añadir ejercicio}: el usuario debe poder añadir un ejercicio.
			\item \textbf{RF-4.1.2 Eliminar ejercicio}: el usuario debe poder eliminar un ejercicio.
			\item \textbf{RF-4.1.3 Modificar ejercicio}: el usuario debe poder modificar un ejercicio.
			\item \textbf{RF-4.1.4 Editar nombre ejercicio}: el usuario debe poder editar el nombre de un ejercicio.
			\item \textbf{RF-4.1.5 Guardar ejercicio}: el usuario debe poder guardar un ejercicio modificado.
		\end{itemize}
	\end{itemize}
	\item \textbf{RF-5 Subir modelos}: la aplicación debe ser capaz de manejar un listado de ejercicios de cada modelo.
\end{itemize}

\subsection{Requisitos no funcionales}
Estos requisitos coinciden con los de la versión anterior a excepción de uno. Aun así, mencionaremos todos para que quede más claro.

\begin{itemize}
	\item \textbf{RNF-1 Usabilidad}: el conjunto de elementos visuales de la interfaz deben ser intuitivos y conocidos por el usuario medio, permitiendo un rápido aprendizaje.
	\item \textbf{RNF-2 Mantenibilidad}: la aplicación debe desarrollarse siguiendo alguna técnica que permita facilidad de mantenimiento e incorporación de nuevas características, así como corrección de errores.
	\item \textbf{RNF-3 Soporte}: la aplicación debe poder emplearse sobre un amplio conjunto de navegadores.
	\item \textbf{RNF-4 Internacionalización}: la aplicación debe estar diseñada para soportar diferentes idiomas.
	\item \textbf{RNF-5 Control de acceso}: la aplicación debe soportar control de acceso de usuarios.
	\item \textbf{RNF-6 Despliegue}: la aplicación debe estar desplegada de manera que sea accesible sin necesidad de una ejecución local.
\end{itemize}

\section{Especificación de requisitos}

\subsection{Actores}
En nuestro caso solamente tendremos dos actores, que serán el alumno y el profesor(administrador).

\subsection{Diagrama de casos de uso}
A continuación se mostrará el diagrama de casos de uso de nuestro proyecto por niveles en las figuras~\ref{fig:nivel1},~\ref{fig:nivel2} y~\ref{fig:nivel3}:

\imagen{nivel1}{Nivel 1 del diagrama de casos de uso.}{0.9}
\imagen{nivel2}{Nivel 2 del diagrama de casos de uso.}{0.9}
\imagen{nivel3}{Nivel 3 del diagrama de casos de uso.}{0.9}

\tablaAncho
{CU-1 Visor de modelos}
{p{2.9cm} X}
{use-case-1}
{	
	\textbf{CU-1} & \textbf{Visor de modelo} \\ \otoprule
	\textbf{Versión} & 2.0 \\ \midrule
	\textbf{Autor} & Jose Manuel Moral Garrido \\ \midrule
	\textbf{Requisitos asociados} & RF-1.1, RF-1.2, RF-1.3, RF-1.4 \\ \midrule
	\textbf{Descripción} & Permite al usuario visualizar un modelo y realizar operaciones sobre el mismo. \\ \midrule
	\textbf{Precondiciones} & 
	\tabitem El usuario ha seleccionado un modelo.
	\\ \midrule
	\textbf{Acciones} & 
	\enumeratecompacto{
		\item El usuario abre un modelo.
		\item Se muestran el modelo sin anotaciones ni medidas.
		\item Se da la posibilidad de gestionar tanto anotaciones como medidas.
	}
	\\ \midrule
	\textbf{Postcondiciones} & - \\ \midrule
	\textbf{Excepciones} & - \\ \midrule
	\textbf{Importancia} & Alta \\ 
}


\tablaAncho
{CU-1.1 Gestión de anotaciones}
{p{2.9cm} X}
{use-case-1.1}
{
	\textbf{CU-1.1} & \textbf{Gestión de anotaciones} \\ \otoprule
	\textbf{Versión} & 2.0 \\ \midrule
	\textbf{Autor} & Jose Manuel Moral Garrido \\ \midrule
	\textbf{Requisitos asociados} & RF-1, RF-1.1.1, RF-1.1.2, RF-1.1.3, RF-1.1.4, RF-1.1.5 \\ \midrule
	\textbf{Descripción} & Permite gestionar las anotaciones (añadirlas, eliminarlas, etc.). \\ \midrule
	\textbf{Precondiciones} & 
	\tabitem El usuario tiene un modelo abierto.
	\\ \midrule
	\textbf{Acciones} & 
	\enumeratecompacto{
		\item Se muestran las anotaciones del modelo visualizado.
		\item Se muestran las opciones de añadir, eliminar, editar la etiqueta de y deseleccionar las anotaciones.
	}
	\\ \midrule
	\textbf{Postcondiciones} & - \\ \midrule
	\textbf{Excepciones} & - \\ \midrule
	\textbf{Importancia} & Alta \\ 
}


\tablaAncho
{CU-1.1.1 Añadir anotación}
{p{2.9cm} X}
{use-case-1.1.1}
{	
	\textbf{CU-1.1.1} & \textbf{Añadir anotación} \\ \otoprule
	\textbf{Versión} & 2.0 \\ \midrule
	\textbf{Autor} & Jose Manuel Moral Garrido \\ \midrule
	\textbf{Requisitos asociados} & RF-1.1 \\ \midrule
	\textbf{Descripción} & Permite al usuario añadir una anotación al modelo. \\ \midrule
	\textbf{Precondiciones} & - \\ \midrule
	\textbf{Acciones} & 
	\enumeratecompacto{
		\item El usuario pincha en añadir anotación.
		\item El usuario pincha sobre el modelo en la zona donde quiere crear una anotación.
		\item Una esfera es añadida en el modelo para representar la anotación.
		\item Un elemento aparece en un menú para representar la etiqueta de la anotación.
	}
	\\ \midrule
	\textbf{Postcondiciones} & 
	\tabitem Se añade una anotación al modelo.
	\\ \midrule
	\textbf{Excepciones} & - \\ \midrule
	\textbf{Importancia} & Alta \\ 
}


\tablaAncho
{CU-1.1.2 Eliminar anotación}
{p{2.9cm} X}
{use-case-1.1.2}
{	
	\textbf{CU-1.1.2} & \textbf{Eliminar anotación} \\ \otoprule
	\textbf{Versión} & 2.0 \\ \midrule
	\textbf{Autor} & Jose Manuel Moral Garrido \\ \midrule
	\textbf{Requisitos asociados} & RF-1.1 \\ \midrule
	\textbf{Descripción} & Permite al usuario eliminar una anotación. \\ \midrule
	\textbf{Precondiciones} & 
	\tabitem Que exista una anotación en el modelo.
	
	\tabitem (opcional) Que una anotación se encuentre seleccionada.
	\\ \midrule
	\textbf{Acciones} & 
	\enumeratecompacto{
		\item Seleccionar una anotación (opcional).
		\item Pulsar sobre borrar anotación. Si se ha realizado anterior, saltamos el siguiente paso.
		\item Seleccionar una anotación (opcional).
		\item Se borra la anotación.
	}
	\\ \midrule
	\textbf{Postcondiciones} & 
	\tabitem Se elimina la anotación seleccionada.
	\\ \midrule
	\textbf{Excepciones} & - \\ \midrule
	\textbf{Importancia} & Media \\ 
}


\tablaAncho
{CU-1.1.3 Editar anotación}
{p{2.9cm} X}
{use-case-1.1.3}
{	
	\textbf{CU-1.1.3} & \textbf{Editar anotación} \\ \otoprule
	\textbf{Versión} & 2.0 \\ \midrule
	\textbf{Autor} & Jose Manuel Moral Garrido \\ \midrule
	\textbf{Requisitos asociados} & RF-1.1 \\ \midrule
	\textbf{Descripción} & Permite editar la etiqueta de una anotación. \\ \midrule
	\textbf{Precondiciones} & 
	\tabitem Que haya anotaciones.
	\\ \midrule
	\textbf{Acciones} & 
	\enumeratecompacto{
		\item Seleccionar una anotación.
		\item Pulsar sobre editar anotación.
		\item Se muestra un diálogo con la etiqueta actual de la anotación.
		\item El usuario edita el texto.
		\item Pulsar sobre guardar.
		\item El texto de la anotación cambia.
	}
	\\ \midrule
	\textbf{Postcondiciones} & 
	\tabitem La anotación modificada cambia su texto.
	\\ \midrule
	\textbf{Excepciones} & - \\ \midrule
	\textbf{Importancia} & Media \\ 
}


\tablaAncho
{CU-1.1.4 Seleccionar anotación}
{p{2.9cm} X}
{use-case-1.1.4}
{
	\textbf{CU-1.1.4} & \textbf{Seleccionar anotación} \\ \otoprule
	\textbf{Versión} & 2.0 \\ \midrule
	\textbf{Autor} & Jose Manuel Moral Garrido \\ \midrule
	\textbf{Requisitos asociados} & RF-1.1 \\ \midrule
	\textbf{Descripción} & Permite seleccionar una anotación resaltándola. \\ \midrule
	\textbf{Precondiciones} & 
	\tabitem Tiene que existir al menos una anotación.
	\\ \midrule
	\textbf{Acciones} & 
	\enumeratecompacto{
		\item El usuario selecciona una anotación en el modelo (opción 1).
		\item El usuario selecciona una anotación en la lista lateral (opción 2).
		\item La anotación se destaca tanto en el modelo como en la lista lateral.
	}
	\\ \midrule
	\textbf{Postcondiciones} \\ \midrule
	\textbf{Excepciones} & - \\ \midrule	
	\textbf{Importancia} & Media \\ 
}


\tablaAncho
{CU-1.1.5 Deseleccionar anotación}
{p{2.9cm} X}
{use-case-1.1.5}
{
	\textbf{CU-1.1.5} & \textbf{Deseleccionar anotación} \\ \otoprule
	\textbf{Versión} & 2.0 \\ \midrule
	\textbf{Autor} & Jose Manuel Moral Garrido \\ \midrule
	\textbf{Requisitos asociados} & RF-1.1 \\ \midrule
	\textbf{Descripción} & Permite al usuario deseleccionar una anotación. \\ \midrule
	\textbf{Precondiciones} & 
	\tabitem Que haya al menos una anotación seleccionada.
	\\ \midrule
	\textbf{Acciones} & 
	\enumeratecompacto{
		\item Pinchar sobre una anotación en el modelo (opción 1).
		\item Pinchar sobre una anotación en la lista lateral (opción 2).
		\item Pinchar sobre deseleccionar todo en la lista lateral (opción 3).
		\item Si se ha realizado opcion 1 o 2, se deselecciona dicha anotación.
		\item Si se realiza opción 3, se deseleccionan todas las anotaciones.
	}
	\\ \midrule
	\textbf{Postcondiciones} \\ \midrule
	\textbf{Excepciones} & - \\ \midrule
	\textbf{Importancia} & Media \\ 
}


\tablaAncho
{CU-1.2 Gestión de medidas}
{p{2.9cm} X}
{use-case-1.2}
{
	\textbf{CU-1.2} & \textbf{Gestión de medida} \\ \otoprule
	\textbf{Versión} & 1.0 \\ \midrule
	\textbf{Autor} & Jose Manuel Moral Garrido \\ \midrule
	\textbf{Requisitos asociados} & RF-1, RF-1.2.1, RF-1.2.2, RF-1.2.3, RF-1.2.4, RF-1.2.5 \\ \midrule
	\textbf{Descripción} & Permite gestionar las medidas (añadirlas, eliminarlas, \dots). \\ \midrule
	\textbf{Precondiciones} & 
	\tabitem El usuario tiene abierto un modelo.
	\\ \midrule
	\textbf{Acciones} & 
	\enumeratecompacto{
		\item Se muestran las medidas del modelo visualizado.
		\item Se muestran las opciones de añadir, eliminar, editar la etiqueta de y deseleccionar las medidas.
	}
	\\ \midrule
	\textbf{Postcondiciones} & - \\ \midrule
	\textbf{Excepciones} & - \\ \midrule
	\textbf{Importancia} & Alta \\ 
}


\tablaAncho
{CU-1.2.1 Añadir medida}
{p{2.9cm} X}
{use-case-1.2.1}
{
	\textbf{CU-1.2.1} & \textbf{Añadir medida} \\ \otoprule
	\textbf{Versión} & 2.0 \\ \midrule
	\textbf{Autor} & Jose Manuel Moral Garrido \\ \midrule
	\textbf{Requisitos asociados} & RF-1.2 \\ \midrule
	\textbf{Descripción} & Permite al usuario añadir una medida al modelo. \\ \midrule
	\textbf{Precondiciones} & - \\ \midrule
	\textbf{Acciones} & 
	\enumeratecompacto{
		\item El usuario pincha en añadir medida.
		\item El usuario pincha sobre el modelo en la zona donde quiere crear una anotación.
		\item El usuario pincha sobre el modelo en la posición donde quiere que vaya el otro extremo de la medida.
		\item Dos esferas y una raya aparecen en el visor para representar la medida.
		\item Un elemento aparece en un menú para representar la etiqueta de la medida, que será la etiqueta y las unidades.
	}
	\\ \midrule
	\textbf{Postcondiciones} & 
	\tabitem Se añade una medida al modelo.
	\\ \midrule
	\textbf{Excepciones} & 
	\tabitem No se añaden los dos puntos de la medida.
	\\ \midrule
	\textbf{Importancia} & Alta \\ 
}


\tablaAncho
{CU-1.2.2 Eliminar medida}
{p{2.9cm} X}
{use-case-1.2.2}
{
	\textbf{CU-1.2.2} & \textbf{Eliminar medida} \\ \otoprule
	\textbf{Versión} & 2.0 \\ \midrule
	\textbf{Autor} & Jose Manuel Moral Garrido \\ \midrule
	\textbf{Requisitos asociados} & RF-1.2 \\ \midrule
	\textbf{Descripción} & Permite al usuario eliminar una medida. \\ \midrule
	\textbf{Precondiciones} & 
	\tabitem Que exista una medida en el modelo.
	
	\tabitem (opcional) Que una medida se encuentre seleccionada.
	\\ \midrule
	\textbf{Acciones} & 
	\enumeratecompacto{
		\item Seleccionar una medida (opción 1).
		\item Pulsar sobre borrar medida.
		\item Seleccionar una medida (opción 2).
		\item Se borra la medida.
	}
	\\ \midrule
	\textbf{Postcondiciones} & 
	\tabitem Se elimina la medida seleccionada.
	\\ \midrule
	\textbf{Excepciones} & - \\ \midrule
	\textbf{Importancia} & Media \\ 
}


\tablaAncho
{CU-1.2.3 Editar medida}
{p{2.9cm} X}
{use-case-1.2.3}
{
	\textbf{CU-1.2.3} & \textbf{Editar medida} \\ \otoprule
	\textbf{Versión} & 2.0 \\ \midrule
	\textbf{Autor} & Jose Manuel Moral Garrido \\ \midrule
	\textbf{Requisitos asociados} & RF-1.2 \\ \midrule
	\textbf{Descripción} & Permite editar la etiqueta de una medida. \\ \midrule
	\textbf{Precondiciones} & 
	\tabitem Que haya medidas.
	\\ \midrule
	\textbf{Acciones} & 
	\enumeratecompacto{
		\item Seleccionar una medida.
		\item Pulsar sobre editar medida.
		\item Mostrar un diálogo con la etiqueta actual de la medida.
		\item El usuario edita el texto.
		\item Pulsar sobre guardar.
		\item El texto de la medida cambia.
	}
	\\ \midrule
	\textbf{Postcondiciones} & 
	\tabitem La medida modificada cambia su texto.
	\\ \midrule
	\textbf{Excepciones} & - \\ \midrule
	\textbf{Importancia} & Media \\ 
}


\tablaAncho
{CU-1.2.4 Seleccionar medida}
{p{2.9cm} X}
{use-case-1.2.4}
{
	\textbf{CU-1.2.4} & \textbf{Seleccionar medida} \\ \otoprule
	\textbf{Versión} & 2.0 \\ \midrule
	\textbf{Autor} & Jose Manuel Moral Garrido \\ \midrule
	\textbf{Requisitos asociados} & RF-1.2 \\ \midrule
	\textbf{Descripción} & Permite seleccionar una medida resaltándola. \\ \midrule
	\textbf{Precondiciones} & 
	\tabitem Tiene que existir al menos una medida.
	\\ \midrule
	\textbf{Acciones} & 
	\enumeratecompacto{
		\item El usuario selecciona una medida en el modelo (opción 1).
		\item El usuario selecciona una medida en la lista lateral (opción 2).
		\item La medida se destaca tanto en el modelo como en la lista lateral.
	}
	\\ \midrule
	\textbf{Postcondiciones} & - \\ \midrule
	\textbf{Excepciones} & - \\ \midrule
	\textbf{Importancia} & Media \\ 
}


\tablaAncho
{CU-1.2.5 Deseleccionar medida}
{p{2.9cm} X}
{use-case-1.2.5}
{
	\textbf{CU-1.2.5} & \textbf{Deseleccionar medida} \\ \otoprule
	\textbf{Versión} & 2.0 \\ \midrule
	\textbf{Autor} & Jose Manuel Moral Garrido \\ \midrule
	\textbf{Requisitos asociados} & RF-1.2 \\ \midrule
	\textbf{Descripción} & Permite al usuario deseleccionar una medida. \\ \midrule
	\textbf{Precondiciones} & 
	\tabitem Que haya al menos una medida seleccionada.
	\\ \midrule
	\textbf{Acciones} & 
	\enumeratecompacto{
		\item Pinchar sobre una medida en el modelo (opción 1).
		\item Pinchar sobre una medida en la lista lateral (opción 2).
		\item Pinchar sobre deseleccionar todo en la lista lateral (opción 3).
		\item Si se ha realizado opcion 1 o 2, se deselecciona dicha medida.
		\item Si se realiza opción 3, se deseleccionan todas las medidas.
	}
	\\ \midrule
	\textbf{Postcondiciones} & - \\ \midrule
	\textbf{Excepciones} & - \\ \midrule
	\textbf{Importancia} & Media \\ 
}


\tablaAncho
{CU-1.3 Exportar puntos}
{p{2.9cm} X}
{use-case-1.3}
{
	\textbf{CU-1.3} & \textbf{Exportar punto} \\ \otoprule
	\textbf{Versión} & 2.0 \\ \midrule
	\textbf{Autor} & Jose Manuel Moral Garrido \\ \midrule
	\textbf{Requisitos asociados} & RF-1 \\ \midrule
	\textbf{Descripción} & Permite exportar los puntos que se estén visualizando. \\ \midrule
	\textbf{Precondiciones} & - \\ \midrule
	\textbf{Acciones} & 
	\enumeratecompacto{
		\item Pulsar sobre exportar puntos.
		\item Un archivo se descarga.
	}
	\\ \midrule
	\textbf{Postcondiciones} & 
	\tabitem Un archivo nuevo con nuestros puntos se almacena.
	\tabitem Si el archivo es exportado por un alumno, quedará reflejado en archivo almacenado.
	\\ \midrule
	\textbf{Excepciones} & - \\ \midrule
	\textbf{Importancia} & Media \\ 
}


\tablaAncho
{CU-1.4 Importar puntos}
{p{2.9cm} X}
{use-case-1.4}
{
	\textbf{CU-1.4} & \textbf{Importar punto} \\ \otoprule
	\textbf{Versión} & 2.0 \\ \midrule
	\textbf{Autor} & Jose Manuel Moral Garrido \\ \midrule
	\textbf{Requisitos asociados} & RF-1 \\ \midrule
	\textbf{Descripción} & Permite al usuario cargar puntos creados previamente. \\ \midrule
	\textbf{Precondiciones} & 
	\tabitem Tener un archivo correctamente formado con puntos para el modelo.
	\\ \midrule
	\textbf{Acciones} & 
	\enumeratecompacto{
		\item Pulsar sobre importar puntos.
		\item Seleccionar el archivo deseado.
		\item Pulsar sobre abrir.
		\item Las anotaciones y medidas se añaden.
		\item El archivo importado contendrá información de quién la ha realizado.
	}
	\\ \midrule
	\textbf{Postcondiciones} & 
	\tabitem Se añaden añaden nuevas anotaciones y medidas.
	\\ \midrule
	\textbf{Excepciones} &
	\tabitem El archivo está mal formado.
	
	\tabitem El archivo no contiene anotaciones y medidas.
	\\ \midrule	
	\textbf{Importancia} & Media \\ 
}


\tablaAncho
{CU-2 Visor de ejercicios}
{p{2.9cm} X}
{use-case-2}
{	
	\textbf{CU-2} & \textbf{Visor de ejercicios} \\ \otoprule
	\textbf{Versión} & 1.0 \\ \midrule
	\textbf{Autor} & Jose Manuel Moral Garrido \\ \midrule
	\textbf{Requisitos asociados} & RF-2.1, RF-2.2, RF-2.3, RF-2.4, RF-2.5, RF-2.6, RF-1 \\ \midrule
	\textbf{Descripción} & Permite al usuario visualizar un ejercicio y realizar operaciones sobre el mismo. \\ \midrule
	\textbf{Precondiciones} & 
	\tabitem El usuario ha seleccionado un ejercicio.
	\\ \midrule
	\textbf{Acciones} & 
	\enumeratecompacto{
		\item El usuario abre un ejercicio.
		\item Se muestran el ejercicio con las anotaciones y medidas guardadas.
		\item Se da la posibilidad de gestionar tanto anotaciones como medidas y además incluir ejercicios de los alumnos.
	}
	\\ \midrule
	\textbf{Postcondiciones} & - \\ \midrule
	\textbf{Excepciones} & - \\ \midrule
	\textbf{Importancia} & Alta \\ 
}


\tablaAncho
{CU-2.1 Gestión de anotaciones}
{p{2.9cm} X}
{use-case-2.1}
{
	\textbf{CU-1.1} & \textbf{Gestión de anotaciones} \\ \otoprule
	\textbf{Versión} & 2.0 \\ \midrule
	\textbf{Autor} & Jose Manuel Moral Garrido \\ \midrule
	\textbf{Requisitos asociados} & RF-2, RF-2.1.1, RF-2.1.2, RF-2.1.3, RF-2.1.4, RF-2.1.5 \\ \midrule
	\textbf{Descripción} & Permite gestionar las anotaciones (añadirlas, eliminarlas, etc.). \\ \midrule
	\textbf{Precondiciones} & 
	\tabitem El usuario tiene un ejercicio abierto.
	\\ \midrule
	\textbf{Acciones} & 
	\enumeratecompacto{
		\item Se muestran las anotaciones del ejercicio visualizado.
		\item Se muestran las opciones de añadir, eliminar, editar la etiqueta de y deseleccionar las anotaciones.
	}
	\\ \midrule
	\textbf{Postcondiciones} & - \\ \midrule
	\textbf{Excepciones} & - \\ \midrule
	\textbf{Importancia} & Alta \\ 
}


\tablaAncho
{CU-2.1.1 Añadir anotación}
{p{2.9cm} X}
{use-case-2.1.1}
{	
	\textbf{CU-1.1.1} & \textbf{Añadir anotación} \\ \otoprule
	\textbf{Versión} & 2.0 \\ \midrule
	\textbf{Autor} & Jose Manuel Moral Garrido \\ \midrule
	\textbf{Requisitos asociados} & RF-2.1 \\ \midrule
	\textbf{Descripción} & Permite al usuario añadir una anotación al ejercicio. \\ \midrule
	\textbf{Precondiciones} & - \\ \midrule
	\textbf{Acciones} & 
	\enumeratecompacto{
		\item El usuario pincha en añadir anotación.
		\item El usuario pincha sobre el ejercicio en la zona donde quiere crear una anotación.
		\item Una esfera es añadida en el ejercicio para representar la anotación.
		\item Un elemento aparece en un menú para representar la etiqueta de la anotación.
	}
	\\ \midrule
	\textbf{Postcondiciones} & 
	\tabitem Se añade una anotación al ejercicio.
	\\ \midrule
	\textbf{Excepciones} & - \\ \midrule
	\textbf{Importancia} & Alta \\ 
}


\tablaAncho
{CU-2.1.2 Eliminar anotación}
{p{2.9cm} X}
{use-case-2.1.2}
{	
	\textbf{CU-2.1.2} & \textbf{Eliminar anotación} \\ \otoprule
	\textbf{Versión} & 1.0 \\ \midrule
	\textbf{Autor} & Jose Manuel Moral Garrido \\ \midrule
	\textbf{Requisitos asociados} & RF-2.1 \\ \midrule
	\textbf{Descripción} & Permite al usuario eliminar una anotación. \\ \midrule
	\textbf{Precondiciones} & 
	\tabitem Que exista una anotación en el ejercicio.
	
	\tabitem (opcional) Que una anotación se encuentre seleccionada.
	\\ \midrule
	\textbf{Acciones} & 
	\enumeratecompacto{
		\item Seleccionar una anotación (opcional).
		\item Pulsar sobre borrar anotación. Si se ha realizado anterior, saltamos el siguiente paso.
		\item Seleccionar una anotación (opcional).
		\item Se borra la anotación.
	}
	\\ \midrule
	\textbf{Postcondiciones} & 
	\tabitem Se elimina la anotación seleccionada.
	\\ \midrule
	\textbf{Excepciones} & - \\ \midrule
	\textbf{Importancia} & Media \\ 
}


\tablaAncho
{CU-2.1.3 Editar anotación}
{p{2.9cm} X}
{use-case-2.1.3}
{	
	\textbf{CU-2.1.3} & \textbf{Editar anotación} \\ \otoprule
	\textbf{Versión} & 1.0 \\ \midrule
	\textbf{Autor} & Jose Manuel Moral Garrido \\ \midrule
	\textbf{Requisitos asociados} & RF-2.1 \\ \midrule
	\textbf{Descripción} & Permite editar la etiqueta de una anotación. \\ \midrule
	\textbf{Precondiciones} & 
	\tabitem Que haya anotaciones.
	\\ \midrule
	\textbf{Acciones} & 
	\enumeratecompacto{
		\item Seleccionar una anotación.
		\item Pulsar sobre editar anotación.
		\item Se muestra un diálogo con la etiqueta actual de la anotación.
		\item El usuario edita el texto.
		\item Pulsar sobre guardar.
		\item El texto de la anotación cambia.
	}
	\\ \midrule
	\textbf{Postcondiciones} & 
	\tabitem La anotación modificada cambia su texto.
	\\ \midrule
	\textbf{Excepciones} & - \\ \midrule
	\textbf{Importancia} & Media \\ 
}


\tablaAncho
{CU-2.1.4 Seleccionar anotación}
{p{2.9cm} X}
{use-case-2.1.4}
{
	\textbf{CU-2.1.4} & \textbf{Seleccionar anotación} \\ \otoprule
	\textbf{Versión} & 1.0 \\ \midrule
	\textbf{Autor} & Jose Manuel Moral Garrido \\ \midrule
	\textbf{Requisitos asociados} & RF-2.1 \\ \midrule
	\textbf{Descripción} & Permite seleccionar una anotación resaltándola. \\ \midrule
	\textbf{Precondiciones} & 
	\tabitem Tiene que existir al menos una anotación.
	\\ \midrule
	\textbf{Acciones} & 
	\enumeratecompacto{
		\item El usuario selecciona una anotación en el ejercicio (opción 1).
		\item El usuario selecciona una anotación en la lista lateral (opción 2).
		\item La anotación se destaca tanto en el ejercicio como en la lista lateral.
	}
	\\ \midrule
	\textbf{Postcondiciones} & - \\ \midrule
	\textbf{Excepciones} & - \\ \midrule	
	\textbf{Importancia} & Media \\ 
}


\tablaAncho
{CU-2.1.5 Deseleccionar anotación}
{p{2.9cm} X}
{use-case-2.1.5}
{
	\textbf{CU-2.1.5} & \textbf{Deseleccionar anotación} \\ \otoprule
	\textbf{Versión} & 1.0 \\ \midrule
	\textbf{Autor} & Jose Manuel Moral Garrido \\ \midrule
	\textbf{Requisitos asociados} & RF-2.1 \\ \midrule
	\textbf{Descripción} & Permite al usuario deseleccionar una anotación. \\ \midrule
	\textbf{Precondiciones} & 
	\tabitem Que haya al menos una anotación seleccionada.
	\\ \midrule
	\textbf{Acciones} & 
	\enumeratecompacto{
		\item Pinchar sobre una anotación en el ejercicio (opción 1).
		\item Pinchar sobre una anotación en la lista lateral (opción 2).
		\item Pinchar sobre deseleccionar todo en la lista lateral (opción 3).
		\item Si se ha realizado opcion 1 o 2, se deselecciona dicha anotación.
		\item Si se realiza opción 3, se deseleccionan todas las anotaciones.
	}
	\\ \midrule
	\textbf{Postcondiciones} & - \\ \midrule
	\textbf{Excepciones} & - \\ \midrule
	\textbf{Importancia} & Media \\ 
}


\tablaAncho
{CU-2.2 Gestión de medidas}
{p{2.9cm} X}
{use-case-2.2}
{
	\textbf{CU-2.1} & \textbf{Gestión de medidas} \\ \otoprule
	\textbf{Versión} & 1.0 \\ \midrule
	\textbf{Autor} & Jose Manuel Moral Garrido \\ \midrule
	\textbf{Requisitos asociados} & RF-2, RF-2.2.1, RF-2.2.2, RF-2.2.3, RF-2.2.4, RF-2.2.5 \\ \midrule
	\textbf{Descripción} & Permite gestionar las medidas (añadirlas, eliminarlas, \dots). \\ \midrule
	\textbf{Precondiciones} & 
	\tabitem El usuario tiene abierto un ejercicio.
	\\ \midrule
	\textbf{Acciones} & 
	\enumeratecompacto{
		\item Se muestran las medidas del ejercicio visualizado.
		\item Se muestran las opciones de añadir, eliminar, editar la etiqueta de y deseleccionar las medidas.
	}
	\\ \midrule
	\textbf{Postcondiciones} & - \\ \midrule
	\textbf{Excepciones} & - \\ \midrule
	\textbf{Importancia} & Alta \\ 
}


\tablaAncho
{CU-2.2.1 Añadir medida}
{p{2.9cm} X}
{use-case-2.2.1}
{
	\textbf{CU-2.2.1} & \textbf{Añadir medida} \\ \otoprule
	\textbf{Versión} & 1.0 \\ \midrule
	\textbf{Autor} & Jose Manuel Moral Garrido \\ \midrule
	\textbf{Requisitos asociados} & RF-2.2 \\ \midrule
	\textbf{Descripción} & Permite al usuario añadir una medida al modelo. \\ \midrule
	\textbf{Precondiciones} & - \\ \midrule
	\textbf{Acciones} & 
	\enumeratecompacto{
		\item El usuario pincha en añadir medida.
		\item El usuario pincha sobre el ejercicio en la zona donde quiere crear una anotación.
		\item El usuario pincha sobre el ejercicio en la posición donde quiere que vaya el otro extremo de la medida.
		\item Dos esferas y una raya aparecen en el visor para representar la medida.
		\item Un elemento aparece en un menú para representar la etiqueta de la medida, que será la etiqueta y las unidades.
	}
	\\ \midrule
	\textbf{Postcondiciones} & 
	\tabitem Se añade una medida al ejercicio.
	\\ \midrule
	\textbf{Excepciones} & 
	\tabitem No se añaden los dos puntos de la medida.
	\\ \midrule
	\textbf{Importancia} & Alta \\ 
}


\tablaAncho
{CU-2.2.2 Eliminar medida}
{p{2.9cm} X}
{use-case-2.2.2}
{
	\textbf{CU-2.2.2} & \textbf{Eliminar medida} \\ \otoprule
	\textbf{Versión} & 1.0 \\ \midrule
	\textbf{Autor} & Jose Manuel Moral Garrido \\ \midrule
	\textbf{Requisitos asociados} & RF-2.2 \\ \midrule
	\textbf{Descripción} & Permite al usuario eliminar una medida. \\ \midrule
	\textbf{Precondiciones} & 
	\tabitem Que exista una medida en el ejercicio.
	
	\tabitem (opcional) Que una medida se encuentre seleccionada.
	\\ \midrule
	\textbf{Acciones} & 
	\enumeratecompacto{
		\item Seleccionar una medida (opción 1).
		\item Pulsar sobre borrar medida.
		\item Seleccionar una medida (opción 2).
		\item Se borra la medida.
	}
	\\ \midrule
	\textbf{Postcondiciones} & 
	\tabitem Se elimina la medida seleccionada.
	\\ \midrule
	\textbf{Excepciones} & - \\ \midrule
	\textbf{Importancia} & Media \\ 
}


\tablaAncho
{CU-2.2.3 Editar medida}
{p{2.9cm} X}
{use-case-2.2.3}
{
	\textbf{CU-1.2.3} & \textbf{Editar medida} \\ \otoprule
	\textbf{Versión} & 1.0 \\ \midrule
	\textbf{Autor} & Jose Manuel Moral Garrido \\ \midrule
	\textbf{Requisitos asociados} & RF-2.2 \\ \midrule
	\textbf{Descripción} & Permite editar la etiqueta de una medida. \\ \midrule
	\textbf{Precondiciones} & 
	\tabitem Que haya medidas.
	\\ \midrule
	\textbf{Acciones} & 
	\enumeratecompacto{
		\item Seleccionar una medida.
		\item Pulsar sobre editar medida.
		\item Mostrar un diálogo con la etiqueta actual de la medida.
		\item El usuario edita el texto.
		\item Pulsar sobre guardar.
		\item El texto de la medida cambia.
	}
	\\ \midrule
	\textbf{Postcondiciones} & 
	\tabitem La medida modificada cambia su texto.
	\\ \midrule
	\textbf{Excepciones} & - \\ \midrule
	\textbf{Importancia} & Media \\ 
}


\tablaAncho
{CU-2.2.4 Seleccionar medida}
{p{2.9cm} X}
{use-case-2.2.4}
{
	\textbf{CU-2.2.4} & \textbf{Seleccionar medida} \\ \otoprule
	\textbf{Versión} & 1.0 \\ \midrule
	\textbf{Autor} & Jose Manuel Moral Garrido \\ \midrule
	\textbf{Requisitos asociados} & RF-2.2 \\ \midrule
	\textbf{Descripción} & Permite seleccionar una medida resaltándola. \\ \midrule
	\textbf{Precondiciones} & 
	\tabitem Tiene que existir al menos una medida.
	\\ \midrule
	\textbf{Acciones} & 
	\enumeratecompacto{
		\item El usuario selecciona una medida en el ejercicio (opción 1).
		\item El usuario selecciona una medida en la lista lateral (opción 2).
		\item La medida se destaca tanto en el ejercicio como en la lista lateral.
	}
	\\ \midrule
	\textbf{Postcondiciones} & - \\ \midrule
	\textbf{Excepciones} & - \\ \midrule
	\textbf{Importancia} & Media \\ 
}


\tablaAncho
{CU-2.2.5 Deseleccionar medida}
{p{2.9cm} X}
{use-case-2.2.5}
{
	\textbf{CU-2.2.5} & \textbf{Deseleccionar medida} \\ \otoprule
	\textbf{Versión} & 1.0 \\ \midrule
	\textbf{Autor} & Jose Manuel Moral Garrido \\ \midrule
	\textbf{Requisitos asociados} & RF-2.2 \\ \midrule
	\textbf{Descripción} & Permite al usuario deseleccionar una medida. \\ \midrule
	\textbf{Precondiciones} & 
	\tabitem Que haya al menos una medida seleccionada.
	\\ \midrule
	\textbf{Acciones} & 
	\enumeratecompacto{
		\item Pinchar sobre una medida en el ejercicio (opción 1).
		\item Pinchar sobre una medida en la lista lateral (opción 2).
		\item Pinchar sobre deseleccionar todo en la lista lateral (opción 3).
		\item Si se ha realizado opcion 1 o 2, se deselecciona dicha medida.
		\item Si se realiza opción 3, se deseleccionan todas las medidas.
	}
	\\ \midrule
	\textbf{Postcondiciones} & - \\ \midrule
	\textbf{Excepciones} & - \\ \midrule
	\textbf{Importancia} & Media \\ 
}


\tablaAncho
{CU-2.3 Restaurar datos}
{p{2.9cm} X}
{use-case-2.3}
{
	\textbf{CU-2.3} & \textbf{Restaurar datos} \\ \otoprule
	\textbf{Versión} & 1.0 \\ \midrule
	\textbf{Autor} & Jose Manuel Moral Garrido \\ \midrule
	\textbf{Requisitos asociados} & RF-2 \\ \midrule
	\textbf{Descripción} & Permite al usuario restaurar los valores de anotaciones y medidas iniciales (previamente guardados). \\ \midrule
	\textbf{Precondiciones} & 
	\tabitem Que un ejercicio tenga importado los datos del alumno.
	\\ \midrule
	\textbf{Acciones} & 
	\enumeratecompacto{
		\item Pinchar en el botón de restaurar.
		\item Se eliminarán todas las anotaciones y medidas que no pertenezcan al ejercicio previamente guardado (Si no hay ejercicio hecho, se borra todo).
	}
	\\ \midrule
	\textbf{Postcondiciones} & - \\ \midrule
	\tabitem Se recupera el ejercicio guardado.
	\textbf{Excepciones} & - \\ \midrule
	\textbf{Importancia} & Alta \\ 
}


\tablaAncho
{CU-2.4 Cancelar ejercicio}
{p{2.9cm} X}
{use-case-2.4}
{
	\textbf{CU-2.3} & \textbf{Cancelar ejercicio} \\ \otoprule
	\textbf{Versión} & 1.0 \\ \midrule
	\textbf{Autor} & Jose Manuel Moral Garrido \\ \midrule
	\textbf{Requisitos asociados} & RF-2 \\ \midrule
	\textbf{Descripción} & Permite al usuario cancelar un ejercicio abierto y salir de el. \\ \midrule
	\textbf{Precondiciones} & 
	\tabitem Tener un ejercicio abierto.
	\\ \midrule
	\textbf{Acciones} & 
	\enumeratecompacto{
		\item Pinchar en el botón de cancelar.
		\item Si existen cambios sobre ele ejercicio de partida, sale un diálogo preguntado si deseamos guardar los cambios.
		\item Si guardamos lo cambios, el ejercicio se actualiza.
		\item Si no guardamos los cambios, el ejercicio no se modifica.
		\item Si cancelamos, nos quedamos en el visor de ejercicios.
	}
	\\ \midrule
	\textbf{Postcondiciones} & - \\ \midrule
	\tabitem Se vuelve a la lista de ejercicios.
	\textbf{Excepciones} & - \\ \midrule
	\textbf{Importancia} & Alta \\ 
}


\tablaAncho
{CU-2.5 Exportar puntos}
{p{2.9cm} X}
{use-case-2.5}
{
	\textbf{CU-2.5} & \textbf{Exportar punto} \\ \otoprule
	\textbf{Versión} & 2.0 \\ \midrule
	\textbf{Autor} & Jose Manuel Moral Garrido \\ \midrule
	\textbf{Requisitos asociados} & RF-2 \\ \midrule
	\textbf{Descripción} & Permite exportar los puntos que se estén visualizando. \\ \midrule
	\textbf{Precondiciones} & - \\ \midrule
	\textbf{Acciones} & 
	\enumeratecompacto{
		\item Pulsar sobre exportar puntos.
		\item Un archivo se descarga.
	}
	\\ \midrule
	\textbf{Postcondiciones} & 
	\tabitem Un archivo nuevo con nuestros puntos se almacena.
	\\ \midrule
	\textbf{Excepciones} & - \\ \midrule
	\textbf{Importancia} & Media \\ 
}


\tablaAncho
{CU-2.6 Importar puntos}
{p{2.9cm} X}
{use-case-2.6}
{
	\textbf{CU-2.6} & \textbf{Importar punto} \\ \otoprule
	\textbf{Versión} & 2.0 \\ \midrule
	\textbf{Autor} & Jose Manuel Moral Garrido \\ \midrule
	\textbf{Requisitos asociados} & RF-2 \\ \midrule
	\textbf{Descripción} & Permite al usuario cargar puntos creados previamente. \\ \midrule
	\textbf{Precondiciones} & 
	\tabitem Tener un archivo correctamente formado con puntos para el ejercicio.
	\\ \midrule
	\textbf{Acciones} & 
	\enumeratecompacto{
		\item Pulsar sobre importar puntos.
		\item Seleccionar el archivo deseado.
		\item Pulsar sobre abrir.
		\item Las anotaciones y medidas se añaden.
		\item Si las anotaciones eran de un alumno, se importan con un color de esfera distinto.
	}
	\\ \midrule
	\textbf{Postcondiciones} & 
	\tabitem Se añaden nuevas anotaciones y medidas.
	\\ \midrule
	\textbf{Excepciones} &
	\tabitem El archivo está mal formado.
	\tabitem El archivo no contiene anotaciones y medidas.
	\\ \midrule	
	\textbf{Importancia} & Media \\ 
}


\tablaAncho
{CU-3 Listar modelos}
{p{2.9cm} X}
{use-case-3}
{
	\textbf{CU-3} & \textbf{Listar modelos} \\ \otoprule
	\textbf{Versión} & 1.0 \\ \midrule
	\textbf{Autor} & Jose Manuel Moral Garrido \\ \midrule
	\textbf{Requisitos asociados} & RF-3.1 \\ \midrule
	\textbf{Descripción} & Permite al usuario ver los modelos disponibles. \\ \midrule
	\textbf{Precondiciones} & 
	\tabitem Haber accedido a la aplicación correctamente.
	\\ \midrule
	\textbf{Acciones} & 
	\enumeratecompacto{
		\item Entrar en la aplicación.
		\item Elegir la opción de Modelos.
	}
	\\ \midrule
	\textbf{Postcondiciones} &- \\ \midrule
	\textbf{Excepciones} & - \\ \midrule
	\textbf{Importancia} & Alta \\ 
}


\tablaAncho
{CU-3.1 Eliminar modelos}
{p{2.9cm} X}
{use-case-3.1}
{
	\textbf{CU-3.1} & \textbf{Eliminar modelos} \\ \otoprule
	\textbf{Versión} & 1.0 \\ \midrule
	\textbf{Autor} & Jose Manuel Moral Garrido \\ \midrule
	\textbf{Requisitos asociados} & RF-3 \\ \midrule
	\textbf{Descripción} & Permite al usuario eliminar modelos. \\ \midrule
	\textbf{Precondiciones} & 
	\tabitem Haber accedido a la aplicación correctamente.
	\tabitem Tener rol de Profesor.
	\\ \midrule
	\textbf{Acciones} & 
	\enumeratecompacto{
		\item Entrar en la aplicación.
		\item Elegir la opción de Modelos.
		\item Elegir el modelo a eliminar
	}
	\\ \midrule
	\textbf{Postcondiciones} & - \\ \midrule
	\textbf{Excepciones} & - \\ \midrule
	\textbf{Importancia} & Alta \\ 
}


\tablaAncho
{CU-4 Listar ejercicios}
{p{2.9cm} X}
{use-case-4}
{
	\textbf{CU-4} & \textbf{Listar ejercicios} \\ \otoprule
	\textbf{Versión} & 1.0 \\ \midrule
	\textbf{Autor} & Jose Manuel Moral Garrido \\ \midrule
	\textbf{Requisitos asociados} & RF-4.1 \\ \midrule
	\textbf{Descripción} & Permite al usuario listar los ejercicios. \\ \midrule
	\textbf{Precondiciones} & 
	\tabitem Haber accedido a la aplicación correctamente.
	\tabitem Tener rol de Profesor.
	\\ \midrule
	\textbf{Acciones} & 
	\enumeratecompacto{
		\item Entrar en la aplicación.
		\item Elegir la opción de Ejercicios.
		\item Elegir un modelo para ver sus ejercicios.
	}
	\\ \midrule
	\textbf{Postcondiciones} &  - \\ \midrule
	\textbf{Excepciones} & - \\ \midrule
	\textbf{Importancia} & Alta \\ 
}


\tablaAncho
{CU-4.1 Gestión de ejercicios}
{p{2.9cm} X}
{use-case-4.1}
{
	\textbf{CU-4.1} & \textbf{Gestión de ejercicios} \\ \otoprule
	\textbf{Versión} & 1.0 \\ \midrule
	\textbf{Autor} & Jose Manuel Moral Garrido \\ \midrule
	\textbf{Requisitos asociados} & RF-4, RF-4.1.1 ,RF-4.1.2, RF-4.1.3, RF-4.1.4, RF-4.1.5 \\ \midrule
	\textbf{Descripción} & Permite al realizar operaciones sobres los ejercicios. \\ \midrule
	\textbf{Precondiciones} & 
	\tabitem Haber accedido a la aplicación correctamente.
	\tabitem Tener rol de Profesor.
	\\ \midrule
	\textbf{Acciones} & 
	\enumeratecompacto{
		\item Entrar en la aplicación.
		\item Elegir la opción de Ejercicios.
		\item Elegir un modelo para ver sus ejercicios.
		\item Elegir la opción correspondiente.
	}
	\\ \midrule
	\textbf{Postcondiciones} &  - \\ \midrule
	\textbf{Excepciones} & - \\ \midrule
	\textbf{Importancia} & Alta \\
}


\tablaAncho
{CU-4.1.1 Añadir ejercicios}
{p{2.9cm} X}
{use-case-4.1.1}
{
	\textbf{CU-4.1.1} & \textbf{Añadir ejercicios} \\ \otoprule
	\textbf{Versión} & 1.0 \\ \midrule
	\textbf{Autor} & Jose Manuel Moral Garrido \\ \midrule
	\textbf{Requisitos asociados} & RF-4.1 \\ \midrule
	\textbf{Descripción} & Permite añadir un ejercicio. \\ \midrule
	\textbf{Precondiciones} & 
	\tabitem Haber accedido a la aplicación correctamente.
	\tabitem Tener rol de Profesor.
	\\ \midrule
	\textbf{Acciones} & 
	\enumeratecompacto{
		\item Entrar en la aplicación.
		\item Elegir la opción de Ejercicios.
		\item Elegir un modelo para ver sus ejercicios.
		\item Elegir la opción de añadir.
	}
	\\ \midrule
	\textbf{Postcondiciones} &
	\tabitem Se abre el visor de ejercicios para dicho modelo.
	\\ \midrule
	\textbf{Excepciones} & - \\ \midrule
	\textbf{Importancia} & Alta \\
}


\tablaAncho
{CU-4.1.2 Eliminar ejercicios}
{p{2.9cm} X}
{use-case-4.1.2}
{
	\textbf{CU-4.1.2} & \textbf{Eliminar ejercicios} \\ \otoprule
	\textbf{Versión} & 1.0 \\ \midrule
	\textbf{Autor} & Jose Manuel Moral Garrido \\ \midrule
	\textbf{Requisitos asociados} & RF-4.1 \\ \midrule
	\textbf{Descripción} & Permite eliminar un ejercicio. \\ \midrule
	\textbf{Precondiciones} & 
	\tabitem Haber accedido a la aplicación correctamente.
	\tabitem Tener rol de Profesor.
	\\ \midrule
	\textbf{Acciones} & 
	\enumeratecompacto{
		\item Entrar en la aplicación.
		\item Elegir la opción de Ejercicios.
		\item Elegir un modelo para ver sus ejercicios.
		\item Elegir la opción de eliminar (botón de papelera).
		\item Elegimos en el diálogo emergente si queremos seguir adelante.
	}
	\\ \midrule
	\textbf{Postcondiciones} &
	\tabitem Se elimina el ejercicio si elegimos que se elimine en la comprobación.
	\\ \midrule
	\textbf{Excepciones} & \\
	\textbf{Importancia} & Alta \\
}


\tablaAncho
{CU-4.1.3 Modificar ejercicios}
{p{2.9cm} X}
{use-case-4.1.3}
{
	\textbf{CU-4.1.3} & \textbf{Modificar ejercicios} \\ \otoprule
	\textbf{Versión} & 1.0 \\ \midrule
	\textbf{Autor} & Jose Manuel Moral Garrido \\ \midrule
	\textbf{Requisitos asociados} & RF-4.1 \\ \midrule
	\textbf{Descripción} & Permite Modificar un ejercicio. \\ \midrule
	\textbf{Precondiciones} & 
	\tabitem Haber accedido a la aplicación correctamente.
	\tabitem Tener rol de Profesor.
	\\ \midrule
	\textbf{Acciones} & 
	\enumeratecompacto{
		\item Entrar en la aplicación.
		\item Elegir la opción de Ejercicios.
		\item Elegir un modelo para ver sus ejercicios.
		\item Elegir la opción de modificar (botón de llave inglesa).
		\item Se abre el ejercicio guardado correspondiente.
	}
	\\ \midrule
	\textbf{Postcondiciones} &  - \\ \midrule
	\textbf{Excepciones} & - \\ \midrule
	\textbf{Importancia} & Alta \\
}


\tablaAncho
{CU-4.1.4 Editar nombre ejercicios}
{p{2.9cm} X}
{use-case-4.1.4}
{
	\textbf{CU-4.1.4} & \textbf{Editar nombre ejercicios} \\ \otoprule
	\textbf{Versión} & 1.0 \\ \midrule
	\textbf{Autor} & Jose Manuel Moral Garrido \\ \midrule
	\textbf{Requisitos asociados} & RF-4.1 \\ \midrule
	\textbf{Descripción} & Permite editar el nombre de un ejercicio. \\ \midrule
	\textbf{Precondiciones} & 
	\tabitem Haber accedido a la aplicación correctamente.
	\tabitem Tener rol de Profesor.
	\\ \midrule
	\textbf{Acciones} & 
	\enumeratecompacto{
		\item Entrar en la aplicación.
		\item Elegir la opción de Ejercicios.
		\item Elegir un modelo para ver sus ejercicios.
		\item Elegir la opción de modificar (botón de lápiz).
		\item Se un cuadro de edición del nombre del ejercicio.
	}
	\\ \midrule
	\textbf{Postcondiciones} & - \\ \midrule
	\textbf{Excepciones} & - \\ \midrule
	\textbf{Importancia} & Alta \\
}


\tablaAncho
{CU-4.1.5 Guardar ejercicios}
{p{2.9cm} X}
{use-case-4.1.5}
{
	\textbf{CU-4.1.5} & \textbf{Editar nombre ejercicios} \\ \otoprule
	\textbf{Versión} & 1.0 \\ \midrule
	\textbf{Autor} & Jose Manuel Moral Garrido \\ \midrule
	\textbf{Requisitos asociados} & RF-4.1 \\ \midrule
	\textbf{Descripción} & Permite guardar un ejercicio modificado. \\ \midrule
	\textbf{Precondiciones} & 
	\tabitem Haber accedido a la aplicación correctamente.
	\tabitem Tener rol de Profesor.
	\\ \midrule
	\textbf{Acciones} & 
	\enumeratecompacto{
		\item Entrar en la aplicación.
		\item Elegir la opción de Ejercicios.
		\item Elegir un modelo para ver sus ejercicios.
		\item Elegir la opción de modificar (botón de llave inglesa).
		\item Se abre el visor de ejercicios.
		\item Si se realizan modificaciones se pulsa el botón de guardar para salvar el ejercicio.
	}
	\\ \midrule
	\textbf{Postcondiciones} & - \\ \midrule
	\textbf{Excepciones} & - \\ \midrule
	\textbf{Importancia} & Alta \\
}


\tablaAncho
{CU-5 Subir modelo}
{p{2.9cm} X}
{use-case-5}
{
	\textbf{CU-5} & \textbf{Subir modelo} \\ \otoprule
	\textbf{Versión} & 1.0 \\ \midrule
	\textbf{Autor} & Jose Manuel Moral Garrido \\ \midrule
	\textbf{Requisitos asociados} & \\ \midrule
	\textbf{Descripción} & Permite subir modelo. \\ \midrule
	\textbf{Precondiciones} & 
	\tabitem Haber accedido a la aplicación correctamente.
	\tabitem Tener rol de Profesor.
	\\ \midrule
	\textbf{Acciones} & 
	\enumeratecompacto{
		\item Entrar en la aplicación.
		\item Elegir la opción de Subir Modelo.
	}
	\\ \midrule
	\textbf{Postcondiciones} & - \\ \midrule
	\tabitem Elegir modelo a subir.
	\textbf{Excepciones} & - \\ \midrule
	\textbf{Importancia} & Alta \\
}
\apendice{Especificación de diseño}

\section{Introducción}
A continuación especificaremos la manera en la que hemos organizado cada uno de los elementos del proyecto y detallaremos las razones de por qué lo hemos hecho así.

\section{Diseño de datos}
Este apartado junto con el diagrama de clases del apartado~\ref{sec:diseño-arquitectonico} se ha sacado de la primera versión de nuestro proyecto puesto que no se han realizado cambios. A su vez, se han modificado los diagramas de flujo sustituyéndolos por diagramas de secuencia para conocer el \textit{Proceso de login}, \textit{Selección de un Modelo} y \textit{Selección de un Ejercicio}.

\subsection{Puntos}
Necesitamos una convención para que nuestro sistema de importar y exportar puntos funcione correctamente. El archivo \textit{JSON} (ver código \ref{JSON-schema}) se compone por:
\begin{itemize}
	\item \texttt{filename} El nombre del modelo al que pertenece.
	\item \texttt{annotations} Una lista con las diferentes anotaciones.
	\item \texttt{measurements} Una lista con las medidas.
	\item \texttt{timestamp} Una marca de tiempo de cuando se ha validado en el servidor.
	\item \texttt{checksum} Un código de comprobación (\textit{md5} para ser concretos.)
\end{itemize}
Cada una de las anotaciones se compone por una etiqueta y un punto. Las medidas son una etiqueta y dos puntos. Finalmente, los puntos son tres números de tipo \textit{float} (\texttt{x}, \texttt{y}, \texttt{z}).

La estructura de los archivos \textit{JSON} puede verse en el código \ref{JSON-schema}.

\begin{lstlisting}[language=json, float, caption=Esquema JSON, label=JSON-schema]
{"filename": "a filename",
 "annotations": [annotation, ..., annotation],
 "measurements": [measurement, ..., measurement]}

annotation = {"tag": "a tag",
			  points: [point]}

measurement = {"tag": "a tag",
			   points: [point, point]}

point = {"x": float,
		 "y": float,
		 "z": float}
\end{lstlisting}

\section{Diseño procedimental}
En el momento que el servidor está en línea, éste puede aceptar peticiones de inicio de sesión. Cuando un usuario lo solicita, es redirigido a la página de \textit{login}, en el que tendrá que introducir su correo y contraseña de UBUVirtual. Tras ello, el servidor buscará en su base de datos de usuarios si éste está presente en ella. Si no está, redirigiremos al usuarios a la página de \textit{login} anteriormente mencionada. Si pertenece, entonces el servidor lanza una consulta a la \textit{API} de UBUVirtual, para saber si también se encuentra reconocido como usuario en dicha plataforma y, en caso afirmativo, conocer su rol en una asignatura en concreto. Si no consta el correo, o si la contraseña es incorrecta, se redirige a la página de \textit{login}. En caso de que ambas preguntas sean correctas, dependiendo del rol que dicho usuario ocupe, tendrá la posibilidad de acceder únicamente a visualizar un modelo (rol de alumno) o podrá acceder a visualizar un modelo o un ejercicio (rol de profesor). En el caso de que se quiera visualizar un modelo, se redirigirá a la estantería de modelos desde la cual se elegirá el modelo a visualizar. En caso de querer consultar un ejercicio (únicamente el profesor), se realizará la misma mecánica pero esta vez se redirigirá a la estantería de ejercicios y, desde allí, se elegirá el ejercicio que corresponda.

En la figura \ref{fig:login-sequence} se aprecia cómo sucede el proceso de \textit{login}.
\imagen{login-sequence}{Proceso de \textit{login}}{0.9}

Por ejemplo, si el siguiente proceso que queremos realizar es elegir el modelo, se seguirá el diagrama de la figura \ref{fig:select-model-sequence}.
\imagen{select-model-sequence}{Selección de un modelo}{0.9}

Si somos profesor, podremos a su vez elegir un ejercicio tal y como se aprecia en la figura \ref{fig:select-exercise-sequence}.
\imagen{select-exercise-sequence}{Selección de un ejercicio}{0.9}

\section{Diseño arquitectónico}\label{sec:diseño-arquitectonico}
En la figura \ref{fig:class-diagram} el diagrama de clases que define la parte JavaScript del proyecto. La clase \texttt{<<Utils>>} aunque separada del resto, sí está relacionada con las demás clases, aunque para facilitar la comprensión hemos evitado las uniones. Dependen de ella las clases \texttt{<<AnnotationTool>>}, \texttt{<<MeasurementTool>>}, \texttt{<<PointManager>>} y \texttt{<<Scene>>}.
\imagen{class-diagram}{Diagrama de clases de parte JavaScript}{1.0}

Sin embargo, no añadiremos el diagrama de paquetes JavaScript, puesto que solamente existe uno.



\apendice{Documentación técnica de programación}

\section{Introducción}

\section{Estructura de directorios}

\section{Manual del programador}\label{sec:manual-programador}

\subsection{Instalación y configuración de Moodle como API Rest}
Como se mencionó en los aspectos relevantes de la memoria, es necesario configurar \textit{Moodle} para poder utilizarlo como una \textit{API Rest}. A continuación se detalla como configurar la \textit{API}.

En primer lugar debemos tener un usuario con el rol de administrador de la plataforma para poder acceder a la \textit{Administración del sitio} para poder activar los servicios web que por defecto vienen desactivados. Deberemos dirigirnos a \textit{Administración del sitio--Características avanzadas} y habilitar los servicios web. Una vez hecho esto deberemos activar el protocolo \textit{Rest}, que básicamente es el protocolo seguido por una \textit{API Rest}, el cual se accede mediante \textit{Administración del sitio -- Extensiones -- Servicios Web -- Administrar protocolos} y habilitamos dicho protocolo.

A su vez, para que podamos acceder a dichas funcionalidades, además de tener que estar el servicio web y el protocolo activado, lo usuarios deben tener una ficha o \textit{token} el cual los identifique de manera única. Para generar desde nuestra \textit{API} dichos \textit{tokenes} nos dirigiremos a \textit{Administración del sitio -- Extensiones -- Servicios web -- Administrar tokens} y ahí generaremos los tokens para los usuarios. De esta manera, cada usuario tendrá un identificador único para realizar las peticiones correspondientes~\cite{moodle:api-rest-config}.

\subsection{Tratamiento de los roles de los usuarios}
Como se mencionó en los aspectos relevantes de la memoria, inicialmente se tenía una idea de realizar una comprobación de la base de datos para obtener los roles de usuario. Posteriormente se decidió que obtener dichos roles directamente de \textit{UBUVirtual} era más correcto.

Para ellos utilizamos las funciones proporcionadas por \textit{Moodle} para realizar peticiones a nuestra \textit{API Rest} (véase la sección de \textit{Conceptos Teóricos}), que en este caso es UBUVirtual. En dicho listado de funciones (~\cite{moodle:web-service-api-functions}) encontramos la función \textit{core enrol get enrolled users}, la cual nos permitirá conocer los usuarios de la asignatura, así como su rol en la misma y más información variada de cada uno de los participantes. Dicha función nos muestra esta información en forma de diccionario \textit{JSON} desde el que buscaremos al usuario correspondiente para así conocer su rol en la asignatura correspondiente. Dicha información nos es presentada con la estructura definida en la figura~\ref{fig:user-info-JSON}:
\imagen{user-info-JSON}{Estructura de la información proporcionada por la API en formato JSON.}{0.9}

Como se puede apreciar en el campo \textit{roles} nos encontramos con el rol correspondiente del usuario en cuestión, que en este caso es \textit{Profesor} y el id de dicho rol es $3$.

\subsection{Obtención de los modelos}
Ya que la idea inicial era la de obtener los modelos a través de \textit{UBUVirtual} mediante recursos, cabe mencionar como conseguimos obtenerlos aunque la idea no prosperase.

Para ello, recurrimos de nuevo a las funciones \textit{API Rest} siendo esta vez la función \textit{mod resource get resources by courses} la elegida ~\cite{moodle:web-service-api-functions}. Dicha función nos ofrece la información de los recursos presentes en la \textit{API} de UBUVirtual en los cursos correspondientes. En el caso de no seleccionar un curso en concreto, nos devuelve cada uno de los recursos a los que dicho usuario puede acceder. La información resultante tiene la estructura definida en la figura~\ref{fig:JSON-resources}:
\imagen{JSON-resources}{Estructura de la información proporcionada por la API en formato JSON.}{0.9}

De esta manera podemos acceder al nombre de recurso con su correspondiente extensión y comprobar que es del curso correspondiente mediante el campo \textit{course}.

Pero posteriormente, la Universidad de Burgos nos proporcionó un servidor privado, de nombre <<Arquímedes>> en el que podemos desplegar nuestra \textit{API} sin necesitar por ello todo lo mencionado anteriormente acerca de los albergar los modelos como recursos de \textit{UBUVirtual}, ya que podremos albergarlos en nuestro servidor.

\section{Encriptado y desencriptado de los modelos}
Para otorgar seguridad a los modelos, hemos tenido que encriptar los mismos de manera que el modelo que se alberga en el servidor esté modificado de tal manera que alguien ajeno a la \textit{API} que quiera obtener datos o modificar los datos guardados de los modelos sea incapaz.

Con el fin de realizar modificaciones en los modelos 3D para así preservar su seguridad y unicidad, se ha decidido generar una secuencia de números aleatorios con una estructura determinada. De esta manera, conociendo la semilla utilizada para la generación de los números, podremos codificar y decodificar nuestros modelos sin miedo a perder datos importantes de los mismos. Para obtener dichos números aleatorios hemos introducido en el código una función la cual dada un número, nos devuelve otro, con lo cual realizando esta operación un cierto número de veces, obtendremos una secuencia de números <<aleatorios>> (se encuentra entrecomillado porque los valores son aleatorios, pero si se conoce la semilla inicial siempre obtendremos la lista de valores en el mismo orden)\footnote{\url{https://cdsmith.wordpress.com/2011/10/10/build-your-own-simple-random-numbers/}}.

\subsubsection{Obtención de los números aleatorios}\label{sec:numero-aleatorio}
La manera de obtener números aleatorios en el caso de los vértices es la siguiente: \text{$7$ * número aleatorio anterior \% $101$}, mientras que en el caso de las caras obtendremos los valores aleatorios de la siguiente manera: \text{$7$ * número aleatorio anterior \% $11$}. Vemos que la diferencia reside en el valor máximo que puede alcanzar el número aleatorio.

\subsubsection{Para la modificación de los valores de los vértices}
Llegados a este punto tendremos que decidir cómo modificar los valores de los modelos, para lo cual hemos realizado un estudio de los tiempos que tarda el modelo en ser encriptado. La operación a realizar en cada caso será determinada por el tipo de los valores que se vayan a modificar.

Siendo la manera de generar los números aleatorios la mencionada en la sección~\ref{sec:numero-aleatorio} y la manera de modificar los valores de los vértices: \text{valor del vértice * ($2$ * número aleatorio obtenido)}, estaríamos en la encrucijada de elegir la cantidad de operaciones a realizar debido al gran volumen de datos que queremos modificar, por lo tanto hemos obtenido:

\textbf{Para los modelos ASCII:}
\begin{itemize}
	\item Si modificamos todos los vértices que componen al modelo obtenemos un tiempo de: $12.0131$ segundos
	\item Si modificamos solamente los vértices que ocupan posiciones pares del modelo (la mitad de operaciones) obtenemos un tiempo de: $10.2731$ segundos
\end{itemize}

\textbf{Para los modelos Binarios:}
\begin{itemize}
	\item Si modificamos todos los vértices que componen al modelo obtenemos un tiempo de: $2.7190$ segundos
	\item Si modificamos solamente los vértices que ocupan posiciones pares del modelo (la mitad de operaciones) obtenemos un tiempo de: $2.0486$ segundos
\end{itemize}

Una vez conocidos estos datos, decidimos modificar únicamente los valores que ocupan posiciones pares, ya que la encriptación del modelo es suficiente para conservar su seguridad como podemos observar a continuación y el tiempo de codificación es menor:

Teniendo un modelo inicial como el de la figura~\ref{fig:skull-corrected-notencripted}:
\imagen{skull-corrected-notencripted}{Modelo de partida.}{0.9}

Obtendremos un modelo encriptado como el mostrado en la figura~\ref{fig:skull-corrected-encripted}:
\imagen{skull-corrected-encripted}{Modelo de encriptado.}{0.9}

Como se puede apreciar, el modelo encriptado es lo suficientemente difuso como para poder obtener mediciones o datos del mismo en caso de que este fuera robado de la carpeta de almacenamiento de la aplicación. Pero a partir de aquí nos surge el problema relacionado con el redondeo de los decimales, así como de la cantidad de decimales que se devuelven al realizar un \textit{casteo} a otra clase (por ejemplo de \textit{str a float} en \textit{Python}).

Tras realizar las operaciones de codificación del modelo, procedimos a comprobar que los valores obtenidos dividido entre los valores originales nos devolvieran el multiplicando (nuestro número aleatorio~\ref{sec:numero-aleatorio}) y es aquí cuando nos damos cuenta de que no podemos utilizar los valores en coma flotante de los vértices ya que al multiplicar o dividir, los valores de los mismo son corrompidos por los redondeos y el número de decimales. Por ejemplo, para un multiplicando de $0,7$, al dividir el valor obtenido entre el valor inicial obtenemos que el multiplicando es $0.7129$, con lo que podemos concluir que esta no es una manera viable de encriptar los modelos. A continuación mostramos el modelo de la figura~\ref{fig:skull-corrected-notencripted} desencriptado tras modificar sus vértices en la figura~\ref{fig:skull-modified-floats}:
\imagen{skull-modified-floats}{Modelo desencriptado utilizando los valores de los vértices.}{0.9}

\subsubsection{Para la modificación de los valores de las caras}
Tras el resultado nefasto de modificar los valores de los vértices, decidimos que podríamos alterar los valores de las caras formadas por los vértices, las cuales son enteras y no tendremos el problema de los decimales. En este caso, siendo la manera de generar los números aleatorios la mencionada en la sección~\ref{sec:numero-aleatorio} y la manera de modificar los valores de las caras: \text{valor de la cara * (número aleatorio obtenido * $10$) + (número aleatorio obtenido * $100$)} habiendo también realizado un estudio de los tiempos de codificación de los modelos, teniendo en cuanta que el número de caras en el modelo cogido de ejemplo es $248\,999$ mientras que el número de vértices es de $126\,720$:

\textbf{Para los modelos ASCII:}
\begin{itemize}
	\item Si modificamos todos los vértices que componen al modelo obtenemos un tiempo de: $14.1245$ segundos
	\item Si modificamos solamente los vértices que ocupan posiciones múltiplos de cuatro del modelo (la cuarta parte de operaciones) obtenemos un tiempo de: $11.1153$ segundos
\end{itemize}

\textbf{Para los modelos Binarios:}
\begin{itemize}
	\item Si modificamos todos los vértices que componen al modelo obtenemos un tiempo de: $2.9409$ segundos
	\item Si modificamos solamente los vértices que ocupan posiciones múltiplos de cuatro del modelo (la cuarta parte de operaciones) obtenemos un tiempo de: $1.9845$ segundos
\end{itemize}

Una vez conocidos estos datos, decidimos modificar únicamente los valores que ocupan posiciones múltiplos de cuatro. Con esta modificación, la encriptación del modelo es suficiente para conservar su seguridad (el modelo no se llega a mostrar en el navegador ya que no es capaz de dibujarlo) y el tiempo de codificación es menor. De este modo obtenemos tanto una mejor codificación en lo relacionado con la seguridad como de precisión de resultados tras la decodificación. Por lo tanto, concluiremos adjudicando a las \textbf{caras} la encriptación en lugar de a los \textbf{vértices}.

Obtendremos un modelo encriptado como el mostrado en la figura~\ref{fig:skull-corrected-encripted}:
\imagen{skull-corrected-encripted}{Modelo de encriptado.}{0.9}

\subsection{Ejecución de nuestra aplicación en arquimedes}
Hay un script que ejecuta la aplicación. Desde Linux con el comando ssh (usuario y contraseña), en windows con putty y lo mismo. Una vez que se esté ejecutando, accederemos a la aplicación en la dirección url: \url{https://arquimedes.ubu.es/visor3d/}.

\section{Compilación, instalación y ejecución del proyecto}

\section{Pruebas del sistema}

\apendice{Documentación de usuario}

\section{Introducción}

\section{Requisitos de usuarios}

\section{Instalación}

\section{Manual del usuario}
En este apartado, se explicará cada una de las funcionalidades disponibles de nuestra aplicación.

\subsection{Barra de navegación}\label{sec:barra-navegacion}
Este elemento se encontrará en todas las vistas de la aplicación a excepción de la página de \textit{login}. Dicho elemento nos proporcionará una navegación mas rápida y fluida a cada uno de los elementos incluidos en nuestra barra de navegación. A su vez, este elemento nos dotará de una herramienta conocida como \textit{miga de pan(breadcrumb)}. Dicha herramienta nos ayudará a encontrar el camino de vuelta a cualquier pantalla por la que hayamos pasado sin necesidad de volver al inicio y empezar de nuevo. Podemos apreciar en la figura~\ref{fig:barra-nav-main} la barra de navegación inicial, y en la figura~\ref{fig:barra-nav-breadcrumb} la barra de navegación con la utilización de \textit{migas de pan}.
(IMAGEEEEEEEEEEEEEEEEN)
(IMAGEEEEEEEEEEEEEEEEN)

Nuestra barra de navegación diferencia también entre usuarios con diferente rol. Dependiendo de nuestro rol (alumno o profesor), podremos acceder al apartado de subida de modelos (sección~\ref{sec:subir-modelos}). La diferencia entre las barras de navegación según el rol del usuario se pueden apreciar en las figuras~\ref{fig:barra-nav-profesor} y \ref{fig:barra-nav-alumno}.
(IMAGEEEEEEEEEEEEEEEEN)
(IMAGEEEEEEEEEEEEEEEEN)

\subsection{Página de inicio}
Una vez se haya logueado correctamente, según sea su rol en la asignatura correspondiente, se encontrará con dos estructuras diferentes.
Por un lado, si su rol es el de \textbf{profesor}, se topará con dos bloques que distinguen entre \textbf{Modelos} y \textbf{Ejercicios} como se representa en la figura~\ref{fig:main-page-profesor}.
(IMAGEEEEEEEEEEEEEEEEN)

El bloque de \textbf{Modelos} nos llevará a la estantería de los modelos en la que podremos elegir qué modelo visualizar, como se explica en la sección~\ref{sec:}. Por otro lado, el bloque de \textbf{Ejercicios} nos llevará al listado de los modelos disponibles para la realización de ejercicios, como se muestra en la sección~\ref{sec:rep-ejercicios}.

Por otro lado, si su rol es el de \textbf{Alumno}, se encontrará con un único bloque llamado \textbf{Modelos} que nos redirigirá a la estantería de modelos en la que podremos elegir el modelo a visualizar. Sigue la estructura de la figura~\ref{fig:main-page-alumno}.
(IMAAAGEEEEEEEEEEEEEN)

\subsection{Repositorio de ejercicios}\label{sec:rep-ejercicios}
En esta página podremos observar el listado de los modelos disponibles sobre los que podremos realizar ejercicios. Aquí podremos elegir qué modelo utilizar simplemente pinchando sobre este y automáticamente se abrirá una página como la mostrada en la sección~\ref{sec:rep-ejercicios-modelos}. Es entonces cuando encontraremos el listado de ejercicios disponibles para dicho modelo. Esta página sigue la estructura de la figura~\ref{fig:rep-ejercicios}.
(IMAGEEEEEEEEEEEEEEEEN)

\subsection{Repositorio de ejercicios para cada modelos}\label{sec:rep-ejercicios-modelos}
Es en esta página donde encontraremos el listado de ejercicios disponibles para un modelo en concreto. Dicha página tiene la estructura de la figura~\ref{fig:rep-ejercicios-por-modelo}.
(IMAGEEEEEEEEEEEEEEEEN)

Como se puede apreciar, tenemos la imagen del modelo a un lado de la página y al otro el listado de ejercicios disponibles para dicho modelo. Cuando naveguemos por la lista de ejercicios veremos como se ilumina cada ejercicio al paso del ratón como se muestra en la figura~\ref{fig:resaltado-ejercicio}.
(IMAGEEEEEEEEEEEEEEEEN)

A su vez, se resaltan tres botones en el ejercicio correspondiente con distintas funcionalidades. Por un lado tenemos el botón de \textbf{Editar}, el cual nos redirigirá al visor de ejercicios mencionado en la sección~\ref{sec:visor-ejercicios}. Por otro lado tenemos el botón de \textbf{Eliminar} con el nos aparecerá un diálogo de confirmación para la eliminación de dicho ejercicio como el mostrado en la figura~\ref{fig:confirmacion-eliminacion-ejercicio}.
(IMAGEEEEEEEEEEEEEEEEN)

Por último tenemos el botón de \textbf{Editar nombre} con el que podremos editar el nombre del ejercicio a nuestro gusto siempre y cuando respetemos las reglas de nombres correspondientes. Cuando pinchemos en dicho botón nos aparecerá un cuadro como el mostrado en la figura~\ref{fig:dialogo-edicion-nombre}, el cual nos permitirá guardar los cambios o cancelar el proceso.
(IMAGEEEEEEEEEEEEEEEEN)

Si la comvención de nombres no es respetada, nuestra aplicación mostrará una alerta como la de la figura~\ref{fig:alert-nombre}.
(IMAGEEEEEEEEEEEEEEEEN)

\subsection{Visor}
\subsubsection{Acciones comunes en el visor de modelos}
Lo miso que en lo de Alberto.

\subsection{Visor de ejercicios}\label{sec:visor-ejercicios}
Este visor únicamente será visible para los usuarios con rol de \textbf{profesor} debido a que aquí se realizarán las plantillas de corrección para determinados ejercicios. Se compondrá de funcionalidades ampliadas para facilitar al usuario la corrección y visualización de ejercicios.

\subsubsection{Acciones comunes en el visor de ejercicios}
Dicho visor se compondrá de las mismas funcionalidades que en visor de modelos visible para los alumnos (añadir, borrar, editar, etc), con la diferencia que en este no tendremos la posibilidad de importar y exportar modelos. En este caso, los ejercicios realizados por el profesor será almacenados en el servidor, pudiendo editar estos en cualquier momento(como se puede ver en la sección~\ref{sec:rep-ejercicios-modelos}). La diferencia de este visor con respecto al anterior reside en la aparición de dos botones nuevos, uno de <<Guardar>> y otro de <<Salir>>. El visor mencionado tendrá la estructura de la figura~\ref{fig:visor-ejercicios}.
(IMAGEEEEEEEEEEEEEEEEEEEEEEEEEEN)

El botón de guardar simplemente guardará los cambios realizados en las medidas y anotaciones, mientras que el botón de salir nos devolverá a la pantalla de ejercicios para cada modelo en el caso de no haber realizado cambios(sección~\ref{sec:rep-ejercicios-modelos}). Si cuando pulsamos este botón (<<Salir>>) se detectan cambios, nos aparecerá un diálogo de confirmación en la pantalla como el mostrado en la figura4~\ref{fig:confirmacion-salida-visor}.
(IMAGEEEEEEEEEEEEEEEEEEEEEEEEEEN)

De esta manera podremos salir del visor de ejercicios descartando o guardando los cambios, o simplemente cancelar el proceso de salida del ejercicio.

\subsection{Subir modelos}\label{sec:subir-modelos}
Finalmente, nos encontramos con la página de subida de modelos, la cual solamente estará visible en la barra de navegación (sección~\ref{sec:barra-navegacion}) para los usuarios con rol de \textbf{profesor}. Nos dirigiremos al apartado <<Subir>> en la barra de navegación y nos redirigirá a la página de la figura~\ref{fig:subida-modelos}.
(IMAAAAAAGEEEEEEEEEEEEN)

A la hora de subir un modelo, pincharemos en <<Selección de archivo>> y nos aparecerá un diálogo en el que seleccionaremos el modelo con extensión <<\textit{.ply}>> correspondiente. Tras seleccionar el archivo pulsar en aceptar, pincharemos en el botón de \textbf{Subir} y tendremos que esperar a que nos aparezca el cuadro de confirmación como el de la figura~\ref{fig:confirmacion-subida}.
(IMAAAAAAGEEEEEEEEEEEEN)

Si no realizamos la espera correspondiente, la cual puede demorarse en función del tamaño del modelo, la subida de dicho modelo podría verse alterada. Esta alteración se debería a que interrumpimos el proceso de encriptación de los modelos (véase la sección\ref{sec:manual-programador}), pudiendo conllevar a fallos en la carga de los modelos.

Por lo tanto, partiendo de un modelos como el de la figura~\ref{fig:skull-corrected-notencripted}
\imagen{skull-corrected-notencripted}{Visualización del modelo encriptado correctamente.}

Obtendremos un resultado como el de la figura~\ref{fig:skull-not-waiting} por no dejar a la aplicación realizar la encriptación completa.
\imagen{skull-not-waiting}{Visualización del modelo cargado con una encriptación incompleta.}

A su vez, podrían darse casos en los que ni siquiera se termine de subir el modelo elegido.


\bibliographystyle{plain}
\bibliography{bibliografiaAnexos}

\end{document}