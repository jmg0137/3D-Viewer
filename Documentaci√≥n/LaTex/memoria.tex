\documentclass[a4paper,12pt,twoside]{memoir}

% Castellano
\usepackage[spanish,es-tabla]{babel}
\selectlanguage{spanish}
\usepackage[utf8]{inputenc}
\usepackage[T1]{fontenc}
\usepackage{lmodern} % Scalable font
\usepackage{microtype}
\usepackage{placeins}
\usepackage{lscape}

\RequirePackage{booktabs}
\RequirePackage[table]{xcolor}
\RequirePackage{xtab}
\RequirePackage{multirow}

% Links
\usepackage[colorlinks]{hyperref}
\hypersetup{
	allcolors = {red}
}

% Ecuaciones
\usepackage{amsmath}

% Rutas de fichero / paquete
\newcommand{\ruta}[1]{{\sffamily #1}}

% Párrafos
\nonzeroparskip


% Imagenes
\usepackage{graphicx}
\newcommand{\imagen}[3]{
	\begin{figure}
		\centering
		\includegraphics[width=#3\textwidth]{#1}
		\caption{#2}\label{fig:#1}
	\end{figure}
	\FloatBarrier
}

\newcommand{\imagenflotante}[2]{
	\begin{figure}%[!h]
		\centering
		\includegraphics[width=0.9\textwidth]{#1}
		\caption{#2}\label{fig:#1}
	\end{figure}
}



% El comando \figura nos permite insertar figuras comodamente, y utilizando
% siempre el mismo formato. Los parametros son:
% 1 -> Porcentaje del ancho de página que ocupará la figura (de 0 a 1)
% 2 --> Fichero de la imagen
% 3 --> Texto a pie de imagen
% 4 --> Etiqueta (label) para referencias
% 5 --> Opciones que queramos pasarle al \includegraphics
% 6 --> Opciones de posicionamiento a pasarle a \begin{figure}
\newcommand{\figuraConPosicion}[6]{%
  \setlength{\anchoFloat}{#1\textwidth}%
  \addtolength{\anchoFloat}{-4\fboxsep}%
  \setlength{\anchoFigura}{\anchoFloat}%
  \begin{figure}[#6]
    \begin{center}%
      \Ovalbox{%
        \begin{minipage}{\anchoFloat}%
          \begin{center}%
            \includegraphics[width=\anchoFigura,#5]{#2}%
            \caption{#3}%
            \label{#4}%
          \end{center}%
        \end{minipage}
      }%
    \end{center}%
  \end{figure}%
}

%
% Comando para incluir imágenes en formato apaisado (sin marco).
\newcommand{\figuraApaisadaSinMarco}[5]{%
  \begin{figure}%
    \begin{center}%
    \includegraphics[angle=90,height=#1\textheight,#5]{#2}%
    \caption{#3}%
    \label{#4}%
    \end{center}%
  \end{figure}%
}
% Para las tablas
\newcommand{\otoprule}{\midrule [\heavyrulewidth]}
%
% Nuevo comando para tablas pequeñas (menos de una página).
\newcommand{\tablaSmall}[5]{%
 \begin{table}
  \begin{center}
   \rowcolors {2}{gray!35}{}
   \begin{tabular}{#2}
    \toprule
    #4
    \otoprule
    #5
    \bottomrule
   \end{tabular}
   \caption{#1}
   \label{tabla:#3}
  \end{center}
 \end{table}
}

%
% Nuevo comando para tablas pequeñas (menos de una página).
\newcommand{\tablaSmallSinColores}[5]{%
 \begin{table}[H]
  \begin{center}
   \begin{tabular}{#2}
    \toprule
    #4
    \otoprule
    #5
    \bottomrule
   \end{tabular}
   \caption{#1}
   \label{tabla:#3}
  \end{center}
 \end{table}
}

\newcommand{\tablaApaisadaSmall}[5]{%
\begin{landscape}
  \begin{table}
   \begin{center}
    \rowcolors {2}{gray!35}{}
    \begin{tabular}{#2}
     \toprule
     #4
     \otoprule
     #5
     \bottomrule
    \end{tabular}
    \caption{#1}
    \label{tabla:#3}
   \end{center}
  \end{table}
\end{landscape}
}

%
% Nuevo comando para tablas grandes con cabecera y filas alternas coloreadas en gris.
\newcommand{\tabla}[6]{%
  \begin{center}
    \tablefirsthead{
      \toprule
      #5
      \otoprule
    }
    \tablehead{
      \multicolumn{#3}{l}{\small\sl continúa desde la página anterior}\\
      \toprule
      #5
      \otoprule
    }
    \tabletail{
      \hline
      \multicolumn{#3}{r}{\small\sl continúa en la página siguiente}\\
    }
    \tablelasttail{
      \hline
    }
    \bottomcaption{#1}
    \rowcolors {2}{gray!35}{}
    \begin{xtabular}{#2}
      #6
      \bottomrule
    \end{xtabular}
    \label{tabla:#4}
  \end{center}
}

%
% Nuevo comando para tablas grandes con cabecera.
\newcommand{\tablaSinColores}[6]{%
  \begin{center}
    \tablefirsthead{
      \toprule
      #5
      \otoprule
    }
    \tablehead{
      \multicolumn{#3}{l}{\small\sl continúa desde la página anterior}\\
      \toprule
      #5
      \otoprule
    }
    \tabletail{
      \hline
      \multicolumn{#3}{r}{\small\sl continúa en la página siguiente}\\
    }
    \tablelasttail{
      \hline
    }
    \bottomcaption{#1}
    \begin{xtabular}{#2}
      #6
      \bottomrule
    \end{xtabular}
    \label{tabla:#4}
  \end{center}
}

%
% Nuevo comando para tablas grandes sin cabecera.
\newcommand{\tablaSinCabecera}[5]{%
  \begin{center}
    \tablefirsthead{
      \toprule
    }
    \tablehead{
      \multicolumn{#3}{l}{\small\sl continúa desde la página anterior}\\
      \hline
    }
    \tabletail{
      \hline
      \multicolumn{#3}{r}{\small\sl continúa en la página siguiente}\\
    }
    \tablelasttail{
      \hline
    }
    \bottomcaption{#1}
  \begin{xtabular}{#2}
    #5
   \bottomrule
  \end{xtabular}
  \label{tabla:#4}
  \end{center}
}



\definecolor{cgoLight}{HTML}{EEEEEE}
\definecolor{cgoExtralight}{HTML}{FFFFFF}

%
% Nuevo comando para tablas grandes sin cabecera.
\newcommand{\tablaSinCabeceraConBandas}[5]{%
  \begin{center}
    \tablefirsthead{
      \toprule
    }
    \tablehead{
      \multicolumn{#3}{l}{\small\sl continúa desde la página anterior}\\
      \hline
    }
    \tabletail{
      \hline
      \multicolumn{#3}{r}{\small\sl continúa en la página siguiente}\\
    }
    \tablelasttail{
      \hline
    }
    \bottomcaption{#1}
    \rowcolors[]{1}{cgoExtralight}{cgoLight}

  \begin{xtabular}{#2}
    #5
   \bottomrule
  \end{xtabular}
  \label{tabla:#4}
  \end{center}
}


















\graphicspath{ {./img/} }

% Capítulos
\chapterstyle{bianchi}
\newcommand{\capitulo}[2]{
	\setcounter{chapter}{#1}
	\setcounter{section}{0}
	\chapter*{#2}
	\addcontentsline{toc}{chapter}{#2}
	\markboth{#2}{#2}
}

% Apéndices
\renewcommand{\appendixname}{Apéndice}
\renewcommand*\cftappendixname{\appendixname}

\newcommand{\apendice}[1]{
	%\renewcommand{\thechapter}{A}
	\chapter{#1}
}

\renewcommand*\cftappendixname{\appendixname\ }

% Formato de portada
\makeatletter
\usepackage{xcolor}
\newcommand{\tutor}[1]{\def\@tutor{#1}}
\newcommand{\course}[1]{\def\@course{#1}}
\definecolor{cpardoBox}{HTML}{E6E6FF}
\def\maketitle{
  \null
  \thispagestyle{empty}
  % Cabecera ----------------
\noindent\includegraphics[width=\textwidth]{cabecera}\vspace{1cm}%
  \vfill
  % Título proyecto y escudo informática ----------------
  \colorbox{cpardoBox}{%
    \begin{minipage}{.8\textwidth}
      \vspace{.5cm}\Large
      \begin{center}
      \textbf{TFG del Grado en Ingeniería Informática}\vspace{.6cm}\\
      \textbf{\LARGE\@title{}}
      \end{center}
      \vspace{.2cm}
    \end{minipage}

  }%
  \hfill\begin{minipage}{.20\textwidth}
    \includegraphics[width=\textwidth]{escudoInfor}
  \end{minipage}
  \vfill
  % Datos de alumno, curso y tutores ------------------
  \begin{center}%
  {%
    \noindent\LARGE
    Presentado por \@author{}\\ 
    en Universidad de Burgos --- \@date{}\\
    Tutor: \@tutor{}\\
  }%
  \end{center}%
  \null
  \cleardoublepage
  }
\makeatother

\newcommand{\nombre}{Jose Manuel Moral Garrido} %%% cambio de comando

% Datos de portada
\title{Visor 3D v2.0}
\author{\nombre}
\tutor{José Francisco Díez Pastor, Álvar Arnaiz González}
\date{\today}

\begin{document}

\maketitle


\newpage\null\thispagestyle{empty}\newpage


%%%%%%%%%%%%%%%%%%%%%%%%%%%%%%%%%%%%%%%%%%%%%%%%%%%%%%%%%%%%%%%%%%%%%%%%%%%%%%%%%%%%%%%%
\thispagestyle{empty}


\noindent\includegraphics[width=\textwidth]{cabecera}\vspace{1cm}

\noindent D. Álvar Arnaiz González y D. José Francisco Díez Pastor profesores del departamento de Ingeniería Civil, área de Lenguajes y Sistemas Informáticos.

\noindent Expone:

\noindent Que el alumno D. \nombre, con DNI 71301434-P, ha realizado el Trabajo final de Grado en Ingeniería Informática titulado Visor~3D v2.0. 

\noindent Y que dicho trabajo ha sido realizado por el alumno bajo la dirección del que suscribe, en virtud de lo cual se autoriza su presentación y defensa.

\begin{center} %\large
En Burgos, {\large \today}
\end{center}

\vfill\vfill\vfill

% Author and supervisor
\begin{minipage}{0.45\textwidth}
\begin{flushleft} %\large
Vº. Bº. del Tutor:\\[2cm]
D. Álvar Arnaiz González
\end{flushleft}
\end{minipage}
\hfill
\begin{minipage}{0.45\textwidth}
\begin{flushleft} %\large
Vº. Bº. del co-tutor:\\[2cm]
D. José Francisco Díez Pastor
\end{flushleft}
\end{minipage}
\hfill

\vfill

% para casos con solo un tutor comentar lo anterior
% y descomentar lo siguiente
%Vº. Bº. del Tutor:\\[2cm]
%D. nombre tutor


\newpage\null\thispagestyle{empty}\newpage




\frontmatter

% Abstract en castellano
\renewcommand*\abstractname{Resumen}
\begin{abstract}
En este primer apartado se hace una \textbf{breve} presentación del tema que se aborda en el proyecto.
\end{abstract}

\renewcommand*\abstractname{Descriptores}
\begin{abstract}
	Visor 3D, \textit{STL}, \textit{PLY}, osteología, \textit{e-learning}, docencia virtual
\end{abstract}

\clearpage

% Abstract en inglés
\renewcommand*\abstractname{Abstract}
\begin{abstract}
A \textbf{brief} presentation of the topic addressed in the project.
\end{abstract}

\renewcommand*\abstractname{Keywords}
\begin{abstract}
	3D Viewer, \textit{STL}, \textit{PLY}, osteology, \textit{e-learning}
\end{abstract}

\clearpage

% Indices
\tableofcontents

\clearpage

\listoffigures

\clearpage

\listoftables
\clearpage

\mainmatter
\capitulo{1}{Introducción}

Siendo una de la grandes dificultades la de impartir de manera \textit{online} el material docente explicado en clases de prácticas, partiremos de un proyecto que busca enriquecer las herramientas disponibles para dicho fin en el Grado en Historia y Patrimonio.

Nuestro objetivo principal será seguir la estela del proyecto de Alberto Vivar Arribas~\cite{github:alberto-viewer} y aumentar las funcionalidades de esta herramienta, así como pulir las partes actualmente funcionales. Un punto importante será el poder obtener acceso desde cualquier ordenador a nuestra herramienta. Para ello, trataremos de desplegar la aplicación en un servidor accesible para todos los alumnos de la Universidad de Burgos que se encuentren cursando el Grado en Historia y Patrimonio. A su vez, como objetivo clave se tiene el proporcionar seguridad en los modelos 3D de los huesos debido a su unicidad y privacidad, por lo que se adoptarán las medidas necesarias para dicho fin.

Por otro lado, cabe mencionar la importancia de los docentes en nuestra aplicación, por lo que se les tratará de facilitar el uso de la misma enfocándonos en el ámbito de la corrección de ejercicios y comprobación de copias entre alumnos. Así mismo, incluiremos una correcta gestión de roles en nuestra aplicación para que ésta sea capaz de distinguir qué usuario está haciendo uso de ella para poder así proporcionar unas funcionalidades u otras.

Los alumnos no juegan un papel secundario en la aplicación, pero sí es verdad que tienen unas funcionalidades más acotadas en la misma debido a que su mal uso podría dar lugar a la eliminación de algún modelo óseo 3D que pudiera ser irreversible. A su vez, prevendremos que se introduzcan modelos que no tengan que ver con la asignatura, por parte de los alumnos, en la aplicación.

\capitulo{2}{Objetivos del proyecto}

Dentro de nuestros objetivos estará la gestión de roles de usuario otorgando funcionalidades diferentes dependiendo el rol de dicho usuario dentro de la asignatura. También incluiremos en la aplicación una página en la que el profesor de la asignatura pueda corregir los ejercicios docentes de manera sencilla y clara visualmente. Para ello, se utilizará el visor de modelos con el fin comparar las soluciones de los ejercicios (resueltas por el profesor) con las soluciones propuestas por los alumnos a dichos ejercicios.

A su vez, nos centraremos principalmente en proporcionar seguridad a los modelos que estarán albergados en el servidor debido a su unicidad y dificultad de encontrar. Esto es muy importante ya que una vez que estos modelos estén subidos a nuestro servidor, podrán ser accesibles a ciertas amenazas como podría ser su eliminación.

\section{Gestionar roles de usuario}
Previo a explicar el resto de los objetivos deberemos establecer qué funcionalidades tiene cada usuario dependiendo de su rol, aunque en nuestra aplicación definiremos dos clases de usuarios (siendo los posibles usuarios los proporcionados por \textit{Moodle}):

\begin{itemize}
	\item \textbf{Profesor}: Este será el \textit{superusuario} de la aplicación teniendo acceso a todas las funcionalidades posibles que se integren en el proyecto. Podrá acceder a visualizar modelos, realización de ejercicios, subida y eliminado de los modelos al servidor en el que estará alojada la aplicación, etc.
	
	\item \textbf{Cualquier otro usuario}: Este apartado engloba los usuarios de tipo Estudiante, Invitado, Creador del curso, Mánager y Profesor sin permiso de edición, es decir, cualquier usuario que no sea Profesor está en este grupo. Dicho grupo simplemente tiene la opción de visualizar los modelos que se encuentren en la aplicación, así como hacer uso de las herramientas incluidas en el mismo (anotaciones, medidas, importación y exportación de puntos, etc.).
\end{itemize}

\section{Añadir restricciones de escritura para usuarios}
Otro aspecto importante a tener en cuenta dentro de la aplicación es que el usuario puede almacenar nombres de anotaciones, medidas, ejercicios, entre otros, de manera autónoma sin ningún tipo de control. Esto puede suponer un problema ya que si el profesor guarda el nombre de un ejercicio como \texttt{.ejercicio\_numero\_1} podría interpretarse por parte del sistema operativo como que dicho nombre en realidad es una extensión. Dicho problema supondría la eliminación del fichero que alberga el ejercicio y por consiguiente la pérdida del mismo. A su vez, si un alumno o profesor exporta los nombres de anotaciones y medidas incluyendo algún carácter especial como puede ser un <<@>> nos causaría otro problema importante relacionado con la corrección de ejercicios por parte del profesor. Esto es debido a que las anotaciones  de alumnos a la hora de ser exportadas se guardan con ciertos caracteres especiales (los cuales son los restringidos) para que al importarlos, el visor sea capaz de diferenciar entre importaciones de alumnos y profesores.

\section{Posibilitar plataforma de ejercicios para el profesor}
Con el fin de que el profesor pueda realizar pruebas a sus alumnos en la docencia online del grado, se quiere incluir en la aplicación un apartado que permita al profesor realizar y albergar ejercicios propuestos por él mismo. De esta manera, este profesor podrá presentar ejercicios a los alumnos y, tras recibir una solución a dichos ejercicios, poder cotejarlas con los ejercicios previamente resueltos en la aplicación. Dentro de la herramienta de ejercicios se podrá apreciar, en el visor de ejercicios, la cercanía de los puntos elegidos como referencia de los alumnos a los puntos de referencia del profesor así como nombres técnicos de ciertas partes del modelos examinado.

\section{Despliegue de la aplicación en un servidor}
Encontramos también el despliegue de nuestra aplicación en un servidor como objetivo del proyecto ya que queremos que nuestra aplicación pueda servir como herramienta docente para los alumnos de la asignatura de osteología humana del Grado en Historia y Patrimonio. Para ello deberemos configurar un servidor el cual nos será proporcionado por la Universidad de Burgos con el fin de albergar tanto la aplicación como los modelos en sí mismos.

Cabe mencionar la gran complejidad que supone desplegar una aplicación en un servidor real teniendo que realizar previamente un configuración del mismo. Esto es debido a la necesidad de configuración de una máquina cualquiera para ser capaz de ejecutar en nuestra aplicación en un dominio específico.

\section{Dar seguridad a los modelos}
Debido a que los modelos utilizados en el grado son únicos, caros y muy difíciles de encontrar, deberemos dotar a nuestra aplicación de la seguridad necesaria para poder almacenar estos modelos en nuestro servidor. Centraremos nuestro objetivo en dar una encriptación a los modelos que serán subidos al servidor con su posterior desencriptación a la hora de visualizarlos, de manera que estos no puedan ser obtenidos fácilmente desde un navegador web. De esta manera, conseguimos que los modelos queden inservibles en caso de que se intente obtener información de ellos.
\capitulo{3}{Conceptos teóricos}

\section{API Rest}\label{sec:api-rest}
Para comprender como funciona nuestra \textit{API} primero debemos comprender en que consiste una \textit{API Rest}. Una \textit{API Rest} es una arquitectura de desarrollo web basada en el protocolo \textit{HTTP}~\cite{wiki:http}, el cual es un protocolo de acceso sin estado en el que cada mensaje intercambiado tiene la información necesaria para su comprensión sin necesidad de mantener el estado de las comunicaciones.

Se compone de un conjunto de operaciones \textit{CRUD} (create, read, update, delete) con sus métodos \textit{POST, GET, PUT, DELETE, PATCH} correspondientemente, de los cuales se obtendrá información en un determinado formato (en nuestro caso \textit{JSON})

Una manera rápida de comprender como se realizan las peticiones a la \textit{API} de UBUVirtual es comprobar la llamada con este formato:~\ref{fig:curl-rest}
\imagen{curl-rest}{Url de ejemplo de llamada a la API Rest correspondiente}~\cite{moodle:api-rest-config}

\section{Web Server Gateway Interface (WSGI)}\label{sec:wsgi}
Se trata de una especificación para una interfaz simple y universal entre servidores web y aplicaciones (o \textit{frameworks}) para aplicaciones programadas en Python.
De esta manera se tienen dos partes:
\begin{itemize}
	\item La parte del servidor
	\item La parte de la aplicación
\end{itemize}
Para procesar la petición \textit{WSGI}~\cite{wiki:wsgi}, la parte del servidor recibe una petición del cliente y la pasa al \textit{middleware}. Después de ser procesada, esta petición pasa a la parte de la aplicación. La respuesta proporcionada por la parte de la aplicación es transmitida al \textit{middleware}, seguidamente a la parte del servidor y consecutivamente al cliente.

Esta capara de \textit{middleware} puede tener las siguientes funcionalidades:
\begin{itemize}
	\item Relanzar la petición a múltiples objetos de tipo aplicación basados en \textit{URL}.
	\item Permitiendo a múltiples aplicaciones ejecutarse de manera concurrente
	\item Mantener un equilibrio de carga
\end{itemize}
\capitulo{4}{Técnicas y herramientas}

\section{Moodle}\label{sec:moodle-local}
Previo a centrarnos en \textit{Moodle} en particular, tendremos que saber que se trata de un sistema de gestión de aprendizaje (\textit{Learning Management System}). Estos sistemas se utilizan para administrar y gestionar las actividades de formación no presencial, permitiendo un trabajo asíncrono entre los participantes. Dentro de sus competencias encontramos la gestión de usuarios, gestión de recursos o material docente, actividades de formación, seguimiento en el proceso de aprendizaje de los participantes, realizar evaluaciones, foros de consulta, etc.~\cite{wiki:sistemaGestionAprendizaje}. Una vez conocido esto, indaguemos en por qué \textit{Moodle} forma parte de este apartado.

Aunque nuestra aplicación esté orientada a impartir una docencia online desde la \textit{API} de UBUVirtual, cabe mencionar \textit{Moodle} como una herramienta. Esto es debido a que ha sido instalada de manera local con el fin de no tener que depender un super usuario que nos proporcione los recursos necesarios para las pruebas pertienentes (asignaturas, cursos, etc.) que necesitamos en cada caso.

\textit{Moodle} es una herramienta escalable, personalizable, económica y segura a la par que flexible. Dentro del gran número de funcionalidades podemos destacar las siguientes:

\begin{itemize}
	\item Facilidad de uso
	\item Gestión de perfiles de usuario
	\item Facilidad de acceso
	\item Administración sencilla
	\item Realización de exámenes en línea
	\item Gestión de tareas
	\item Implementación de aulas virtuales
\end{itemize}

Moodle ofrece ciertas características de administración, dentro de las cuales entran los roles de usuario, que tienen un papel importante en nuestra \textit{API} (sección~\ref{sec:roles}). Los privilegios de cada uno de los roles son diferentes y cada uno tiene una funcionalidad diferente y es por esto que necesitábamos una instalación local, para poder probar cada uno de los roles~\cite{wiki:moodle}.

\section{Sublime Text}
Para la edición del los diferentes \textit{scripts} utilizaremos este editor de texto ya que, además de ser gratuito (aunque un tanto cargante con solicitar la compra de la versión de pago), es intuitivo y nos proporciona una interfaz cómoda para trabajar,además de una función de auto completar altamente útil.

\section{\TeX Studio}
Como se menciona en \textit{Wikipedia}: "\TeX{} studio es un editor de \LaTeX{} de código abierto y multiplataforma con una interfaz similar a \TeX{} maker". Esta herramienta es un IDE de \LaTeX{} que proporciona soporte de escritura incluyendo la corrección ortográfica, plegado de código y resaltado de sintaxis~\cite{wiki:texstudio}.


\section{JSONMate}
Hemos utilizado esta herramienta, la cual en realidad es una página web que nos ayuda a interpretar la información obtenida en formato \textit{JSON} y simplificarnos su vista~\cite{json:jsnomate}.

\section{PuTTY}
\textit{PuTTY} es un cliente \textit{SSH, Telnet, rlogin, y TCP raw} disponible para \textit{Windows} y en varias plataformas de \textit{Unix} (Versión \textit{Mac OS}). Hemos utilizado esta herramienta para acceder al servidor arquimedes (sección \ref{sec:arquimedes}) desde nuestra máquina con \textit{Windows}, ya que desde \textit{Linux} realizamos las llamadas mediante el comando \texttt{ssh}~\cite{wiki:putty}.

\section{Servidor arquimedes}\label{sec:arquimedes}
Se trata de un servidor proporcionado por la Universidad de Burgos para poder desplegar nuestra aplicación. Este servidor contiene una máquina con un sistema operativo \textit{Ubuntu} 16.04\footnote{\url{https://arquimedes.ubu.es/}}.

\section{Comparativa servidores para desplegar Flask}\label{sec:comparativa-servidores}
Dado que la Universidad de Burgos nos ha dotado con un servidor (ver sección ~\ref{sec:arquimedes}) tenemos que elegir la manera de desplegar nuestra \textit{API Flask}, para lo cual hemos realizado una comparativa con diversas herramientas para desplegarla con el fin de encontrar la mejor manera de hacerlo.

A continuación, mostraremos una tabla con las diferentes herramientas seleccionadas con el fin de elegir la que mejor se amolde a nuestro caso:

\begin{landscape}
	\begin{table}
		\centering
		\caption{Tabla comparativa servidores. ¿Por qué utilizar cada servidor?}
		\label{serverTable}
		\begin{tabular}{p{3.5cm} p{3.5cm} p{3.5cm} p{3.5cm}}
			\toprule
			\textbf{Apache} & \textbf{uWSGI} & \textbf{Stand-Alone (Gunicorn)} & \textbf{Stand-Alone (Twisted Web)} \\
			\midrule
			En el caso de tener experiencia utilizando Apache, así como que se tenga una dependencia del mismo, esto significará \textbf{estabilidad} en el entorno de producción de la aplicación, teniendo gran variedad de módulos estable sy completos. A su vez, es un software muy probado y fiable, teneindo gran variedad de información en la web. & Soporta aplicaciones Python por completo corriendo en WSGI, pudiendo sus componentes realizar muchas más funciones que correr la aplicación, con la correspondiente bajón en el uso de la memoria. Como \textbf{desventaja} hemos considerado que, como está actualmente en desarrollo, podría conllevar a un fallo que aún no se halla contemplado, teniendo a su vez una convención de nombres confusa. & \textbf{Gunicorn}: Si se desea extender de \textit{Apache} utilizando Python(siempre y cuando sea necesario) y programarlo para alguna tarea en concreto. Además, tiene la ventaja de ser sencillo de ejecutar si no se necesita extender de \textit{Apache} & \textbf{Twisted Web}: Si se desea extender de \textit{Apache} utilizando Python(siempre y cuando sea necesario ) siendo simple, estable y maduro. A su vez, puede soportar clientes virtuales.\\
			\bottomrule
		\end{tabular}
	\end{table}
\end{landscape}

Inicialmente escogimos \textit{Gunicorn} como método de despliegue de nuestra aplicación debido a su sencillez y compatibilidad con \textit{Apache} (compatible con extensiones y utilidades). A medida que avanzamos, nos dimos cuenta de que \textit{Gunicorn} no cumplía nuestras expectativas, así como una falta de información de configuración en nuestro caso. Por ello, finalmente nos decantamos por \textit{Apache} (mod\_wsgi) para el despliegue de nuestra aplicación.

\section{<<Librería>> Pyntcloud}
La palabra librería se encuentra entrecomillada ya que no es una biblioteca en sí misma, sino una serie de recursos que en nuestro caso han sido utilizados para la lectura de os archivos \textit{PLY}, tanto formato \textit{ASCII} como \textit{Binario} (ver sección \textbf{3.2 Formato PLY}~\cite{github:alberto-viewer}\footnote{Consultar apartado Conceptos Teóricos del proyecto predecesor al nuestro.})\footnote{\url{https://github.com/daavoo/pyntcloud/blob/master/pyntcloud/io/ply.py}}.

De esta manera, los modelos se leerán tanto en formato \textit{ASCII} como en \textit{Binario} almacenando los vértices y las caras por separado en dos \textit{DataFrames} diferentes. A su vez, diferencia entre los modelos con color a los modelos en escala de grises utilizando estructuras como las que se aprecia en las figuras~\ref{fig:dataframeColor} y~\ref{fig:dataframeNoColor}:
\imagen{dataframeColor}{DataFrame que contiene los vértices y colores.}{0.9}
\imagen{dataframeNoColor}{DataFrame que contiene únicamente los vértices.}{0.7}
\capitulo{5}{Aspectos relevantes del desarrollo del proyecto}

\section{Tratamiento de los roles de los usuarios}\label{sec:roles}
Durante el progreso en la aplicación de partida, nos dimos cuenta de que el tratamiento de los roles podía también ser realizado mediante la cotejación del mismo contra la \textit{API} de UBUVirtual en lugar de tener los roles almacenados en la base de datos junto con los usuarios autorizados. Por otro lado, cabe mencionar que en la aplicación de partida no se estableció ningún tratamiento de roles de usuario, aunque se mencionara.

Inicialmente se decidió llevar a cabo el objetivo inicial de tratamiento de roles, es decir, mediante la definición de los mismos en la base de datos con su posterior consulta a la hora de definir el usuario, aunque después nos dimos cuenta de que la \textit{API} de UBUVirtual podría proporcionarnos estos roles, los cuales vienen dados a cada usuario en la asignatura correspondiente.

\section{Obtención de los Modelos}
La idea inicial de la aplicación era que los modelos proporcionados para su posterior visualización se administraran de manera local, es decir, en una carpeta con todos los modelos. Llegados al punto de pensar cómo podíamos proporcionar al usuario los modelos óseos privados, decidimos que la mejor manera de hacerlo era mediante la administración de dichos modelos como recursos en la \textit{API} de UBUVirtual, siendo estos recursos invisibles para el alumno y a los que solo el profesor tenga acceso para modificar.

\section{Instalación y configuración de Moodle como API Rest}
Para poder realizar las pruebas pertinentes en cada parte del proyecto, hemos decidido que lo ideal es tener instalado \textit{Moodle} de manera local para realizar las llamadas, así como la subida de recursos, asignación de roles, etc, sin tener que depender de un tutor (el cual puede crear una asignatura ficticia y realizar las pruebas ahí). Por ello, hemos realizado la instalación de \textit{Moodle} con un paquete instalador (sección ~\ref{sec:moodle-local}) para \textit{Windows} en el cual viene incluido \textit{XAMPP}~\cite{wiki:xampp} así como \textit{MySql}~\cite{wiki:mysql}.

Una vez instalado \textit{Moodle} y creado un curso y un alumno para poder realizar las pruebas, nos hemos encontrado con el problema de que la \textit{API} no era una \textit{API Rest}~\ref{sec:api-rest} ya que a la hora de realizar las peticiones necesarias para la obtención de información del usuario se nos denegaba el acceso. Para poder solucionar este problema hemos tenido que configurar nuestra \textit{API} cambiando los parámetros correspondientes (véase \textit{Manual del Programador}).

\section{Encriptado y desencriptado de los modelos}
Con la finalidad de obtener seguridad en nuestra \textit{API} en lo relacionado a los modelos decidimos realizar una operación de encriptado y desencriptado de los mismos para que alguien ajeno a la \textit{API} no sea capaz de obtener los modelos o modificarlos.

Para realizar el encriptado realizamos la lectura del modelo y modificamos ciertos valores con el fin de que a la hora de cargar dicho modelo, el cargador de modelos sea capaz de desencriptar el modelo en cuestión y así visualizarlo. Sin embargo nos hemos encontrado con ciertos problemas a la hora de encriptar y desencriptar los modelos (véase \textit{Manual del Programador}).
\capitulo{6}{Trabajos relacionados}

Cabe mencionar en este apartado el congreso \textit{EDULEARN} en el que se presentó la primera versión de nuestro proyecto realizándose una comparativa de diferentes herramientas de visores 3D~\cite{VIVARARRIBAS2017ONL}. Se muestran las diferentes herramientas comparadas en este artículo a continuación:

\tablaSmall{Tabla comparativa de herramientas}
{p{3.5cm} c c c c}
{Comparativa de herramientas}
{
	\textbf{Nombre de la herramienta} & \textbf{Soporte PLY} & \textbf{Escritorio/Web} & \textbf{Software libre}  & \textbf{Activo}\\
}
{
	ViewSTL & No & Web(WebGL) & No(de pago) & Yes \\
	OpenJSCAD & No & Web(WebGL) & Yes & Yes \\
	Mesh Viewer & Yes & Desktop & Yes & No \\
	Open 3D Model Viewer & Yes & Desktop & Yes & Yes \\
	Pointcloud-PLY-Vewer & Yes & Web(Flash) & Yes & No \\
	Online 3D PLY & Yes & Web(WebGL) & No(de pago) & Yes \\
	Autodesk A360 Viewer & Yes & Web(plug-in) & No(de pago) & Yes \\
	Nuestro 3D Viewer & Yes & Web(WebGL) & Yes & Yes \\
}

\capitulo{7}{Conclusiones y Líneas de trabajo futuras}

\section{Conclusiones}

\section{Líneas de trabajo futuras}

\subsection{Integraión de <<MorphoJ>>}
Un aspecto interesante sería la utilización de \textit{MorphoJ} en nuestra aplicación con el fin de obtener datos matemáticamente analizables a partir de la <<forma>> del modelo 3D en cuestión. Esta herramienta sería de gran utilidad si, por ejemplo, nuestra finalidad sería la de analizar matemáticamente un modelo y, a partir de esa respuesta, obtener la mayor información posible.

\subsection{Integración con Moodle}
Por otro lado, otro aspecto interesante de ampliación con nuestra aplicación sería la integración de los servicios ofrecidos en \textit{UBUVirtual}. El más relevante de todos es el sistema de cuestionarios integrado en \textit{Moodle} el cual permitiría a los usuarios realizar otro tipo de ejercicios, así como en un menor número de pasos.

\subsection{Comercialización}
Finalmente, otro punto interesante es el de pensar en monetizar nuestra aplicación con varios enfoques. Por un lado permitir el acceso a otros investigadores o profesores de otras instituciones.
Por otro lado pensaríamos en desplegar nuestro proyecto sobre otros campos con necesidades similares a las cuales se les puedan añadir más características como podría ser el soporte de otros formatos.


\bibliographystyle{plain}
\bibliography{bibliografia}

\end{document}