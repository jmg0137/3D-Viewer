\capitulo{3}{Conceptos teóricos}
En este apartado explicaremos ciertos conceptos que creemos necesarios para la correcta comprensión de este proyecto. Así mismo, se han incluido únicamente los conceptos teóricos nuevos. El resto de conceptos teóricos comunes a la versión anterior pueden consultarse en las páginas 5-8 de la memoria del proyecto de Alberto Vivar Arribas~\cite{github:alberto-viewer}.

\section{API Rest}\label{sec:api-rest}
Para comprender como funciona nuestra \textit{API} primero debemos comprender en que consiste una \textit{API Rest}. Una \textit{API Rest} es una arquitectura de desarrollo web basada en el protocolo \textit{HTTP}~\cite{wiki:http}, el cual es un protocolo de acceso sin estado en el que cada mensaje intercambiado tiene la información necesaria para su comprensión sin necesidad de mantener el estado de las comunicaciones~\cite{apirestbook}.

Se compone de un conjunto de operaciones \textit{CRUD} (create, read, update, delete) con sus métodos \textit{POST, GET, PUT, DELETE, PATCH} correspondientes, de los cuales se obtendrá información en un determinado formato (en nuestro caso \textit{JSON}).

Una manera rápida de comprender como se realizan las peticiones a la \textit{API} de UBUVirtual es comprobar la llamada con el formato que se ve a continuaciín:
\begin{lstlisting}[basicstyle=\tiny,]
https://your.site.com/moodle/webservice/rest/server.php?wstoken=...&wsfunction=...&moodlewsrestformat=json
\end{lstlisting}

\section{Web Server Gateway Interface (WSGI)}\label{sec:wsgi}
Se trata de una especificación para una interfaz simple y universal entre servidores web y aplicaciones (o \textit{frameworks}) para aplicaciones programadas en Python. De esta manera podremos distinguir entre dos grandes bloques:
\begin{itemize}
	\item El lado del servidor
	\item El lado de la aplicación
\end{itemize}
Para procesar la petición \textit{WSGI}~\cite{wiki:wsgi}, el lado del servidor recibe una petición del cliente y la pasa al \textit{middleware}. Después de ser procesada, esta petición pasa a la parte de la aplicación. La respuesta proporcionada del lado de la aplicación es transmitida al \textit{middleware}, que posteriormente reenvía dicha respuestaal lado del servidor y finalmente es reenviada al cliente.

Esta capa de \textit{middleware} puede tener las siguientes funcionalidades:
\begin{itemize}
	\item Relanzar la petición a múltiples objetos de tipo aplicación basados en \textit{URL}.
	\item Permitir a múltiples aplicaciones ejecutarse de manera concurrente.
	\item Mantener un equilibrio de carga.
\end{itemize}