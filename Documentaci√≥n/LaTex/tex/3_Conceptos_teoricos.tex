\capitulo{3}{Conceptos teóricos}

\section{API Rest}\label{sec:api-rest}
Para comprender como funciona nuestra \textit{API} primero debemos comprender en que consiste una \textit{API Rest}. Una \textit{API Rest} es una arquitectura de desarrollo web basada en el protocolo \textit{HTTP}~\cite{wiki:http}, el cual es un protocolo de acceso sin estado en el que cada mensaje intercambiado tiene la información necesaria para su comprensión sin necesidad de mantener el estado de las comunicaciones.

Se compone de un conjunto de operaciones \textit{CRUD} (create, read, update, delete) con sus métodos \textit{POST, GET, PUT, DELETE, PATCH} correspondientemente, de los cuales se obtendrá información en un determinado formato (en nuestro caso \textit{JSON})

Una manera rápida de comprender como se realizan las peticiones a la \textit{API} de UBUVirtual es comprobar la llamada con este formato:~\ref{fig:curl-rest}
\imagen{Url API Rest}{Url de ejemplo de llamada a la API Rest correspondiente}~\cite{moodle:api-rest-config}

\section{Web Server Gateway Interface (WSGI)}\label{sec:wsgi}
Se trata de una especificación para una interfaz simple y universal entre servidores web y aplicaciones (o \textit{frameworks}) para aplicaciones programadas en Python.
De esta manera se tienen dos partes:
\begin{itemize}
	\item La parte del servidor
	\item La parte de la aplicación
\end{itemize}
Para procesar la petición \textit{WSGI}~\cite{wiki:wsgi}, la parte del servidor recibe una petición del cliente y la pasa al \textit{middleware}. Después de ser procesada, esta petición pasa a la parte de la aplicación. La respuesta proporcionada por la parte de la aplicación es transmitida al \textit{middleware}, seguidamente a la parte del servidor y consecutivamente al cliente.

Esta capara de \textit{middleware} puede tener las siguientes funcionalidades:
\begin{itemize}
	\item Relanzar la petición a múltiples objetos de tipo aplicación basados en \textit{URL}.
	\item Permitiendo a múltiples aplicaciones ejecutarse de manera concurrente
	\item Mantener un equilibrio de carga
\end{itemize}

\section{Listas de items}

Existen tres posibilidades:

\begin{itemize}
	\item primer item.
	\item segundo item.
\end{itemize}

\begin{enumerate}
	\item primer item.
	\item segundo item.
\end{enumerate}

\begin{description}
	\item[Primer item] más información sobre el primer item.
	\item[Segundo item] más información sobre el segundo item.
\end{description}
	
\begin{itemize}
\item 
\end{itemize}

\section{Tablas}

Igualmente se pueden usar los comandos específicos de \LaTeX o bien usar alguno de los comandos de la plantilla.

\tablaSmall{Herramientas y tecnologías utilizadas en cada parte del proyecto}{l c c c c}{herramientasportipodeuso}
{ \multicolumn{1}{l}{Herramientas} & App AngularJS & API REST & BD & Memoria \\}{ 
HTML5 & X & & &\\
CSS3 & X & & &\\
BOOTSTRAP & X & & &\\
JavaScript & X & & &\\
AngularJS & X & & &\\
Bower & X & & &\\
PHP & & X & &\\
Karma + Jasmine & X & & &\\
Slim framework & & X & &\\
Idiorm & & X & &\\
Composer & & X & &\\
JSON & X & X & &\\
PhpStorm & X & X & &\\
MySQL & & & X &\\
PhpMyAdmin & & & X &\\
Git + BitBucket & X & X & X & X\\
Mik\TeX{} & & & & X\\
\TeX{}Maker & & & & X\\
Astah & & & & X\\
Balsamiq Mockups & X & & &\\
VersionOne & X & X & X & X\\
} 
