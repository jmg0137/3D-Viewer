\apendice{Documentación de usuario}

\section{Introducción}

\section{Requisitos de usuarios}

\section{Instalación}

\section{Manual del usuario}
En este apartado, se explicará cada una de las funcionalidades disponibles de nuestra aplicación.

\subsection{Barra de navegación}\label{sec:barra-navegacion}
Este elemento se encontrará en todas las vistas de la aplicación a excepción de la página de \textit{login}. Dicho elemento nos proporcionará una navegación mas rápida y fluida a cada uno de los elementos incluidos en nuestra barra de navegación. A su vez, este elemento nos dotará de una herramienta conocida como \textit{miga de pan(breadcrumb)}. Dicha herramienta nos ayudará a encontrar el camino de vuelta a cualquier pantalla por la que hayamos pasado sin necesidad de volver al inicio y empezar de nuevo. Podemos apreciar en la figura~\ref{fig:barra-nav-main} la barra de navegación inicial, y en la figura~\ref{fig:barra-nav-breadcrumb} la barra de navegación con la utilización de \textit{migas de pan}.
(IMAGEEEEEEEEEEEEEEEEN)
(IMAGEEEEEEEEEEEEEEEEN)

Nuestra barra de navegación diferencia también entre usuarios con diferente rol. Dependiendo de nuestro rol (alumno o profesor), podremos acceder al apartado de subida de modelos (sección~\ref{sec:subir-modelos}). La diferencia entre las barras de navegación según el rol del usuario se pueden apreciar en las figuras~\ref{fig:barra-nav-profesor} y \ref{fig:barra-nav-alumno}.
(IMAGEEEEEEEEEEEEEEEEN)
(IMAGEEEEEEEEEEEEEEEEN)

\subsection{Página de inicio}
Una vez se haya logueado correctamente, según sea su rol en la asignatura correspondiente, se encontrará con dos estructuras diferentes.
Por un lado, si su rol es el de \textbf{profesor}, se topará con dos bloques que distinguen entre \textbf{Modelos} y \textbf{Ejercicios} como se representa en la figura~\ref{fig:main-page-profesor}.
(IMAGEEEEEEEEEEEEEEEEN)

El bloque de \textbf{Modelos} nos llevará a la estantería de los modelos en la que podremos elegir qué modelo visualizar, como se explica en la sección~\ref{sec:}. Por otro lado, el bloque de \textbf{Ejercicios} nos llevará al listado de los modelos disponibles para la realización de ejercicios, como se muestra en la sección~\ref{sec:rep-ejercicios}.

Por otro lado, si su rol es el de \textbf{Alumno}, se encontrará con un único bloque llamado \textbf{Modelos} que nos redirigirá a la estantería de modelos en la que podremos elegir el modelo a visualizar. Sigue la estructura de la figura~\ref{fig:main-page-alumno}.
(IMAAAGEEEEEEEEEEEEEN)

\subsection{Repositorio de ejercicios}\label{sec:rep-ejercicios}
En esta página podremos observar el listado de los modelos disponibles sobre los que podremos realizar ejercicios. Aquí podremos elegir qué modelo utilizar simplemente pinchando sobre este y automáticamente se abrirá una página como la mostrada en la sección~\ref{sec:rep-ejercicios-modelos}. Es entonces cuando encontraremos el listado de ejercicios disponibles para dicho modelo. Esta página sigue la estructura de la figura~\ref{fig:rep-ejercicios}.
(IMAGEEEEEEEEEEEEEEEEN)

\subsection{Repositorio de ejercicios para cada modelos}\label{sec:rep-ejercicios-modelos}
Es en esta página donde encontraremos el listado de ejercicios disponibles para un modelo en concreto. Dicha página tiene la estructura de la figura~\ref{fig:rep-ejercicios-por-modelo}.
(IMAGEEEEEEEEEEEEEEEEN)

Como se puede apreciar, tenemos la imagen del modelo a un lado de la página y al otro el listado de ejercicios disponibles para dicho modelo. Cuando naveguemos por la lista de ejercicios veremos como se ilumina cada ejercicio al paso del ratón como se muestra en la figura~\ref{fig:resaltado-ejercicio}.
(IMAGEEEEEEEEEEEEEEEEN)

A su vez, se resaltan tres botones en el ejercicio correspondiente con distintas funcionalidades. Por un lado tenemos el botón de \textbf{Editar}, el cual nos redirigirá al visor de ejercicios mencionado en la sección~\ref{sec:visor-ejercicios}. Por otro lado tenemos el botón de \textbf{Eliminar} con el nos aparecerá un diálogo de confirmación para la eliminación de dicho ejercicio como el mostrado en la figura~\ref{fig:confirmacion-eliminacion-ejercicio}.
(IMAGEEEEEEEEEEEEEEEEN)

Por último tenemos el botón de \textbf{Editar nombre} con el que podremos editar el nombre del ejercicio a nuestro gusto siempre y cuando respetemos las reglas de nombres correspondientes. Cuando pinchemos en dicho botón nos aparecerá un cuadro como el mostrado en la figura~\ref{fig:dialogo-edicion-nombre}, el cual nos permitirá guardar los cambios o cancelar el proceso.
(IMAGEEEEEEEEEEEEEEEEN)

Si la comvención de nombres no es respetada, nuestra aplicación mostrará una alerta como la de la figura~\ref{fig:alert-nombre}.
(IMAGEEEEEEEEEEEEEEEEN)

\subsection{Visor}
\subsubsection{Acciones comunes en el visor de modelos}
Lo miso que en lo de Alberto.

\subsection{Visor de ejercicios}\label{sec:visor-ejercicios}
Este visor únicamente será visible para los usuarios con rol de \textbf{profesor} debido a que aquí se realizarán las plantillas de corrección para determinados ejercicios. Se compondrá de funcionalidades ampliadas para facilitar al usuario la corrección y visualización de ejercicios.

\subsubsection{Acciones comunes en el visor de ejercicios}
Dicho visor se compondrá de las mismas funcionalidades que en visor de modelos visible para los alumnos (añadir, borrar, editar, etc), con la diferencia que en este no tendremos la posibilidad de importar y exportar modelos. En este caso, los ejercicios realizados por el profesor será almacenados en el servidor, pudiendo editar estos en cualquier momento(como se puede ver en la sección~\ref{sec:rep-ejercicios-modelos}). La diferencia de este visor con respecto al anterior reside en la aparición de dos botones nuevos, uno de <<Guardar>> y otro de <<Salir>>. El visor mencionado tendrá la estructura de la figura~\ref{fig:visor-ejercicios}.
(IMAGEEEEEEEEEEEEEEEEEEEEEEEEEEN)

El botón de guardar simplemente guardará los cambios realizados en las medidas y anotaciones, mientras que el botón de salir nos devolverá a la pantalla de ejercicios para cada modelo en el caso de no haber realizado cambios(sección~\ref{sec:rep-ejercicios-modelos}). Si cuando pulsamos este botón (<<Salir>>) se detectan cambios, nos aparecerá un diálogo de confirmación en la pantalla como el mostrado en la figura4~\ref{fig:confirmacion-salida-visor}.
(IMAGEEEEEEEEEEEEEEEEEEEEEEEEEEN)

De esta manera podremos salir del visor de ejercicios descartando o guardando los cambios, o simplemente cancelar el proceso de salida del ejercicio.

\subsection{Subir modelos}\label{sec:subir-modelos}
Finalmente, nos encontramos con la página de subida de modelos, la cual solamente estará visible en la barra de navegación (sección~\ref{sec:barra-navegacion}) para los usuarios con rol de \textbf{profesor}. Nos dirigiremos al apartado <<Subir>> en la barra de navegación y nos redirigirá a la página de la figura~\ref{fig:subida-modelos}.
(IMAAAAAAGEEEEEEEEEEEEN)

A la hora de subir un modelo, pincharemos en <<Selección de archivo>> y nos aparecerá un diálogo en el que seleccionaremos el modelo con extensión <<\textit{.ply}>> correspondiente. Tras seleccionar el archivo pulsar en aceptar, pincharemos en el botón de \textbf{Subir} y tendremos que esperar a que nos aparezca el cuadro de confirmación como el de la figura~\ref{fig:confirmacion-subida}.
(IMAAAAAAGEEEEEEEEEEEEN)

Si no realizamos la espera correspondiente, la cual puede demorarse en función del tamaño del modelo, la subida de dicho modelo podría verse alterada. Esta alteración se debería a que interrumpimos el proceso de encriptación de los modelos (véase la sección\ref{sec:manual-programador}), pudiendo conllevar a fallos en la carga de los modelos.

Por lo tanto, partiendo de un modelos como el de la figura~\ref{fig:skull-corrected-notencripted}
\imagen{skull-corrected-notencripted}{Visualización del modelo encriptado correctamente.}

Obtendremos un resultado como el de la figura~\ref{fig:skull-not-waiting} por no dejar a la aplicación realizar la encriptación completa.
\imagen{skull-not-waiting}{Visualización del modelo cargado con una encriptación incompleta.}

A su vez, podrían darse casos en los que ni siquiera se termine de subir el modelo elegido.