\apendice{Plan de Proyecto Software}

\section{Introducción}

\section{Planificación temporal}
A continuación detallaremos los diferentes objetivos que se han establecido para cada \textit{sprint} y, a su vez, el progreso obtenido en los mismos. Para ello hemos utilizado una metodología ágil denominada \textit{SCRUM}~\cite{schwaber2002agile}. Emplearemos a su vez el gestor de tareas provisto por GitHub y generaremos gráficos \textit{burndown} para el seguimiento de los \textit{sprint}, los cuales son provistos por la extensión \textit{ZenHub}.

\subsection{Sprint 1 - 18--92017/24-09-2017}
En este \textit{sprint} se pretenden instalar las herramientas necesarias para la realización del proyecto, así como la lectura y comprensión de la \textit{memoria} y \textit{anexos} del proyecto de partida~\cite{github:alberto-viewer}. A su vez, se pretende probar los métodos creados de la aplicación de partida para comprobar su correcto funcionamiento.

Podemos observar el progreso del sprint en la figura~\ref{fig:sprint-1}
\imagen{Sprint 1}{Progreso en el sprint 1}

\subsection{Sprint 2 - 25-09-2017/01-10-2017}
En este \textit{sprint} se pretende subir a mi repositorio propio toda la información necesaria para continuar el proyecto y solucionar los fallos encontrados la semana anterior en los distintos métodos de la aplicación. Además se prentende conocer la estructura completa del proyecto con el fin de agilizar las tareas en el momento de la búsqueda de los puntos de la aplicación a corregir o modificar.

Podemos observar el progreso del sprint en la figura~\ref{fig:sprint-2}
\imagen{Sprint 2}{Progreso en el sprint 2}

\subsection{Sprint 3 - 02-10-2017/08-10-2017}
En este \textit{sprint} se pretende cambiar la manera que tiene la aplicación de cotejar los roles de los usuarios (mediante base de datos) a ser cotejados y asignados mediante la \textit{API} de UBUVirtual y e intentar interar la aplicación en un servidor público como es \textit{Heroku}~\cite{wiki:heroku}}

Podemos observar el progreso del sprint en la figura~\ref{fig:sprint-3}
\imagen{Sprint 3}{Progreso en el sprint 3}

\subsection{Sprint 4 - 09-10-2017/15-10-2017}
En este \textit{sprint} se pretende continuar con el intento de integración de la aplicación en \textit{Heroku}. A su vez, se pretende albergar los modelos de manera privada para que los alumnos no puedan acceder a ellos nada más que mediante la plataforma o el medio optado. También instalaremos \textit{Moodle} de manera local para poder realizar pruebas sin depender de un tutor que pueda facilitarnos asignaturas, asignación de roles, subida de recursos, etc.

Podemos observar el progreso del sprint en la figura~\ref{fig:sprint-4}
\imagen{Sprint 4}{Progreso en el sprint 4}

\subsection{Sprint 5 - 16-10-2016/22-10-2016}
En este \textit{sprint} se pretende instalar de nuevo \textit{Moodle} ya que el instalado en el \textit{sprint} anterior no funcionaba correctamente. A su vez continuamos con la integración de la aplicación en \textit{Heroku} (tarea que se está alargando por dos motivos: errores en la estructura de la aplicación y que nos encontramos a la espera de que la UBU nos proporcione un servidor que cumpla ciertos requisitos)

Podemos observar el progreso del sprint en la figura~\ref{fig:sprint-5}
\imagen{Sprint 5}{Progreso en el sprint 5}

\subsection{Sprint 6 - 24-10-2016/29-10-2016}
En este \textit{sprint} se pretende revertir al aplicación a un punto anterior ya que podemos decir que el \textit{sprint} anterior ha sido inservible. Esta vuelta atrás la realizaremos de manera manual ya que si la realizamos mediante los \textit{commit}, perderemos unos cambios que no queremos perder. Este cambio manual también involucra cambiar las dependencias que eran necesarias para el lanzamiento de la aplicación en Heroku (\textit{sprint 5}), ya que en este \textit{sprint} hemos sustituido un servidor público como es Heroku por el proporcionado por la Universidad de Burgos. Por otra parte, tendremos que ejecutar la aplicación en el servidor provisto por la \textit{UBU} (\textit{arquimedes}), realizando a su vez una comparativa de los distintos servidores posibles para desplegar nuestra \textit{API}

Podemos observar el progreso del sprint en la figura~\ref{fig:sprint-6}
\imagen{Sprint 6}{Progreso en el sprint 6}

\subsection{Sprint 7 - 30-10-2016/06-11-2016}
En este \textit{sprint} se pretende trabajar los aspectos relacionados con la seguridad de los modelos subidos al servidor. Con esto queremos decir que en este \textit{sprint} nos dedicaremos a realizar una encriptación de los modelos para que únicamente los usuarios autorizados sean capaces de visualizar el modelo tal y como es. Esto se realiza con el fin de conservar la privacidad de los modelos ya que estos son únicos.
La encriptación se realizará en el momento de la subida del modelo a la aplicación alojada en el servidor, y justo en el momento de visualizar el modelo se realizará su desencriptación. La encriptación se hará para los modelos \textit{PLY} tanto en formato \textbf{binario} como en \textbf{ascii}, mientras que la desencriptación se hará únicamente desde formato \textit{ascii} para así ahorrar tiempo, complejidad y la programación de dos desencriptadores diferentes. A su vez, se realizará un estudio de los tiempos de carga de los modelos debido a las operaciones realizadas para proceder con su encriptación y desencriptación.

Podemos observar el progreso del sprint en la figura~\ref{fig:sprint-7}
\imagen{Sprint 7}{Progreso en el sprint 7}


\section{Estudio de viabilidad}

\subsection{Viabilidad económica}

\subsection{Viabilidad legal}


