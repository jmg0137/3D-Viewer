\apendice{Plan de Proyecto Software}

\section{Introducción}
Cabe mencionar que no he realizado una dedicación a tiempo completo al trabajo de final de grado, sino que ha sido una dedicación parcial. Esto es debido a que me encuentro trabajando al mismo tiempo, por lo tanto, en la mayoría de los \textit{sprint} se han cerrado la mayor parte de las tareas el mismo día ya que mi horario laboral cambia cada semana. Por lo tanto, el día o los días que tengo libres los aprovecho al completo avanzando en el proyecto.

\section{Planificación temporal}
A continuación, detallaremos los diferentes objetivos que se han establecido para cada \textit{sprint} y, a su vez, el progreso obtenido en los mismos. Para ello hemos utilizado una metodología ágil denominada \textit{SCRUM}~\cite{schwaber2002agile}. Emplearemos a su vez el gestor de tareas provisto por GitHub y generaremos gráficos \textit{burndown} para el seguimiento de los \textit{sprint}, los cuales son provistos por la extensión \textit{ZenHub}.

\subsection{Sprint 1 - 18--9-2017/24-09-2017}
En este \textit{sprint} se pretenden instalar las herramientas necesarias para la realización del proyecto, así como la lectura y comprensión de la \textit{memoria} y \textit{anexos} del proyecto de partida~\cite{github:alberto-viewer}. A su vez, se pretende probar los métodos creados de la aplicación de partida para comprobar su correcto funcionamiento.

Podemos observar el progreso del \textit{sprint} en la figura~\ref{fig:sprint-1}
\imagen{sprint-1}{Progreso en el \textit{sprint} 1.}{0.9}

\subsection{Sprint 2 - 25-09-2017/01-10-2017}
En este \textit{sprint} se pretende subir a mi repositorio propio toda la información necesaria para continuar el proyecto y solucionar los fallos encontrados la semana anterior en los distintos métodos de la aplicación. Además se prentende conocer la estructura completa del proyecto con el fin de agilizar las tareas en el momento de la búsqueda de los puntos de la aplicación a corregir o modificar.

Podemos observar el progreso del \textit{sprint} en la figura~\ref{fig:sprint-2}
\imagen{sprint-2}{Progreso en el \textit{sprint} 2.}{0.9}

\subsection{Sprint 3 - 02-10-2017/08-10-2017}
En este \textit{sprint} se pretende cambiar la manera que tiene la aplicación de cotejar los roles de los usuarios (mediante base de datos) a ser cotejados y asignados mediante la \textit{API} de UBUVirtual e intentar interar la aplicación en un servidor público como es \textit{Heroku}~\cite{wiki:heroku}.

Podemos observar el progreso del \textit{sprint} en la figura~\ref{fig:sprint-3}
\imagen{sprint-3}{Progreso en el \textit{sprint} 3.}{0.9}

\subsection{Sprint 4 - 09-10-2017/15-10-2017}
En este \textit{sprint} se pretende continuar con el intento de integración de la aplicación en \textit{Heroku}. A su vez, se pretende albergar los modelos de manera privada para que los alumnos no puedan acceder a ellos nada más que mediante la plataforma. También instalaremos \textit{Moodle} de manera local para poder realizar pruebas sin depender de un tutor que pueda facilitarnos asignaturas, asignación de roles, subida de recursos, etc.

Podemos observar el progreso del \textit{sprint} en la figura~\ref{fig:sprint-4}
\imagen{sprint-4}{Progreso en el \textit{sprint} 4.}{0.9}

\subsection{Sprint 5 - 16-10-2017/22-10-2017}
En este \textit{sprint} se pretende instalar de nuevo \textit{Moodle}, ya que el instalado en el \textit{sprint} anterior no funcionaba correctamente. A su vez continuamos con la integración de la aplicación en \textit{Heroku} (tarea que se está alargando por dos motivos: errores en la estructura de la aplicación y que nos encontramos a la espera de que la UBU nos proporcione un servidor que cumpla ciertos requisitos).

Podemos observar el progreso del \textit{sprint} en la figura~\ref{fig:sprint-5}
\imagen{sprint-5}{Progreso en el \textit{sprint} 5.}{0.9}

\subsection{Sprint 6 - 24-10-2017/29-10-2017}
En este \textit{sprint} se pretende revertir la aplicación a un punto anterior ya que podemos decir que no esperábamos que la UBU nos proporcionara un servidor para la ejecución de nuestra aplicación. Esta vuelta atrás la realizaremos de manera manual ya que si la realizamos mediante los \textit{commit}, perderemos unos cambios que no queremos perder. Este cambio manual también involucra cambiar las dependencias que eran necesarias para el lanzamiento de la aplicación en Heroku (\textit{sprint 5}), ya que en este \textit{sprint} hemos sustituido un servidor público como es Heroku por el proporcionado por la Universidad de Burgos. Por otra parte, tendremos que ejecutar la aplicación en el servidor provisto por la \textit{UBU} (\textit{arquimedes}), realizando a su vez una comparativa de los distintos servidores posibles para desplegar nuestra \textit{API}

Podemos observar el progreso del \textit{sprint} en la figura~\ref{fig:sprint-6}
\imagen{sprint-6}{Progreso en el \textit{sprint} 6.}{0.9}

\subsection{Sprint 7 - 30-10-2017/06-11-2017}
En este \textit{sprint} se pretende trabajar los aspectos relacionados con la seguridad de los modelos subidos al servidor. Con esto queremos decir que en este \textit{sprint} nos dedicaremos a realizar una encriptación de los modelos para que únicamente los usuarios autorizados sean capaces de visualizar el modelo tal y como es. Esto se realiza con el fin de conservar la privacidad de los modelos ya que estos son únicos.
La encriptación se realizará en el momento de la subida del modelo a la aplicación alojada en el servidor, y justo en el momento de visualizar el modelo se realizará su desencriptación. La encriptación se hará para los modelos \textit{PLY} tanto en formato \textbf{binario} como en \textbf{ascii}, mientras que la desencriptación se hará únicamente desde formato \textit{ascii} para así ahorrar tiempo, complejidad y la programación de dos desencriptadores diferentes. A su vez, se realizará un estudio de los tiempos de carga de los modelos debido a las operaciones realizadas para proceder con su encriptación y desencriptación.

En este \textit{sprint} no se ha conseguido integrar el \textit{script} de desencriptación en el cargador de modelos \textit{PLY}, por lo que será una tarea prioritaria en el siguiente \textit{sprint}.

Podemos observar el progreso del \textit{sprint} en la figura~\ref{fig:sprint-7}
\imagen{sprint-7}{Progreso en el \textit{sprint} 7.}{0.9}

\subsection{Sprint 8 - 06-11-2017/12-11-2017}
En este \textit{sprint} trataremos de terminar de hacer funcionar la funcionalidad de encriptación y desencriptación de nuestro visor, ya que en el \textit{sprint} anterior tuvimos problemas con el tema de los números en coma flotante, lo cual será mencionado en el \textit{Manual del Programador}~\ref{sec:manual-programador}. A su vez, dedicaremos este \textit{sprint} a documentar al completo los pasos avanzados hasta el momento, así como solucionar los errores pertinentes a la hora de compilar \LaTeX.
Con el fin de mejorar los tiempos de carga del visor así como la precisión de los modelos a la hora de encriptarlos y desencriptarlos, se estudiarán diferentes métodos de encriptación (vértices, caras, etc). También realizaremos la configuración \textit{VPN} correspondiente para poder conectarnos al servidor proporcionado por la Universidad de Burgos desde otra red diferente a la de la misma.

No hemos conseguido comprobar el funcionamiento de la aplicación en el servidor <<Arquímedes>> debido a que no se han instalado correctamente las herramientas requeridas para el funcionamiento de nuestra \textit{API}, lo cual será objetivo para el siguiente \textit{sprint}.

Podemos observar el progreso del \textit{sprint} en la figura~\ref{fig:sprint-8}
\imagen{sprint-8}{Progreso en el \textit{sprint} 8.}{0.9}

\subsection{Sprint 9 - 13-11-2017/19-11-2017}
En este \textit{sprint} realizaremos la prueba no realizada en el \textit{sprint} anterior (prueba de ejecución de la aplicación en el servidor <<Arquímedes>>). También manipularemos la interfaz gráfica de la aplicación cambiando el estilo de ciertas ventanas al mismo tiempo que modificando y ampliando su funcionalidad. A su vez crearemos nuevas pantallas en nuestra aplicación con el fin de aumentar su funcionalidad y facilitar el uso de la misma al usuario (introducción de migas de pan, icono sugerentes y fáciles de comprender, etc). Añadiremos un apartado completo de ejercicios en el que está pensado que interaccione únicamente el profesor y consista en añadir diferentes soluciones a ejercicios propuesto por el mismo, pudiendo albergar una plantilla de cada uno de los ejercicios de cada modelo disponible con el fin de facilitar la enseñanza y corrección. 

Para este \textit{sprint} no hemos conseguido realizar  los siguientes puntos:
\begin{itemize}
	\item Autocarga de datos (anotaciones y medidas) en el inicio del visor de ejercicios.
	\item Documentación del Manual de Usuario (ya que no hemos completado las pantallas que se corresponden con la interfaz de usuario).
\end{itemize}

Podemos observar el progreso del \textit{sprint} en la figura~\ref{fig:sprint-9}
\imagen{sprint-9}{Progreso en el \textit{sprint} 9.}{0.9}

\subsection{Sprint 10 - 20-11-2017/26-11-2017}
En este \textit{sprint} realizaremos la prueba no realizada en el \textit{sprint} anterior (autocarga de anotacione sy medidas, así como la documentación del Manual de Usuario, pero sin incluir las imágenes de las vistas ya que no tenemos aún las versiones finales de las mismas). Además, trataremos de realizar la configuración del servidor de la Universidad de Burgos con el fin de ejecutar nuestra aplicación en el. No hemos podido avanzar mucho en este \textit{sprint} debido a mi dedicación parcial al proyecto debido s estar trabajando al mismo tiempo.

Para este \textit{sprint} no hemos conseguido realizar  los siguientes puntos:
\begin{itemize}
	\item Llevar a cabo la configuración correspondiente que nos permita ejecutar nuestra \textit{API} en el servidor proporcionado por la Universidad de Burgos.
\end{itemize}

Podemos observar el progreso del \textit{sprint} en la figura~\ref{fig:sprint-10}
\imagen{sprint-10}{Progreso en el \textit{sprint} 10.}{0.9}

\subsection{Sprint 11 - 27-11-2017/03-12-2017}
En este \textit{sprint} realizaremos la configuración del servidor <<Arquímedes>> para posibilitar la ejecución de nuestra aplicación. A su vez, corregiremos errores encontrados en los diferentes botones de la aplicación (carga de puntos, confirmación de eliminación de ejercicio, cancelación de ejercicio ya empezado, etc). También corregiremos los errores acaecidos en la memoria y anexos del proyecto.

Aunque lo hemos intentando fehacientemente, no hemos conseguido realizar la configuración de <<Arquímedes>> debido a la aparición de diversos fallos dicha configuración, como es la pérdida de la imágenes en la ejecución. de nuestras aplicación.

Podemos observar el progreso del \textit{sprint} en la figura~\ref{fig:sprint-11}
\imagen{sprint-11}{Progreso en el \textit{sprint} 11.}{0.9}

\subsection{Sprint 12 - 05-12-2017/10-12-2017}
En este \textit{sprint} daremos prioridad a la configuración de nuestro servidor, aunque nos encontramos con dificultades las cuales no es seguro que puedan ser resueltas. A su vez, realizaremos correcciones en el código (refactorización, cambio de nombres, etc), así como cambios en la interfaz con el fin de que el usuario se sienta cómodo con la aplicación y faciliten su entendimiento. Dentro de estos cambios cabe resaltar el cambio de los colores en las esferas importadas pertenecientes a ejercicios de alumnos, así como cambios en los nombre de las etiquetas de las mismas.

Para este \textit{sprint} no hemos conseguido realizar  los siguientes puntos:
\begin{itemize}
	\item Crear botón de \textit{reset} para restablecer los datos de partida del profesor.
	\item Llevar a cabo la configuración correspondiente que nos permita ejecutar nuestra \textit{API} en el visor <<Arquímedes>> (será único objetivo del siguiente \textit{sprint}).
\end{itemize}

Podemos observar el progreso del \textit{sprint} en la figura~\ref{fig:sprint-12}
\imagen{sprint-12}{Progreso en el \textit{sprint} 12.}{0.9}

\subsection{Sprint 13 - 13-12-2017/17-12-2017}
En este \textit{sprint} nos centraremos únicamente en informarnos a fondo acerca de la <<librería>> de Apache (\textit{mod wsgi}) con la que deseamos conseguir ejecutar nuestra aplicación en el servidor <<Arquímedes>>. A su vez, deshabilitaremos el botón de \textit{logueo} para evitar errores, así como realizaremos los cambios necesarios para solucionar los errores acaecidos en la importación de los mismos datos repetidas veces. Tambié dejaremos solucionada la tarea pendiente de \textit{resetear} los datos del profesor en el visor de ejercicios.

Podemos observar el progreso del \textit{sprint} en la figura~\ref{fig:sprint-13}
\imagen{sprint-13}{Progreso en el \textit{sprint} 13.}{0.9}

\subsection{Sprint 14 - 18-12-2017/24-12-2017}
En este \textit{sprint} nos centraremos en la ejecución de nuestra \textit{API} en el servidor <<Arquímedes>> para lo cual generaremos un \textit{script} que facilite el trabajo de configuración del servidor a nuestro operador. Además, incluiremos expresiones regulares con el fin de limitar los caracteres introducidos por lo alumnos y profesores a la etiquetas de anotaciones y medidas. A su vez, dedicaremos tiempo a la búsqueda de imágenes con las que podamos decorar nuestra interfaz gráfica y así dejar pulida la misma.

Podemos observar el progreso del \textit{sprint} en la figura~\ref{fig:sprint-14}
\imagen{sprint-14}{Progreso en el \textit{sprint} 14.}{0.9}

\subsection{Sprint 15 - 08-01-2018/14-01-2018}
En este \textit{sprint} continuaremos pendientes de la configuración de <<Arquímedes>> para ejecutar nuestra aplicación debido a que durante finales del mes de Diciembre no se ha trabajado en ello (operador del servidor). A su vez nos centraremos en realizar pruebas y corregir errores encontrados, así como elaborar la documentación del proyecto.

Podemos observar el progreso del \textit{sprint} en la figura~\ref{fig:sprint-15}
\imagen{sprint-15}{Progreso en el \textit{sprint} 15.}{0.9}

\subsection{Sprint 16 - 15-01-2018/21-01-2018}
En este \textit{sprint} trataremos de corregir los errores más relevantes de la memoria y anexos y completar éstos en la medida de lo posible. A su vez, realizaremos una reorganización de los diagramas de casos de uso para una mejor visibilidad y comprensión. Así mismo, corregiremos errores de vulnerabilidad en nuestra aplicación. Como último objetivo de este \textit{sprint}, cambiaremos la estética de la página de \textit{login}.

Podemos observar el progreso del \textit{sprint} en la figura~\ref{fig:sprint-16}
\imagen{sprint-16}{Progreso en el \textit{sprint} 16.}{0.9}

\subsection{Sprint 17 - 22-01-2018/28-01-2018}
En este \textit{sprint} solucionaremos los errores en la carga de los ejercicios (se intentan cargar los puntos en el modelo antes de que éste se encuentre completamente cargado). Posteriormente, corregiremos errores en la importación de medidas ya que el color obtenido no es el correcto dependiendo de los roles. Refiriéndonos a la estética de la aplicación, cambiaremos el tono de la barra de navegación para mejor visibilidad y mejoraremos la distribución de los mensajes de error de la página de \textit{login}. Por último, realizaremos la documentación referente al estudio de viabilidad y los diagramas de secuencia para la especificación de diseño.

Podemos observar el progreso del \textit{sprint} en la figura~\ref{fig:sprint-17}
\imagen{sprint-17}{Progreso en el \textit{sprint} 17.}{0.9}

\subsection{Sprint 18 - 30-01-2018/04-02-2018}
En este \textit{sprint} realizaremos la documentación restante y estudiaremos dicha ocumentación en busca de fallos o aspectos a mejorar. Así mismo, trataremos de solucionar errores estéticos de nuestra aplicación, así como problemas con la internacionalización de ciertos mensajes.


\section{Estudio de viabilidad}
En este apartado explicaremos los distintos escenarios en los que aplicar nuestro proyecto para lograr su futuro desarrollo.

\subsection{Viabilidad económica}
A continuación, analizaremos la viabilidad económica de nuestro proyecto detallando los gastos que hubiera supuesto en los diferentes niveles de una empresa.

\subsubsection{Coste de personal}
Los recursos humanos empleados en nuestro proyecto han sido el alumno y los tutores. En primer lugar, el alumno ha empleado 20 semanas a media jornada (20 horas semanales). Suponiendo que el alumno tiene un salario mensual bruto de 700\euro{} , al que restándole los gastos de seguridad social del alumno nos deja 656.04\euro{} netos, lo cual ha sido obtenido a partir de\footnote{\url{http://www.seg-social.es/Internet_1/Trabajadores/CotizacionRecaudaci10777/Basesytiposdecotiza36537/index.htm}}:

\begin{itemize}
	\item Desempleo: 1.55\%
	\item Formación Profesional: 0.03\%
	\item Contingencias Comunes: 4.7\%
\end{itemize}

Esto hace un total de un 6,28\% para el trabajador, por lo que multiplicándolo por el número de meses que ha requerido el proyecto hace un total de 3280.2\euro{} por parte del alumno.

Para calcular el coste de las reuniones con los tutores se conoce que el sueldo base de un ayudante de doctor es de 1815,61 euros mensuales o 21787,32 anuales\footnote{\url{http://www.ubu.es/sites/default/files/portal_page/files/pdi_laboral_2017_2.pdf}}. En total se imparten 24 créditos con un coste anual por crédito de 907,805\euro{}, por lo que teniendo dos tutores que son ayudantes de doctor tendremos:

\[ (907,805\euro{} * 0,5 créditos) + (907,805\euro{} * 0.5 créditos) = 907,805\euro{}\]

Por lo que el total de costes sería de:

\[ 3280,2\euro{} + 907,805\euro{} = 4188,005\euro{} \]

\subsubsection{Coste de hardware}
En primer lugar, mencionamos que no se ha necesitado de ningún equipo informático adicional para la correcta realización del proyecto. No obstante, deberemos calcular el coste \textit{hardware} de los equipo utilizados en nuestro proyecto.

En este caso, se ha desarrollado nuestra aplicación únicamente en un equipo portátil y estimando su valor en 1400\euro{} y suponiendo una vida útli de 10 años y 20 semanas de proyecto, tenemos un coste de amortización de 54\euro{} aproximadamente:
\[ \frac{1400\euro}{10 \ a\textit{ñ}os} * \frac{1 \ a\textit{ñ}o}{52 \ semanas} * 20 \ semanas \approx 54\euro \]

\subsubsection{Coste de software}
Este proyecto es una segunda versión y se ha desarrollado empleando plataformas \textit{software} libre para todo lo imprescindible como son las herramientas de documentación, navegadores, \textit{IDE}, etc. Lo único distinto con respecto a su versión anterior ha sido la integración de la aplicación en un servidor proporcionado por la Universidad de Burgos, el cual tiene un SO gratuito, por lo que el coste \textit{software} es nulo.

\subsubsection{Coste total}
Así, realicemos un cálculo del coste total del proyecto:

\subsection{Viabilidad legal}
Como se mencionó en la versión anterior a nuestro proyecto (~\cite{github:alberto-viewer}), uno de los datos potencialmente sensibles que manejará la aplicación serán los modelos a visualizar. Por ello, tendremos que estar atentos a los posibles cambios con el tipo de licencia que puedan poseer dichos modelos. No por el tipo de fichero ni porque dicho tipo sea propietario --pues es un estándar abierto--, sino por la preferencia de mantener los mencionados modelos a buen recaudo. Además de lo anterior, también almacenaremos unos pocos datos de carácter personal (por el momento solo el correo electrónico institucional).
A su vez, cabe mencionar que la imágenes utilizadas para la estética de la aplicación tienen licencia con derecho a reutilización y a reutilización con modificaciones.

\subsection{Software}
En este apartado no se han realizado variaciones de la versión anterior, por lo que incluiremos tal cual la documentación de la primera versión del proyecto~\cite{github:alberto-viewer}.

Pasamos ahora a analizar la licencia que deberíamos tener en nuestro proyecto. Tendremos que observar las licencias de los elementos que hemos utilizado para su realización.
Podemos observar las diferentes licencias de los componentes del proyecto en ~\ref{tabla:licencias}.

\tablaApaisadaSmall{Licencias utilizadas en el proyecto}
{p{3.5cm} c l c}
{licencias}
{
	\textbf{Dependencia} & \textbf{Versión} & \textbf{Descripción} & \textbf{Licencia} \\
}
{
	three.js & 84.0 & Motor de renderizado en navegadores. & MIT \\
	dat.GUI & 0.62 & Librería para generación de menús. & Apache 2.0 \\
	jQuery & 1.12.4 & Biblioteca para manejo \textit{DOM}. & Apache 2.0 \\
	jQueryUI & 1.12.4 & Biblioteca con elementos \textit{UI} \textit{web}. & Apache 2.0 \\
	rollup & 0.41.6 & Empaquetador de código. & MIT \\
	rollup-plugin-node-resolve & 2.0.0 & \textit{Plugin} de rollup para manejar import. & MIT \\
	rollup-plugin-uglify & 2.0.1 & \textit{Plugin} de rollup para comprimir el código. & MIT \\
	rollup-watch & 3.2.2 & \textit{Plugin} para vigilar el proceso de \textit{minifying}. & MIT \\
	uglify-es & 3.0.15 & \textit{Plugin} para comprimir código con formato \textit{ES6}. & BSD-2-Clause \\
	jsdoc & 3.4.3 & Herramienta para extraer documentación de código JavaScript. &  Apache-2.0 \\
	pydocstyle & 2.0.0 & Herramienta para extraer documentación de código Python. & MIT \\
	Flask & 0.12.1 & \textit{Microframework} Python para servidor. & BSD \\
	Flask-WTF & 0.14.2 & \textit{Plugin} Flask para soportar formularios. & BSD \\
	Flask-Login & 0.4.0 & \textit{Plugin} Flask para restringir acceso de usuarios. & MIT \\
	Flask-Babel & 0.11.2 & \textit{Plugin} Flask para soportar i18n y i10n. & BSD-3-Clause \\
	Jinja2 & 2.9.6 & Motor de plantillas para Python. & BSD \\
	requests & 2.14.2 & Librería para realizar peticiones \textit{HTTP}. & Apache 2.0 \\
	Sphinx & 1.6.1 & Herramienta para generar documentación. & PSFL (BSD + GPL) \\
	bitarray & 0.8.1 & Librería para manejo de \textit{arrays} binarios. & PSFL (BSD + GPL) \\
}

Analizando la tabla, vemos que el conjunto de las licencias está en MIT, Apache 2.0, BSD-2-Clause, BSD-3-Clause, BSD. Las licencias BSD y es más restrictiva que su hija la BSD-3-Clause, y esta a su vez más restrictiva que la BSD-2-Clause. Simplemente se diferencian en no poder emplear el nombre de la licencia para promover productos derivados sin permiso y en la obligatoriedad de mencionar la autoría del software en el que te bases para realizar tu proyecto. Pero para más simplicidad, evitaremos incluirlas en la figura~\ref{fig:comparativa-licencias}. De izquierda a derecha, las hemos ordenado de más permisivas a más restrictivas.

\imagen{comparativa-licencias}{Compatibilidad entre licencias del proyecto}{0.9}