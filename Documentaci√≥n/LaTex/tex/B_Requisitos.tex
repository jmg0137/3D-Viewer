\apendice{Especificación de Requisitos}

\section{Introducción}
Se expondrán en esta sección los distintos objetivos que la aplicación debe lograr, así como los requisitos que debe cumplir.

\section{Objetivos generales}
Como objetivos generales de nuestro proyecto tendremos:
\begin{itemize}
	\item Diferenciar en nuestra aplicación entre los diferentes roles de usuario, cada uno con sus distintas funcionalidades.
	\item Añadir restricciones en el uso de caracteres para anotaciones, medidas y ejercicios.
	\item Facilitar la navegabilidad durante el uso de la aplicación.
	\item Creación de una interfaz que permita al profesor facilidad para corregir ejercicios de alumnos.
	\item Proporcionar seguridad para los modelos debido a su unicidad a la hora de ser alojados en el servidor web.
	\item Desplegar nuestra aplicación en un servidor web.
\end{itemize}

\section{Catalogo de requisitos}
A continuación, listaremos el conjunto de requisitos funcionales extraídos a partir de los objetivos generales del proyecto.

\subsection{Requisitos funcionales}
\begin{itemize}
	\item 
\end{itemize}

\subsection{Requisitos no funcionales}\\
\begin{itemize}
	\item \textbf{RF-1 Gestión de roles}: la aplicación debe ser capaz de diferenciar entre roles de usuarios.
	\begin{itemize}
		\item \textbf{RF-1.1 Exportación de puntos}: la aplicación debe ser capaz de exportar la medidas y anotaciones hechas en un modelos diferenciando entre roles.
		\item \textbf{RF-1.2 Importación de puntos}: la aplicación debe ser capaz de importar la medidas y anotaciones hechas en un modelos diferenciando entre roles.
	\end{itemize}
	\item \textbf{RF-2 Gestión de ejercicios}: la aplicación debe ser capaz de manejar un listado de ejercicios de cada modelo.
	\begin{itemize}
		\item \textbf{RF-2.1 Añadir ejercicios}: el usuario debe poder añadir un ejercicio.
		\item \textbf{RF-2.2 Eliminar ejercicios}: el usuario debe poder eliminar un ejercicio.
		\item \textbf{RF-2.3 Modificar ejercicios}: el usuario debe poder modificar un ejercicio.
		\item \textbf{RF-2.4 Editar nombre ejercicio}: el usuario debe poder editar el nombre de un ejercicio.
		\item \textbf{RF-2-5 Salvado de ejercicio}: el usuario debe poder guardar los cambios acaecidos en el ejercicio.
		\item \textbf{RF-2-5 Visualizar ejercicio}: el usuario debe poder visualizar el ejercicio en cuestión.
	\end{itemize}
	\item \textbf{RF-3 Gestión de modelos}: la aplicación debe ser capaz de mostrar de una pasada los diferentes modelos disponibles.
	\begin{itemize}
		\item \textbf{RF-3.1 Añadir modelos}: la aplicación debe ser capaz de encriptar y desencriptar los modelos para aumentar la seguridad de estos.
	\end{itemize}
\end{itemize}

\subsection{Requisitos no funcionales}
\begin{itemize}
	\item \textbf{RNF-1 Usabilidad}: el conjunto de elementos visuales de la interfaz deben ser intuitivos y conocidos por el usuario medio, permitiendo un rápido aprendizaje.
	\item \textbf{RNF-2 Mantenibilidad}: la aplicación debe desarrollarse siguiendo alguna técnica que permita facilidad de mantenimiento e incorporación de nuevas características, así como corrección de errores.
	\item \textbf{RNF-3 Soporte}: la aplicación debe poder emplearse sobre un amplio conjunto de navegadores.
	\item \textbf{RNF-4 Internacionalización}: la aplicación debe estar diseñada para soportar diferentes idiomas.
	\item \textbf{RNF-5 Control de acceso}: la aplicación debe soportar control de acceso de usuarios.
	\item \textbf{RNF-6 Despliegue}: la aplicación debe estar desplegada de manera que sea accesible sin necesidad de una ejecución local.
\end{itemize}

\section{Especificación de requisitos}

\subsection{Actores}
En nuestro caso solamente tendremos dos actores, que serán el alumno y el profesor(administrador).


