\apendice{Especificación de Requisitos}

\section{Introducción}
Se expondrán en esta sección los distintos objetivos que la aplicación debe lograr, así como los requisitos que debe cumplir.

\section{Objetivos generales}
Como objetivos generales de nuestro proyecto tendremos:
\begin{itemize}
	\item Diferenciar en nuestra aplicación entre los diferentes roles de usuario, cada uno con sus distintas funcionalidades.
	\item Añadir restricciones en el uso de caracteres para anotaciones, medidas y ejercicios.
	\item Facilitar la navegabilidad durante el uso de la aplicación.
	\item Creación de una interfaz que permita al profesor facilidad para corregir ejercicios de alumnos.
	\item Proporcionar seguridad para los modelos debido a su unicidad a la hora de ser alojados en el servidor web.
	\item Desplegar nuestra aplicación en un servidor web.
\end{itemize}

\section{Catalogo de requisitos}
A continuación, listaremos el conjunto de requisitos funcionales extraídos a partir de los objetivos generales del proyecto.

\subsection{Requisitos funcionales}
Los requisitos funcionales se han sacado a partir de un diagrama de casos de uso hecho desde cero aunque ciertos requisitos coinciden con los de dicha versión anterior, por lo que serán iguales.

\begin{itemize}
	\item \textbf{RF-1 Visor de modelos}: la aplicación debe ser capaz de visualizar un modelo.
		\begin{itemize}
		\item \textbf{RF-1.1 Gestión de anotaciones}: la aplicación debe ser capaz de manejar anotaciones sobre el modelo.
			\begin{itemize}
				\item \textbf{RF-1.1.1 Añadir anotación}: el usuario debe poder añadir una anotación.
				\item \textbf{RF-1.1.2 Eliminar anotación}: el usuario debe poder eliminar una anotación.
				\item \textbf{RF-1.1.3 Editar anotación}: el usuario debe poder editar la etiqueta de una anotación.
				\item \textbf{RF-1.1.4 Seleccionar anotación}: el usuario debe poder seleccionar una anotación.
				\item \textbf{RF-1.1.5 Deseleccionar anotación}: el usuario debe poder deseleccionar una anotación.
			\end{itemize}
		\end{itemize}
		\begin{itemize}
			\item \textbf{RF-1.2 Gestión de medidas}: la aplicación debe ser capaz de manejar medidas sobre el modelo.
			\begin{itemize}
				\item \textbf{RF-1.2.1 Añadir medida}: el usuario debe poder añadir una medida.
				\item \textbf{RF-1.2.2 Eliminar medida}: el usuario debe poder eliminar una medida.
				\item \textbf{RF-1.2.3 Editar medida}: el usuario debe poder editar la etiqueta de una medida.
				\item \textbf{RF-1.2.4 Seleccionar medida}: el usuario debe poder seleccionar una medida.
				\item \textbf{RF-1.2.5 Deseleccionar medida}: el usuario debe poder deseleccionar una medida.
			\end{itemize}
		\end{itemize}
		\item \textbf{RF-1.3 Exportar puntos}: la aplicación debe ser capaz de exportar las anotaciones y medidas que se encuentren actualmente en el visor.
		\item \textbf{RF-1.4 Importar puntos}: la aplicación debe ser capaz de importar las anotaciones y medidas que se encuentren actualmente en el visor.

	\item \textbf{RF-2 Visor de ejercicio}: la aplicación debe ser capaz de visualizar un ejercicio guardado.
	\begin{itemize}
		\item \textbf{RF-2.1 Gestión de anotaciones}: la aplicación debe ser capaz de manejar anotaciones sobre el modelo.
		\begin{itemize}
			\item \textbf{RF-2.1.1 Añadir anotación}: el usuario debe poder añadir una anotación.
			\item \textbf{RF-2.1.2 Eliminar anotación}: el usuario debe poder eliminar una anotación.
			\item \textbf{RF-2.1.3 Editar anotación}: el usuario debe poder editar la etiqueta de una anotación.
			\item \textbf{RF-2.1.4 Seleccionar anotación}: el usuario debe poder seleccionar una anotación.
			\item \textbf{RF-2.1.5 Deseleccionar anotación}: el usuario debe poder deseleccionar una anotación.
		\end{itemize}
	\end{itemize}
	\begin{itemize}
		\item \textbf{RF-2.2 Gestión de medidas}: la aplicación debe ser capaz de manejar medidas sobre el modelo.
		\begin{itemize}
			\item \textbf{RF-2.2.1 Añadir medida}: el usuario debe poder añadir una medida.
			\item \textbf{RF-2.2.2 Eliminar medida}: el usuario debe poder eliminar una medida.
			\item \textbf{RF-2.2.3 Editar medida}: el usuario debe poder editar la etiqueta de una medida.
			\item \textbf{RF-2.2.4 Seleccionar medida}: el usuario debe poder seleccionar una medida.
			\item \textbf{RF-2.2.5 Deseleccionar medida}: el usuario debe poder deseleccionar una medida.
		\end{itemize}
	\end{itemize}
	\item \textbf{RF-2.3 Restaurar datos}: la aplicación debe ser capaz de restaurar los datos iniciales del ejercicio previo a la carga del ejercicio del alumno.
	\item \textbf{RF-2.4 Cancelar ejercicio}: la aplicación debe ser capaz de cancelar y salir de un ejercicio en curso.
	\item \textbf{RF-2.5 Exportar puntos}: la aplicación debe ser capaz de exportar las anotaciones y medidas que se encuentren actualmente en el visor.
	\item \textbf{RF-2.6 Importar puntos}: la aplicación debe ser capaz de importar las anotaciones y medidas que se encuentren actualmente en el visor.

	\item \textbf{RF-3 Listar modelos}: la aplicación debe ser capaz de mostrar de una pasada los diferentes modelos disponibles.
	\begin{itemize}
		\item \textbf{RF-3.1 Eliminar modelos}: el usuario debe ser capaz de eliminar modelos.
	\end{itemize}
	\item \textbf{RF-4 Listar ejercicios}: la aplicación debe ser capaz de manejar un listado de ejercicios de cada modelo.
	\begin{itemize}
		\item \textbf{RF-4.1 Gestión de ejercicios}: la aplicación debe ser capaz de manejar medidas sobre el modelo.
		\begin{itemize}
			\item \textbf{RF-4.1.1 Añadir ejercicio}: el usuario debe poder añadir un ejercicio.
			\item \textbf{RF-4.1.2 Eliminar ejercicio}: el usuario debe poder eliminar un ejercicio.
			\item \textbf{RF-4.1.3 Modificar ejercicio}: el usuario debe poder modificar un ejercicio.
			\item \textbf{RF-4.1.4 Editar nombre ejercicio}: el usuario debe poder editar el nombre de un ejercicio.
			\item \textbf{RF-4.1.5 Guardar ejercicio}: el usuario debe poder guardar un ejercicio modificado.
		\end{itemize}
	\end{itemize}
	\item \textbf{RF-5 Subir modelos}: la aplicación debe ser capaz de manejar un listado de ejercicios de cada modelo.
\end{itemize}

\subsection{Requisitos no funcionales}
Estos requisitos coinciden con los de la versión anterior a excepción de uno. Aun así, mencionaremos todos para que quede más claro.

\begin{itemize}
	\item \textbf{RNF-1 Usabilidad}: el conjunto de elementos visuales de la interfaz deben ser intuitivos y conocidos por el usuario medio, permitiendo un rápido aprendizaje.
	\item \textbf{RNF-2 Mantenibilidad}: la aplicación debe desarrollarse siguiendo alguna técnica que permita facilidad de mantenimiento e incorporación de nuevas características, así como corrección de errores.
	\item \textbf{RNF-3 Soporte}: la aplicación debe poder emplearse sobre un amplio conjunto de navegadores.
	\item \textbf{RNF-4 Internacionalización}: la aplicación debe estar diseñada para soportar diferentes idiomas.
	\item \textbf{RNF-5 Control de acceso}: la aplicación debe soportar control de acceso de usuarios.
	\item \textbf{RNF-6 Despliegue}: la aplicación debe estar desplegada de manera que sea accesible sin necesidad de una ejecución local.
\end{itemize}

\section{Especificación de requisitos}

\subsection{Actores}
En nuestro caso solamente tendremos dos actores, que serán el alumno y el profesor(administrador).

\subsection{Diagrama de casos de uso}
A continuación se mostrará el diagrama de casos de uso de nuestro proyecto por niveles en las figuras~\ref{fig:nivel1},~\ref{fig:nivel2} y~\ref{fig:nivel3}:

\imagen{nivel1}{Nivel 1 del diagrama de casos de uso.}{0.9}
\imagen{nivel2}{Nivel 2 del diagrama de casos de uso.}{0.9}
\imagen{nivel3}{Nivel 3 del diagrama de casos de uso.}{0.9}

\tablaAncho
{CU-1 Visor de modelos}
{p{2.9cm} X}
{use-case-1}
{	
	\textbf{CU-1} & \textbf{Visor de modelo} \\ \otoprule
	\textbf{Versión} & 2.0 \\ \midrule
	\textbf{Autor} & Jose Manuel Moral Garrido \\ \midrule
	\textbf{Requisitos asociados} & RF-1.1, RF-1.2, RF-1.3, RF-1.4 \\ \midrule
	\textbf{Descripción} & Permite al usuario visualizar un modelo y realizar operaciones sobre el mismo. \\ \midrule
	\textbf{Precondiciones} & 
	\tabitem El usuario ha seleccionado un modelo.
	\\ \midrule
	\textbf{Acciones} & 
	\enumeratecompacto{
		\item El usuario abre un modelo.
		\item Se muestran el modelo sin anotaciones ni medidas.
		\item Se da la posibilidad de gestionar tanto anotaciones como medidas.
	}
	\\ \midrule
	\textbf{Postcondiciones} & - \\ \midrule
	\textbf{Excepciones} & - \\ \midrule
	\textbf{Importancia} & Alta \\ 
}


\tablaAncho
{CU-1.1 Gestión de anotaciones}
{p{2.9cm} X}
{use-case-1.1}
{
	\textbf{CU-1.1} & \textbf{Gestión de anotaciones} \\ \otoprule
	\textbf{Versión} & 2.0 \\ \midrule
	\textbf{Autor} & Jose Manuel Moral Garrido \\ \midrule
	\textbf{Requisitos asociados} & RF-1, RF-1.1.1, RF-1.1.2, RF-1.1.3, RF-1.1.4, RF-1.1.5 \\ \midrule
	\textbf{Descripción} & Permite gestionar las anotaciones (añadirlas, eliminarlas, etc.). \\ \midrule
	\textbf{Precondiciones} & 
	\tabitem El usuario tiene un modelo abierto.
	\\ \midrule
	\textbf{Acciones} & 
	\enumeratecompacto{
		\item Se muestran las anotaciones del modelo visualizado.
		\item Se muestran las opciones de añadir, eliminar, editar la etiqueta de y deseleccionar las anotaciones.
	}
	\\ \midrule
	\textbf{Postcondiciones} & - \\ \midrule
	\textbf{Excepciones} & - \\ \midrule
	\textbf{Importancia} & Alta \\ 
}


\tablaAncho
{CU-1.1.1 Añadir anotación}
{p{2.9cm} X}
{use-case-1.1.1}
{	
	\textbf{CU-1.1.1} & \textbf{Añadir anotación} \\ \otoprule
	\textbf{Versión} & 2.0 \\ \midrule
	\textbf{Autor} & Jose Manuel Moral Garrido \\ \midrule
	\textbf{Requisitos asociados} & RF-1.1 \\ \midrule
	\textbf{Descripción} & Permite al usuario añadir una anotación al modelo. \\ \midrule
	\textbf{Precondiciones} & - \\ \midrule
	\textbf{Acciones} & 
	\enumeratecompacto{
		\item El usuario pincha en añadir anotación.
		\item El usuario pincha sobre el modelo en la zona donde quiere crear una anotación.
		\item Una esfera es añadida en el modelo para representar la anotación.
		\item Un elemento aparece en un menú para representar la etiqueta de la anotación.
	}
	\\ \midrule
	\textbf{Postcondiciones} & 
	\tabitem Se añade una anotación al modelo.
	\\ \midrule
	\textbf{Excepciones} & - \\ \midrule
	\textbf{Importancia} & Alta \\ 
}


\tablaAncho
{CU-1.1.2 Eliminar anotación}
{p{2.9cm} X}
{use-case-1.1.2}
{	
	\textbf{CU-1.1.2} & \textbf{Eliminar anotación} \\ \otoprule
	\textbf{Versión} & 2.0 \\ \midrule
	\textbf{Autor} & Jose Manuel Moral Garrido \\ \midrule
	\textbf{Requisitos asociados} & RF-1.1 \\ \midrule
	\textbf{Descripción} & Permite al usuario eliminar una anotación. \\ \midrule
	\textbf{Precondiciones} & 
	\tabitem Que exista una anotación en el modelo.
	
	\tabitem (opcional) Que una anotación se encuentre seleccionada.
	\\ \midrule
	\textbf{Acciones} & 
	\enumeratecompacto{
		\item Seleccionar una anotación (opcional).
		\item Pulsar sobre borrar anotación. Si se ha realizado anterior, saltamos el siguiente paso.
		\item Seleccionar una anotación (opcional).
		\item Se borra la anotación.
	}
	\\ \midrule
	\textbf{Postcondiciones} & 
	\tabitem Se elimina la anotación seleccionada.
	\\ \midrule
	\textbf{Excepciones} & - \\ \midrule
	\textbf{Importancia} & Media \\ 
}


\tablaAncho
{CU-1.1.3 Editar anotación}
{p{2.9cm} X}
{use-case-1.1.3}
{	
	\textbf{CU-1.1.3} & \textbf{Editar anotación} \\ \otoprule
	\textbf{Versión} & 2.0 \\ \midrule
	\textbf{Autor} & Jose Manuel Moral Garrido \\ \midrule
	\textbf{Requisitos asociados} & RF-1.1 \\ \midrule
	\textbf{Descripción} & Permite editar la etiqueta de una anotación. \\ \midrule
	\textbf{Precondiciones} & 
	\tabitem Que haya anotaciones.
	\\ \midrule
	\textbf{Acciones} & 
	\enumeratecompacto{
		\item Seleccionar una anotación.
		\item Pulsar sobre editar anotación.
		\item Se muestra un diálogo con la etiqueta actual de la anotación.
		\item El usuario edita el texto.
		\item Pulsar sobre guardar.
		\item El texto de la anotación cambia.
	}
	\\ \midrule
	\textbf{Postcondiciones} & 
	\tabitem La anotación modificada cambia su texto.
	\\ \midrule
	\textbf{Excepciones} & - \\ \midrule
	\textbf{Importancia} & Media \\ 
}


\tablaAncho
{CU-1.1.4 Seleccionar anotación}
{p{2.9cm} X}
{use-case-1.1.4}
{
	\textbf{CU-1.1.4} & \textbf{Seleccionar anotación} \\ \otoprule
	\textbf{Versión} & 2.0 \\ \midrule
	\textbf{Autor} & Jose Manuel Moral Garrido \\ \midrule
	\textbf{Requisitos asociados} & RF-1.1 \\ \midrule
	\textbf{Descripción} & Permite seleccionar una anotación resaltándola. \\ \midrule
	\textbf{Precondiciones} & 
	\tabitem Tiene que existir al menos una anotación.
	\\ \midrule
	\textbf{Acciones} & 
	\enumeratecompacto{
		\item El usuario selecciona una anotación en el modelo (opción 1).
		\item El usuario selecciona una anotación en la lista lateral (opción 2).
		\item La anotación se destaca tanto en el modelo como en la lista lateral.
	}
	\\ \midrule
	\textbf{Postcondiciones} \\ \midrule
	\textbf{Excepciones} & - \\ \midrule	
	\textbf{Importancia} & Media \\ 
}


\tablaAncho
{CU-1.1.5 Deseleccionar anotación}
{p{2.9cm} X}
{use-case-1.1.5}
{
	\textbf{CU-1.1.5} & \textbf{Deseleccionar anotación} \\ \otoprule
	\textbf{Versión} & 2.0 \\ \midrule
	\textbf{Autor} & Jose Manuel Moral Garrido \\ \midrule
	\textbf{Requisitos asociados} & RF-1.1 \\ \midrule
	\textbf{Descripción} & Permite al usuario deseleccionar una anotación. \\ \midrule
	\textbf{Precondiciones} & 
	\tabitem Que haya al menos una anotación seleccionada.
	\\ \midrule
	\textbf{Acciones} & 
	\enumeratecompacto{
		\item Pinchar sobre una anotación en el modelo (opción 1).
		\item Pinchar sobre una anotación en la lista lateral (opción 2).
		\item Pinchar sobre deseleccionar todo en la lista lateral (opción 3).
		\item Si se ha realizado opcion 1 o 2, se deselecciona dicha anotación.
		\item Si se realiza opción 3, se deseleccionan todas las anotaciones.
	}
	\\ \midrule
	\textbf{Postcondiciones} \\ \midrule
	\textbf{Excepciones} & - \\ \midrule
	\textbf{Importancia} & Media \\ 
}


\tablaAncho
{CU-1.2 Gestión de medidas}
{p{2.9cm} X}
{use-case-1.2}
{
	\textbf{CU-1.2} & \textbf{Gestión de medida} \\ \otoprule
	\textbf{Versión} & 1.0 \\ \midrule
	\textbf{Autor} & Jose Manuel Moral Garrido \\ \midrule
	\textbf{Requisitos asociados} & RF-1, RF-1.2.1, RF-1.2.2, RF-1.2.3, RF-1.2.4, RF-1.2.5 \\ \midrule
	\textbf{Descripción} & Permite gestionar las medidas (añadirlas, eliminarlas, \dots). \\ \midrule
	\textbf{Precondiciones} & 
	\tabitem El usuario tiene abierto un modelo.
	\\ \midrule
	\textbf{Acciones} & 
	\enumeratecompacto{
		\item Se muestran las medidas del modelo visualizado.
		\item Se muestran las opciones de añadir, eliminar, editar la etiqueta de y deseleccionar las medidas.
	}
	\\ \midrule
	\textbf{Postcondiciones} & - \\ \midrule
	\textbf{Excepciones} & - \\ \midrule
	\textbf{Importancia} & Alta \\ 
}


\tablaAncho
{CU-1.2.1 Añadir medida}
{p{2.9cm} X}
{use-case-1.2.1}
{
	\textbf{CU-1.2.1} & \textbf{Añadir medida} \\ \otoprule
	\textbf{Versión} & 2.0 \\ \midrule
	\textbf{Autor} & Jose Manuel Moral Garrido \\ \midrule
	\textbf{Requisitos asociados} & RF-1.2 \\ \midrule
	\textbf{Descripción} & Permite al usuario añadir una medida al modelo. \\ \midrule
	\textbf{Precondiciones} & - \\ \midrule
	\textbf{Acciones} & 
	\enumeratecompacto{
		\item El usuario pincha en añadir medida.
		\item El usuario pincha sobre el modelo en la zona donde quiere crear una anotación.
		\item El usuario pincha sobre el modelo en la posición donde quiere que vaya el otro extremo de la medida.
		\item Dos esferas y una raya aparecen en el visor para representar la medida.
		\item Un elemento aparece en un menú para representar la etiqueta de la medida, que será la etiqueta y las unidades.
	}
	\\ \midrule
	\textbf{Postcondiciones} & 
	\tabitem Se añade una medida al modelo.
	\\ \midrule
	\textbf{Excepciones} & 
	\tabitem No se añaden los dos puntos de la medida.
	\\ \midrule
	\textbf{Importancia} & Alta \\ 
}


\tablaAncho
{CU-1.2.2 Eliminar medida}
{p{2.9cm} X}
{use-case-1.2.2}
{
	\textbf{CU-1.2.2} & \textbf{Eliminar medida} \\ \otoprule
	\textbf{Versión} & 2.0 \\ \midrule
	\textbf{Autor} & Jose Manuel Moral Garrido \\ \midrule
	\textbf{Requisitos asociados} & RF-1.2 \\ \midrule
	\textbf{Descripción} & Permite al usuario eliminar una medida. \\ \midrule
	\textbf{Precondiciones} & 
	\tabitem Que exista una medida en el modelo.
	
	\tabitem (opcional) Que una medida se encuentre seleccionada.
	\\ \midrule
	\textbf{Acciones} & 
	\enumeratecompacto{
		\item Seleccionar una medida (opción 1).
		\item Pulsar sobre borrar medida.
		\item Seleccionar una medida (opción 2).
		\item Se borra la medida.
	}
	\\ \midrule
	\textbf{Postcondiciones} & 
	\tabitem Se elimina la medida seleccionada.
	\\ \midrule
	\textbf{Excepciones} & - \\ \midrule
	\textbf{Importancia} & Media \\ 
}


\tablaAncho
{CU-1.2.3 Editar medida}
{p{2.9cm} X}
{use-case-1.2.3}
{
	\textbf{CU-1.2.3} & \textbf{Editar medida} \\ \otoprule
	\textbf{Versión} & 2.0 \\ \midrule
	\textbf{Autor} & Jose Manuel Moral Garrido \\ \midrule
	\textbf{Requisitos asociados} & RF-1.2 \\ \midrule
	\textbf{Descripción} & Permite editar la etiqueta de una medida. \\ \midrule
	\textbf{Precondiciones} & 
	\tabitem Que haya medidas.
	\\ \midrule
	\textbf{Acciones} & 
	\enumeratecompacto{
		\item Seleccionar una medida.
		\item Pulsar sobre editar medida.
		\item Mostrar un diálogo con la etiqueta actual de la medida.
		\item El usuario edita el texto.
		\item Pulsar sobre guardar.
		\item El texto de la medida cambia.
	}
	\\ \midrule
	\textbf{Postcondiciones} & 
	\tabitem La medida modificada cambia su texto.
	\\ \midrule
	\textbf{Excepciones} & - \\ \midrule
	\textbf{Importancia} & Media \\ 
}


\tablaAncho
{CU-1.2.4 Seleccionar medida}
{p{2.9cm} X}
{use-case-1.2.4}
{
	\textbf{CU-1.2.4} & \textbf{Seleccionar medida} \\ \otoprule
	\textbf{Versión} & 2.0 \\ \midrule
	\textbf{Autor} & Jose Manuel Moral Garrido \\ \midrule
	\textbf{Requisitos asociados} & RF-1.2 \\ \midrule
	\textbf{Descripción} & Permite seleccionar una medida resaltándola. \\ \midrule
	\textbf{Precondiciones} & 
	\tabitem Tiene que existir al menos una medida.
	\\ \midrule
	\textbf{Acciones} & 
	\enumeratecompacto{
		\item El usuario selecciona una medida en el modelo (opción 1).
		\item El usuario selecciona una medida en la lista lateral (opción 2).
		\item La medida se destaca tanto en el modelo como en la lista lateral.
	}
	\\ \midrule
	\textbf{Postcondiciones} & - \\ \midrule
	\textbf{Excepciones} & - \\ \midrule
	\textbf{Importancia} & Media \\ 
}


\tablaAncho
{CU-1.2.5 Deseleccionar medida}
{p{2.9cm} X}
{use-case-1.2.5}
{
	\textbf{CU-1.2.5} & \textbf{Deseleccionar medida} \\ \otoprule
	\textbf{Versión} & 2.0 \\ \midrule
	\textbf{Autor} & Jose Manuel Moral Garrido \\ \midrule
	\textbf{Requisitos asociados} & RF-1.2 \\ \midrule
	\textbf{Descripción} & Permite al usuario deseleccionar una medida. \\ \midrule
	\textbf{Precondiciones} & 
	\tabitem Que haya al menos una medida seleccionada.
	\\ \midrule
	\textbf{Acciones} & 
	\enumeratecompacto{
		\item Pinchar sobre una medida en el modelo (opción 1).
		\item Pinchar sobre una medida en la lista lateral (opción 2).
		\item Pinchar sobre deseleccionar todo en la lista lateral (opción 3).
		\item Si se ha realizado opcion 1 o 2, se deselecciona dicha medida.
		\item Si se realiza opción 3, se deseleccionan todas las medidas.
	}
	\\ \midrule
	\textbf{Postcondiciones} & - \\ \midrule
	\textbf{Excepciones} & - \\ \midrule
	\textbf{Importancia} & Media \\ 
}


\tablaAncho
{CU-1.3 Exportar puntos}
{p{2.9cm} X}
{use-case-1.3}
{
	\textbf{CU-1.3} & \textbf{Exportar punto} \\ \otoprule
	\textbf{Versión} & 2.0 \\ \midrule
	\textbf{Autor} & Jose Manuel Moral Garrido \\ \midrule
	\textbf{Requisitos asociados} & RF-1 \\ \midrule
	\textbf{Descripción} & Permite exportar los puntos que se estén visualizando. \\ \midrule
	\textbf{Precondiciones} & - \\ \midrule
	\textbf{Acciones} & 
	\enumeratecompacto{
		\item Pulsar sobre exportar puntos.
		\item Un archivo se descarga.
	}
	\\ \midrule
	\textbf{Postcondiciones} & 
	\tabitem Un archivo nuevo con nuestros puntos se almacena.
	\tabitem Si el archivo es exportado por un alumno, quedará reflejado en archivo almacenado.
	\\ \midrule
	\textbf{Excepciones} & - \\ \midrule
	\textbf{Importancia} & Media \\ 
}


\tablaAncho
{CU-1.4 Importar puntos}
{p{2.9cm} X}
{use-case-1.4}
{
	\textbf{CU-1.4} & \textbf{Importar punto} \\ \otoprule
	\textbf{Versión} & 2.0 \\ \midrule
	\textbf{Autor} & Jose Manuel Moral Garrido \\ \midrule
	\textbf{Requisitos asociados} & RF-1 \\ \midrule
	\textbf{Descripción} & Permite al usuario cargar puntos creados previamente. \\ \midrule
	\textbf{Precondiciones} & 
	\tabitem Tener un archivo correctamente formado con puntos para el modelo.
	\\ \midrule
	\textbf{Acciones} & 
	\enumeratecompacto{
		\item Pulsar sobre importar puntos.
		\item Seleccionar el archivo deseado.
		\item Pulsar sobre abrir.
		\item Las anotaciones y medidas se añaden.
		\item El archivo importado contendrá información de quién la ha realizado.
	}
	\\ \midrule
	\textbf{Postcondiciones} & 
	\tabitem Se añaden añaden nuevas anotaciones y medidas.
	\\ \midrule
	\textbf{Excepciones} &
	\tabitem El archivo está mal formado.
	
	\tabitem El archivo no contiene anotaciones y medidas.
	\\ \midrule	
	\textbf{Importancia} & Media \\ 
}


\tablaAncho
{CU-2 Visor de ejercicios}
{p{2.9cm} X}
{use-case-2}
{	
	\textbf{CU-2} & \textbf{Visor de ejercicios} \\ \otoprule
	\textbf{Versión} & 1.0 \\ \midrule
	\textbf{Autor} & Jose Manuel Moral Garrido \\ \midrule
	\textbf{Requisitos asociados} & RF-2.1, RF-2.2, RF-2.3, RF-2.4, RF-2.5, RF-2.6, RF-1 \\ \midrule
	\textbf{Descripción} & Permite al usuario visualizar un ejercicio y realizar operaciones sobre el mismo. \\ \midrule
	\textbf{Precondiciones} & 
	\tabitem El usuario ha seleccionado un ejercicio.
	\\ \midrule
	\textbf{Acciones} & 
	\enumeratecompacto{
		\item El usuario abre un ejercicio.
		\item Se muestran el ejercicio con las anotaciones y medidas guardadas.
		\item Se da la posibilidad de gestionar tanto anotaciones como medidas y además incluir ejercicios de los alumnos.
	}
	\\ \midrule
	\textbf{Postcondiciones} & - \\ \midrule
	\textbf{Excepciones} & - \\ \midrule
	\textbf{Importancia} & Alta \\ 
}


\tablaAncho
{CU-2.1 Gestión de anotaciones}
{p{2.9cm} X}
{use-case-2.1}
{
	\textbf{CU-1.1} & \textbf{Gestión de anotaciones} \\ \otoprule
	\textbf{Versión} & 2.0 \\ \midrule
	\textbf{Autor} & Jose Manuel Moral Garrido \\ \midrule
	\textbf{Requisitos asociados} & RF-2, RF-2.1.1, RF-2.1.2, RF-2.1.3, RF-2.1.4, RF-2.1.5 \\ \midrule
	\textbf{Descripción} & Permite gestionar las anotaciones (añadirlas, eliminarlas, etc.). \\ \midrule
	\textbf{Precondiciones} & 
	\tabitem El usuario tiene un ejercicio abierto.
	\\ \midrule
	\textbf{Acciones} & 
	\enumeratecompacto{
		\item Se muestran las anotaciones del ejercicio visualizado.
		\item Se muestran las opciones de añadir, eliminar, editar la etiqueta de y deseleccionar las anotaciones.
	}
	\\ \midrule
	\textbf{Postcondiciones} & - \\ \midrule
	\textbf{Excepciones} & - \\ \midrule
	\textbf{Importancia} & Alta \\ 
}


\tablaAncho
{CU-2.1.1 Añadir anotación}
{p{2.9cm} X}
{use-case-2.1.1}
{	
	\textbf{CU-1.1.1} & \textbf{Añadir anotación} \\ \otoprule
	\textbf{Versión} & 2.0 \\ \midrule
	\textbf{Autor} & Jose Manuel Moral Garrido \\ \midrule
	\textbf{Requisitos asociados} & RF-2.1 \\ \midrule
	\textbf{Descripción} & Permite al usuario añadir una anotación al ejercicio. \\ \midrule
	\textbf{Precondiciones} & - \\ \midrule
	\textbf{Acciones} & 
	\enumeratecompacto{
		\item El usuario pincha en añadir anotación.
		\item El usuario pincha sobre el ejercicio en la zona donde quiere crear una anotación.
		\item Una esfera es añadida en el ejercicio para representar la anotación.
		\item Un elemento aparece en un menú para representar la etiqueta de la anotación.
	}
	\\ \midrule
	\textbf{Postcondiciones} & 
	\tabitem Se añade una anotación al ejercicio.
	\\ \midrule
	\textbf{Excepciones} & - \\ \midrule
	\textbf{Importancia} & Alta \\ 
}


\tablaAncho
{CU-2.1.2 Eliminar anotación}
{p{2.9cm} X}
{use-case-2.1.2}
{	
	\textbf{CU-2.1.2} & \textbf{Eliminar anotación} \\ \otoprule
	\textbf{Versión} & 1.0 \\ \midrule
	\textbf{Autor} & Jose Manuel Moral Garrido \\ \midrule
	\textbf{Requisitos asociados} & RF-2.1 \\ \midrule
	\textbf{Descripción} & Permite al usuario eliminar una anotación. \\ \midrule
	\textbf{Precondiciones} & 
	\tabitem Que exista una anotación en el ejercicio.
	
	\tabitem (opcional) Que una anotación se encuentre seleccionada.
	\\ \midrule
	\textbf{Acciones} & 
	\enumeratecompacto{
		\item Seleccionar una anotación (opcional).
		\item Pulsar sobre borrar anotación. Si se ha realizado anterior, saltamos el siguiente paso.
		\item Seleccionar una anotación (opcional).
		\item Se borra la anotación.
	}
	\\ \midrule
	\textbf{Postcondiciones} & 
	\tabitem Se elimina la anotación seleccionada.
	\\ \midrule
	\textbf{Excepciones} & - \\ \midrule
	\textbf{Importancia} & Media \\ 
}


\tablaAncho
{CU-2.1.3 Editar anotación}
{p{2.9cm} X}
{use-case-2.1.3}
{	
	\textbf{CU-2.1.3} & \textbf{Editar anotación} \\ \otoprule
	\textbf{Versión} & 1.0 \\ \midrule
	\textbf{Autor} & Jose Manuel Moral Garrido \\ \midrule
	\textbf{Requisitos asociados} & RF-2.1 \\ \midrule
	\textbf{Descripción} & Permite editar la etiqueta de una anotación. \\ \midrule
	\textbf{Precondiciones} & 
	\tabitem Que haya anotaciones.
	\\ \midrule
	\textbf{Acciones} & 
	\enumeratecompacto{
		\item Seleccionar una anotación.
		\item Pulsar sobre editar anotación.
		\item Se muestra un diálogo con la etiqueta actual de la anotación.
		\item El usuario edita el texto.
		\item Pulsar sobre guardar.
		\item El texto de la anotación cambia.
	}
	\\ \midrule
	\textbf{Postcondiciones} & 
	\tabitem La anotación modificada cambia su texto.
	\\ \midrule
	\textbf{Excepciones} & - \\ \midrule
	\textbf{Importancia} & Media \\ 
}


\tablaAncho
{CU-2.1.4 Seleccionar anotación}
{p{2.9cm} X}
{use-case-2.1.4}
{
	\textbf{CU-2.1.4} & \textbf{Seleccionar anotación} \\ \otoprule
	\textbf{Versión} & 1.0 \\ \midrule
	\textbf{Autor} & Jose Manuel Moral Garrido \\ \midrule
	\textbf{Requisitos asociados} & RF-2.1 \\ \midrule
	\textbf{Descripción} & Permite seleccionar una anotación resaltándola. \\ \midrule
	\textbf{Precondiciones} & 
	\tabitem Tiene que existir al menos una anotación.
	\\ \midrule
	\textbf{Acciones} & 
	\enumeratecompacto{
		\item El usuario selecciona una anotación en el ejercicio (opción 1).
		\item El usuario selecciona una anotación en la lista lateral (opción 2).
		\item La anotación se destaca tanto en el ejercicio como en la lista lateral.
	}
	\\ \midrule
	\textbf{Postcondiciones} & - \\ \midrule
	\textbf{Excepciones} & - \\ \midrule	
	\textbf{Importancia} & Media \\ 
}


\tablaAncho
{CU-2.1.5 Deseleccionar anotación}
{p{2.9cm} X}
{use-case-2.1.5}
{
	\textbf{CU-2.1.5} & \textbf{Deseleccionar anotación} \\ \otoprule
	\textbf{Versión} & 1.0 \\ \midrule
	\textbf{Autor} & Jose Manuel Moral Garrido \\ \midrule
	\textbf{Requisitos asociados} & RF-2.1 \\ \midrule
	\textbf{Descripción} & Permite al usuario deseleccionar una anotación. \\ \midrule
	\textbf{Precondiciones} & 
	\tabitem Que haya al menos una anotación seleccionada.
	\\ \midrule
	\textbf{Acciones} & 
	\enumeratecompacto{
		\item Pinchar sobre una anotación en el ejercicio (opción 1).
		\item Pinchar sobre una anotación en la lista lateral (opción 2).
		\item Pinchar sobre deseleccionar todo en la lista lateral (opción 3).
		\item Si se ha realizado opcion 1 o 2, se deselecciona dicha anotación.
		\item Si se realiza opción 3, se deseleccionan todas las anotaciones.
	}
	\\ \midrule
	\textbf{Postcondiciones} & - \\ \midrule
	\textbf{Excepciones} & - \\ \midrule
	\textbf{Importancia} & Media \\ 
}


\tablaAncho
{CU-2.2 Gestión de medidas}
{p{2.9cm} X}
{use-case-2.2}
{
	\textbf{CU-2.1} & \textbf{Gestión de medidas} \\ \otoprule
	\textbf{Versión} & 1.0 \\ \midrule
	\textbf{Autor} & Jose Manuel Moral Garrido \\ \midrule
	\textbf{Requisitos asociados} & RF-2, RF-2.2.1, RF-2.2.2, RF-2.2.3, RF-2.2.4, RF-2.2.5 \\ \midrule
	\textbf{Descripción} & Permite gestionar las medidas (añadirlas, eliminarlas, \dots). \\ \midrule
	\textbf{Precondiciones} & 
	\tabitem El usuario tiene abierto un ejercicio.
	\\ \midrule
	\textbf{Acciones} & 
	\enumeratecompacto{
		\item Se muestran las medidas del ejercicio visualizado.
		\item Se muestran las opciones de añadir, eliminar, editar la etiqueta de y deseleccionar las medidas.
	}
	\\ \midrule
	\textbf{Postcondiciones} & - \\ \midrule
	\textbf{Excepciones} & - \\ \midrule
	\textbf{Importancia} & Alta \\ 
}


\tablaAncho
{CU-2.2.1 Añadir medida}
{p{2.9cm} X}
{use-case-2.2.1}
{
	\textbf{CU-2.2.1} & \textbf{Añadir medida} \\ \otoprule
	\textbf{Versión} & 1.0 \\ \midrule
	\textbf{Autor} & Jose Manuel Moral Garrido \\ \midrule
	\textbf{Requisitos asociados} & RF-2.2 \\ \midrule
	\textbf{Descripción} & Permite al usuario añadir una medida al modelo. \\ \midrule
	\textbf{Precondiciones} & - \\ \midrule
	\textbf{Acciones} & 
	\enumeratecompacto{
		\item El usuario pincha en añadir medida.
		\item El usuario pincha sobre el ejercicio en la zona donde quiere crear una anotación.
		\item El usuario pincha sobre el ejercicio en la posición donde quiere que vaya el otro extremo de la medida.
		\item Dos esferas y una raya aparecen en el visor para representar la medida.
		\item Un elemento aparece en un menú para representar la etiqueta de la medida, que será la etiqueta y las unidades.
	}
	\\ \midrule
	\textbf{Postcondiciones} & 
	\tabitem Se añade una medida al ejercicio.
	\\ \midrule
	\textbf{Excepciones} & 
	\tabitem No se añaden los dos puntos de la medida.
	\\ \midrule
	\textbf{Importancia} & Alta \\ 
}


\tablaAncho
{CU-2.2.2 Eliminar medida}
{p{2.9cm} X}
{use-case-2.2.2}
{
	\textbf{CU-2.2.2} & \textbf{Eliminar medida} \\ \otoprule
	\textbf{Versión} & 1.0 \\ \midrule
	\textbf{Autor} & Jose Manuel Moral Garrido \\ \midrule
	\textbf{Requisitos asociados} & RF-2.2 \\ \midrule
	\textbf{Descripción} & Permite al usuario eliminar una medida. \\ \midrule
	\textbf{Precondiciones} & 
	\tabitem Que exista una medida en el ejercicio.
	
	\tabitem (opcional) Que una medida se encuentre seleccionada.
	\\ \midrule
	\textbf{Acciones} & 
	\enumeratecompacto{
		\item Seleccionar una medida (opción 1).
		\item Pulsar sobre borrar medida.
		\item Seleccionar una medida (opción 2).
		\item Se borra la medida.
	}
	\\ \midrule
	\textbf{Postcondiciones} & 
	\tabitem Se elimina la medida seleccionada.
	\\ \midrule
	\textbf{Excepciones} & - \\ \midrule
	\textbf{Importancia} & Media \\ 
}


\tablaAncho
{CU-2.2.3 Editar medida}
{p{2.9cm} X}
{use-case-2.2.3}
{
	\textbf{CU-1.2.3} & \textbf{Editar medida} \\ \otoprule
	\textbf{Versión} & 1.0 \\ \midrule
	\textbf{Autor} & Jose Manuel Moral Garrido \\ \midrule
	\textbf{Requisitos asociados} & RF-2.2 \\ \midrule
	\textbf{Descripción} & Permite editar la etiqueta de una medida. \\ \midrule
	\textbf{Precondiciones} & 
	\tabitem Que haya medidas.
	\\ \midrule
	\textbf{Acciones} & 
	\enumeratecompacto{
		\item Seleccionar una medida.
		\item Pulsar sobre editar medida.
		\item Mostrar un diálogo con la etiqueta actual de la medida.
		\item El usuario edita el texto.
		\item Pulsar sobre guardar.
		\item El texto de la medida cambia.
	}
	\\ \midrule
	\textbf{Postcondiciones} & 
	\tabitem La medida modificada cambia su texto.
	\\ \midrule
	\textbf{Excepciones} & - \\ \midrule
	\textbf{Importancia} & Media \\ 
}


\tablaAncho
{CU-2.2.4 Seleccionar medida}
{p{2.9cm} X}
{use-case-2.2.4}
{
	\textbf{CU-2.2.4} & \textbf{Seleccionar medida} \\ \otoprule
	\textbf{Versión} & 1.0 \\ \midrule
	\textbf{Autor} & Jose Manuel Moral Garrido \\ \midrule
	\textbf{Requisitos asociados} & RF-2.2 \\ \midrule
	\textbf{Descripción} & Permite seleccionar una medida resaltándola. \\ \midrule
	\textbf{Precondiciones} & 
	\tabitem Tiene que existir al menos una medida.
	\\ \midrule
	\textbf{Acciones} & 
	\enumeratecompacto{
		\item El usuario selecciona una medida en el ejercicio (opción 1).
		\item El usuario selecciona una medida en la lista lateral (opción 2).
		\item La medida se destaca tanto en el ejercicio como en la lista lateral.
	}
	\\ \midrule
	\textbf{Postcondiciones} & - \\ \midrule
	\textbf{Excepciones} & - \\ \midrule
	\textbf{Importancia} & Media \\ 
}


\tablaAncho
{CU-2.2.5 Deseleccionar medida}
{p{2.9cm} X}
{use-case-2.2.5}
{
	\textbf{CU-2.2.5} & \textbf{Deseleccionar medida} \\ \otoprule
	\textbf{Versión} & 1.0 \\ \midrule
	\textbf{Autor} & Jose Manuel Moral Garrido \\ \midrule
	\textbf{Requisitos asociados} & RF-2.2 \\ \midrule
	\textbf{Descripción} & Permite al usuario deseleccionar una medida. \\ \midrule
	\textbf{Precondiciones} & 
	\tabitem Que haya al menos una medida seleccionada.
	\\ \midrule
	\textbf{Acciones} & 
	\enumeratecompacto{
		\item Pinchar sobre una medida en el ejercicio (opción 1).
		\item Pinchar sobre una medida en la lista lateral (opción 2).
		\item Pinchar sobre deseleccionar todo en la lista lateral (opción 3).
		\item Si se ha realizado opcion 1 o 2, se deselecciona dicha medida.
		\item Si se realiza opción 3, se deseleccionan todas las medidas.
	}
	\\ \midrule
	\textbf{Postcondiciones} & - \\ \midrule
	\textbf{Excepciones} & - \\ \midrule
	\textbf{Importancia} & Media \\ 
}


\tablaAncho
{CU-2.3 Restaurar datos}
{p{2.9cm} X}
{use-case-2.3}
{
	\textbf{CU-2.3} & \textbf{Restaurar datos} \\ \otoprule
	\textbf{Versión} & 1.0 \\ \midrule
	\textbf{Autor} & Jose Manuel Moral Garrido \\ \midrule
	\textbf{Requisitos asociados} & RF-2 \\ \midrule
	\textbf{Descripción} & Permite al usuario restaurar los valores de anotaciones y medidas iniciales (previamente guardados). \\ \midrule
	\textbf{Precondiciones} & 
	\tabitem Que un ejercicio tenga importado los datos del alumno.
	\\ \midrule
	\textbf{Acciones} & 
	\enumeratecompacto{
		\item Pinchar en el botón de restaurar.
		\item Se eliminarán todas las anotaciones y medidas que no pertenezcan al ejercicio previamente guardado (Si no hay ejercicio hecho, se borra todo).
	}
	\\ \midrule
	\textbf{Postcondiciones} & - \\ \midrule
	\tabitem Se recupera el ejercicio guardado.
	\textbf{Excepciones} & - \\ \midrule
	\textbf{Importancia} & Alta \\ 
}


\tablaAncho
{CU-2.4 Cancelar ejercicio}
{p{2.9cm} X}
{use-case-2.4}
{
	\textbf{CU-2.3} & \textbf{Cancelar ejercicio} \\ \otoprule
	\textbf{Versión} & 1.0 \\ \midrule
	\textbf{Autor} & Jose Manuel Moral Garrido \\ \midrule
	\textbf{Requisitos asociados} & RF-2 \\ \midrule
	\textbf{Descripción} & Permite al usuario cancelar un ejercicio abierto y salir de el. \\ \midrule
	\textbf{Precondiciones} & 
	\tabitem Tener un ejercicio abierto.
	\\ \midrule
	\textbf{Acciones} & 
	\enumeratecompacto{
		\item Pinchar en el botón de cancelar.
		\item Si existen cambios sobre ele ejercicio de partida, sale un diálogo preguntado si deseamos guardar los cambios.
		\item Si guardamos lo cambios, el ejercicio se actualiza.
		\item Si no guardamos los cambios, el ejercicio no se modifica.
		\item Si cancelamos, nos quedamos en el visor de ejercicios.
	}
	\\ \midrule
	\textbf{Postcondiciones} & - \\ \midrule
	\tabitem Se vuelve a la lista de ejercicios.
	\textbf{Excepciones} & - \\ \midrule
	\textbf{Importancia} & Alta \\ 
}


\tablaAncho
{CU-2.5 Exportar puntos}
{p{2.9cm} X}
{use-case-2.5}
{
	\textbf{CU-2.5} & \textbf{Exportar punto} \\ \otoprule
	\textbf{Versión} & 2.0 \\ \midrule
	\textbf{Autor} & Jose Manuel Moral Garrido \\ \midrule
	\textbf{Requisitos asociados} & RF-2 \\ \midrule
	\textbf{Descripción} & Permite exportar los puntos que se estén visualizando. \\ \midrule
	\textbf{Precondiciones} & - \\ \midrule
	\textbf{Acciones} & 
	\enumeratecompacto{
		\item Pulsar sobre exportar puntos.
		\item Un archivo se descarga.
	}
	\\ \midrule
	\textbf{Postcondiciones} & 
	\tabitem Un archivo nuevo con nuestros puntos se almacena.
	\\ \midrule
	\textbf{Excepciones} & - \\ \midrule
	\textbf{Importancia} & Media \\ 
}


\tablaAncho
{CU-2.6 Importar puntos}
{p{2.9cm} X}
{use-case-2.6}
{
	\textbf{CU-2.6} & \textbf{Importar punto} \\ \otoprule
	\textbf{Versión} & 2.0 \\ \midrule
	\textbf{Autor} & Jose Manuel Moral Garrido \\ \midrule
	\textbf{Requisitos asociados} & RF-2 \\ \midrule
	\textbf{Descripción} & Permite al usuario cargar puntos creados previamente. \\ \midrule
	\textbf{Precondiciones} & 
	\tabitem Tener un archivo correctamente formado con puntos para el ejercicio.
	\\ \midrule
	\textbf{Acciones} & 
	\enumeratecompacto{
		\item Pulsar sobre importar puntos.
		\item Seleccionar el archivo deseado.
		\item Pulsar sobre abrir.
		\item Las anotaciones y medidas se añaden.
		\item Si las anotaciones eran de un alumno, se importan con un color de esfera distinto.
	}
	\\ \midrule
	\textbf{Postcondiciones} & 
	\tabitem Se añaden nuevas anotaciones y medidas.
	\\ \midrule
	\textbf{Excepciones} &
	\tabitem El archivo está mal formado.
	\tabitem El archivo no contiene anotaciones y medidas.
	\\ \midrule	
	\textbf{Importancia} & Media \\ 
}


\tablaAncho
{CU-3 Listar modelos}
{p{2.9cm} X}
{use-case-3}
{
	\textbf{CU-3} & \textbf{Listar modelos} \\ \otoprule
	\textbf{Versión} & 1.0 \\ \midrule
	\textbf{Autor} & Jose Manuel Moral Garrido \\ \midrule
	\textbf{Requisitos asociados} & RF-3.1 \\ \midrule
	\textbf{Descripción} & Permite al usuario ver los modelos disponibles. \\ \midrule
	\textbf{Precondiciones} & 
	\tabitem Haber accedido a la aplicación correctamente.
	\\ \midrule
	\textbf{Acciones} & 
	\enumeratecompacto{
		\item Entrar en la aplicación.
		\item Elegir la opción de Modelos.
	}
	\\ \midrule
	\textbf{Postcondiciones} &- \\ \midrule
	\textbf{Excepciones} & - \\ \midrule
	\textbf{Importancia} & Alta \\ 
}


\tablaAncho
{CU-3.1 Eliminar modelos}
{p{2.9cm} X}
{use-case-3.1}
{
	\textbf{CU-3.1} & \textbf{Eliminar modelos} \\ \otoprule
	\textbf{Versión} & 1.0 \\ \midrule
	\textbf{Autor} & Jose Manuel Moral Garrido \\ \midrule
	\textbf{Requisitos asociados} & RF-3 \\ \midrule
	\textbf{Descripción} & Permite al usuario eliminar modelos. \\ \midrule
	\textbf{Precondiciones} & 
	\tabitem Haber accedido a la aplicación correctamente.
	\tabitem Tener rol de Profesor.
	\\ \midrule
	\textbf{Acciones} & 
	\enumeratecompacto{
		\item Entrar en la aplicación.
		\item Elegir la opción de Modelos.
		\item Elegir el modelo a eliminar
	}
	\\ \midrule
	\textbf{Postcondiciones} & - \\ \midrule
	\textbf{Excepciones} & - \\ \midrule
	\textbf{Importancia} & Alta \\ 
}


\tablaAncho
{CU-4 Listar ejercicios}
{p{2.9cm} X}
{use-case-4}
{
	\textbf{CU-4} & \textbf{Listar ejercicios} \\ \otoprule
	\textbf{Versión} & 1.0 \\ \midrule
	\textbf{Autor} & Jose Manuel Moral Garrido \\ \midrule
	\textbf{Requisitos asociados} & RF-4.1 \\ \midrule
	\textbf{Descripción} & Permite al usuario listar los ejercicios. \\ \midrule
	\textbf{Precondiciones} & 
	\tabitem Haber accedido a la aplicación correctamente.
	\tabitem Tener rol de Profesor.
	\\ \midrule
	\textbf{Acciones} & 
	\enumeratecompacto{
		\item Entrar en la aplicación.
		\item Elegir la opción de Ejercicios.
		\item Elegir un modelo para ver sus ejercicios.
	}
	\\ \midrule
	\textbf{Postcondiciones} &  - \\ \midrule
	\textbf{Excepciones} & - \\ \midrule
	\textbf{Importancia} & Alta \\ 
}


\tablaAncho
{CU-4.1 Gestión de ejercicios}
{p{2.9cm} X}
{use-case-4.1}
{
	\textbf{CU-4.1} & \textbf{Gestión de ejercicios} \\ \otoprule
	\textbf{Versión} & 1.0 \\ \midrule
	\textbf{Autor} & Jose Manuel Moral Garrido \\ \midrule
	\textbf{Requisitos asociados} & RF-4, RF-4.1.1 ,RF-4.1.2, RF-4.1.3, RF-4.1.4, RF-4.1.5 \\ \midrule
	\textbf{Descripción} & Permite al realizar operaciones sobres los ejercicios. \\ \midrule
	\textbf{Precondiciones} & 
	\tabitem Haber accedido a la aplicación correctamente.
	\tabitem Tener rol de Profesor.
	\\ \midrule
	\textbf{Acciones} & 
	\enumeratecompacto{
		\item Entrar en la aplicación.
		\item Elegir la opción de Ejercicios.
		\item Elegir un modelo para ver sus ejercicios.
		\item Elegir la opción correspondiente.
	}
	\\ \midrule
	\textbf{Postcondiciones} &  - \\ \midrule
	\textbf{Excepciones} & - \\ \midrule
	\textbf{Importancia} & Alta \\
}


\tablaAncho
{CU-4.1.1 Añadir ejercicios}
{p{2.9cm} X}
{use-case-4.1.1}
{
	\textbf{CU-4.1.1} & \textbf{Añadir ejercicios} \\ \otoprule
	\textbf{Versión} & 1.0 \\ \midrule
	\textbf{Autor} & Jose Manuel Moral Garrido \\ \midrule
	\textbf{Requisitos asociados} & RF-4.1 \\ \midrule
	\textbf{Descripción} & Permite añadir un ejercicio. \\ \midrule
	\textbf{Precondiciones} & 
	\tabitem Haber accedido a la aplicación correctamente.
	\tabitem Tener rol de Profesor.
	\\ \midrule
	\textbf{Acciones} & 
	\enumeratecompacto{
		\item Entrar en la aplicación.
		\item Elegir la opción de Ejercicios.
		\item Elegir un modelo para ver sus ejercicios.
		\item Elegir la opción de añadir.
	}
	\\ \midrule
	\textbf{Postcondiciones} &
	\tabitem Se abre el visor de ejercicios para dicho modelo.
	\\ \midrule
	\textbf{Excepciones} & - \\ \midrule
	\textbf{Importancia} & Alta \\
}


\tablaAncho
{CU-4.1.2 Eliminar ejercicios}
{p{2.9cm} X}
{use-case-4.1.2}
{
	\textbf{CU-4.1.2} & \textbf{Eliminar ejercicios} \\ \otoprule
	\textbf{Versión} & 1.0 \\ \midrule
	\textbf{Autor} & Jose Manuel Moral Garrido \\ \midrule
	\textbf{Requisitos asociados} & RF-4.1 \\ \midrule
	\textbf{Descripción} & Permite eliminar un ejercicio. \\ \midrule
	\textbf{Precondiciones} & 
	\tabitem Haber accedido a la aplicación correctamente.
	\tabitem Tener rol de Profesor.
	\\ \midrule
	\textbf{Acciones} & 
	\enumeratecompacto{
		\item Entrar en la aplicación.
		\item Elegir la opción de Ejercicios.
		\item Elegir un modelo para ver sus ejercicios.
		\item Elegir la opción de eliminar (botón de papelera).
		\item Elegimos en el diálogo emergente si queremos seguir adelante.
	}
	\\ \midrule
	\textbf{Postcondiciones} &
	\tabitem Se elimina el ejercicio si elegimos que se elimine en la comprobación.
	\\ \midrule
	\textbf{Excepciones} & \\
	\textbf{Importancia} & Alta \\
}


\tablaAncho
{CU-4.1.3 Modificar ejercicios}
{p{2.9cm} X}
{use-case-4.1.3}
{
	\textbf{CU-4.1.3} & \textbf{Modificar ejercicios} \\ \otoprule
	\textbf{Versión} & 1.0 \\ \midrule
	\textbf{Autor} & Jose Manuel Moral Garrido \\ \midrule
	\textbf{Requisitos asociados} & RF-4.1 \\ \midrule
	\textbf{Descripción} & Permite Modificar un ejercicio. \\ \midrule
	\textbf{Precondiciones} & 
	\tabitem Haber accedido a la aplicación correctamente.
	\tabitem Tener rol de Profesor.
	\\ \midrule
	\textbf{Acciones} & 
	\enumeratecompacto{
		\item Entrar en la aplicación.
		\item Elegir la opción de Ejercicios.
		\item Elegir un modelo para ver sus ejercicios.
		\item Elegir la opción de modificar (botón de llave inglesa).
		\item Se abre el ejercicio guardado correspondiente.
	}
	\\ \midrule
	\textbf{Postcondiciones} &  - \\ \midrule
	\textbf{Excepciones} & - \\ \midrule
	\textbf{Importancia} & Alta \\
}


\tablaAncho
{CU-4.1.4 Editar nombre ejercicios}
{p{2.9cm} X}
{use-case-4.1.4}
{
	\textbf{CU-4.1.4} & \textbf{Editar nombre ejercicios} \\ \otoprule
	\textbf{Versión} & 1.0 \\ \midrule
	\textbf{Autor} & Jose Manuel Moral Garrido \\ \midrule
	\textbf{Requisitos asociados} & RF-4.1 \\ \midrule
	\textbf{Descripción} & Permite editar el nombre de un ejercicio. \\ \midrule
	\textbf{Precondiciones} & 
	\tabitem Haber accedido a la aplicación correctamente.
	\tabitem Tener rol de Profesor.
	\\ \midrule
	\textbf{Acciones} & 
	\enumeratecompacto{
		\item Entrar en la aplicación.
		\item Elegir la opción de Ejercicios.
		\item Elegir un modelo para ver sus ejercicios.
		\item Elegir la opción de modificar (botón de lápiz).
		\item Se un cuadro de edición del nombre del ejercicio.
	}
	\\ \midrule
	\textbf{Postcondiciones} & - \\ \midrule
	\textbf{Excepciones} & - \\ \midrule
	\textbf{Importancia} & Alta \\
}


\tablaAncho
{CU-4.1.5 Guardar ejercicios}
{p{2.9cm} X}
{use-case-4.1.5}
{
	\textbf{CU-4.1.5} & \textbf{Editar nombre ejercicios} \\ \otoprule
	\textbf{Versión} & 1.0 \\ \midrule
	\textbf{Autor} & Jose Manuel Moral Garrido \\ \midrule
	\textbf{Requisitos asociados} & RF-4.1 \\ \midrule
	\textbf{Descripción} & Permite guardar un ejercicio modificado. \\ \midrule
	\textbf{Precondiciones} & 
	\tabitem Haber accedido a la aplicación correctamente.
	\tabitem Tener rol de Profesor.
	\\ \midrule
	\textbf{Acciones} & 
	\enumeratecompacto{
		\item Entrar en la aplicación.
		\item Elegir la opción de Ejercicios.
		\item Elegir un modelo para ver sus ejercicios.
		\item Elegir la opción de modificar (botón de llave inglesa).
		\item Se abre el visor de ejercicios.
		\item Si se realizan modificaciones se pulsa el botón de guardar para salvar el ejercicio.
	}
	\\ \midrule
	\textbf{Postcondiciones} & - \\ \midrule
	\textbf{Excepciones} & - \\ \midrule
	\textbf{Importancia} & Alta \\
}


\tablaAncho
{CU-5 Subir modelo}
{p{2.9cm} X}
{use-case-5}
{
	\textbf{CU-5} & \textbf{Subir modelo} \\ \otoprule
	\textbf{Versión} & 1.0 \\ \midrule
	\textbf{Autor} & Jose Manuel Moral Garrido \\ \midrule
	\textbf{Requisitos asociados} & \\ \midrule
	\textbf{Descripción} & Permite subir modelo. \\ \midrule
	\textbf{Precondiciones} & 
	\tabitem Haber accedido a la aplicación correctamente.
	\tabitem Tener rol de Profesor.
	\\ \midrule
	\textbf{Acciones} & 
	\enumeratecompacto{
		\item Entrar en la aplicación.
		\item Elegir la opción de Subir Modelo.
	}
	\\ \midrule
	\textbf{Postcondiciones} & - \\ \midrule
	\tabitem Elegir modelo a subir.
	\textbf{Excepciones} & - \\ \midrule
	\textbf{Importancia} & Alta \\
}