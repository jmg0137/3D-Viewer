\capitulo{1}{Introducción}

Siendo una de la grandes dificultades la de impartir de manera \textit{online} el material docente explicado en clases de prácticas, partiremos de un proyecto que busca enriquecer las herramientas disponibles para dicho fin en el Grado en Historia y Patrimonio.

Nuestro objetivo principal será seguir la estela del proyecto de Alberto Vivar Arribas~\cite{github:alberto-viewer} y aumentar las funcionalidades de esta herramienta, así como pulir las partes actualmente funcionales. Un punto importante será el poder obtener acceso desde cualquier ordenador a nuestra herramienta. Para ello, trataremos de desplegar la aplicación en un servidor accesible para todos los alumnos de la Universidad de Burgos que se encuentren cursando el Grado en Historia y Patrimonio. A su vez, como objetivo clave se tiene el proporcionar seguridad en los modelos 3D de los huesos debido a su unicidad y privacidad, por lo que se adoptarán las medidas necesarias para dicho fin.

Por otro lado, cabe mencionar la importancia de los docentes en nuestra aplicación, por lo que se les tratará de facilitar el uso de la misma enfocándonos en el ámbito de la corrección de ejercicios y comprobación de copias entre alumnos. Así mismo, incluiremos una correcta gestión de roles en nuestra aplicación para que ésta sea capaz de distinguir qué usuario está haciendo uso de ella para poder así proporcionar unas funcionalidades u otras.

Los alumnos no juegan un papel secundario en la aplicación, pero sí es verdad que tienen unas funcionalidades más acotadas en la misma debido a que su mal uso podría dar lugar a la eliminación de algún modelo óseo 3D que pudiera ser irreversible. A su vez, prevendremos que se introduzcan modelos que no tengan que ver con la asignatura, por parte de los alumnos, en la aplicación.
