\capitulo{7}{Conclusiones y Líneas de trabajo futuras}

\section{Conclusiones}

\section{Líneas de trabajo futuras}

\subsection{Integraión de <<MorphoJ>>}
Un aspecto interesante sería la utilización de \textit{MorphoJ} en nuestra aplicación con el fin de obtener datos matemáticamente analizables a partir de la <<forma>> del modelo 3D en cuestión. Esta herramienta sería de gran utilidad si, por ejemplo, nuestra finalidad sería la de analizar matemáticamente un modelo y, a partir de esa respuesta, obtener la mayor información posible.

\subsection{Integración con Moodle}
Por otro lado, otro aspecto interesante de ampliación con nuestra aplicación sería la integración de los servicios ofrecidos en \textit{UBUVirtual}. El más relevante de todos es el sistema de cuestionarios integrado en \textit{Moodle} el cual permitiría a los usuarios realizar otro tipo de ejercicios, así como en un menor número de pasos.

\subsection{Comercialización}
Finalmente, otro punto interesante es el de pensar en monetizar nuestra aplicación con varios enfoques. Por un lado permitir el acceso a otros investigadores o profesores de otras instituciones.
Por otro lado pensaríamos en desplegar nuestro proyecto sobre otros campos con necesidades similares a las cuales se les puedan añadir más características como podría ser el soporte de otros formatos.