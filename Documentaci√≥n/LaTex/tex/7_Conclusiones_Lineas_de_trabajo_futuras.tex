\capitulo{7}{Conclusiones y Líneas de trabajo futuras}
En esta sección expondremos las conclusiones extraídas a lo largo del proyecto, así como posibles líneas de trabajo futuras.

\section{Conclusiones}
Hemos podido extraer las siguientes conclusiones:

\begin{itemize}
	\item Se han conseguido cumplir todos los objetivos del proyecto.
	\item Gracias a la estructura proporcionada de la versión anterior hemos podido trabajar y comprender la estructura de la aplicación de manera sencillas, que en un futuro seguirá sirviendo.
	\item El desconocimiento de las técnicas ha retrasado el avance debido a que las dudas que surgían no eran fáciles de buscar de manera particular, teniendo que buscar de manera genérica y aplicarlo a nuestro caso en su defecto.
	\item El trabajar con una tercera persona de la que se depende (como en este caso el operador de la Universidad de Burgos) puede llegar a retrasar aún más el avance del proyecto. Esto es debido a que el cometido de esta tercera persona no es únicamente avanzar en nuestro trabajo, sino que puede tener otras prioridades.
	\item Ciertas tareas de las que desconocíamos su materia se han estimado de manera errónea, aunque las metodologías ágiles han ayudado en el proceso.
\end{itemize}
\section{Líneas de trabajo futuras}

\subsection{Integraión de <<MorphoJ>>}
Un aspecto interesante sería la utilización de \textit{MorphoJ} en nuestra aplicación con el fin de obtener datos matemáticamente analizables a partir de la <<forma>> del modelo 3D en cuestión. Esta herramienta sería de gran utilidad si, por ejemplo, nuestra finalidad sería la de analizar matemáticamente un modelo y, a partir de esa respuesta, obtener la mayor información posible.

\subsection{Integración con Moodle}
Por otro lado, otro aspecto interesante de ampliación con nuestra aplicación sería la integración de los servicios ofrecidos en \textit{UBUVirtual}. El más relevante de todos es el sistema de cuestionarios integrado en \textit{Moodle} el cual permitiría a los usuarios realizar otro tipo de ejercicios, así como en un menor número de pasos.

\subsection{Comercialización}
Finalmente, otro punto interesante es el de pensar en monetizar nuestra aplicación con varios enfoques. Por un lado permitir el acceso a otros investigadores o profesores de otras instituciones.
Por otro lado pensaríamos en desplegar nuestro proyecto sobre otros campos con necesidades similares a las cuales se les puedan añadir más características como podría ser el soporte de otros formatos.