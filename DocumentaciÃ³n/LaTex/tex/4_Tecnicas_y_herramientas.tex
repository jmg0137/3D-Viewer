\capitulo{4}{Técnicas y herramientas}
A continuación mostraremos las diferentes técnicas y herramientas utilizadas para el desarrollo de nuestro proyecto. Así mismo, se han incluido únicamente las técnicas y herramientas nuevas. El resto de técnicas y herramientas comunes a la versión anterior pueden consultarse en las páginas 9-12 de la memoria del proyecto de Alberto Vivar Arribas~\cite{github:alberto-viewer}.

\section{Moodle}\label{sec:moodle-local}
Previo a centrarnos en \textit{Moodle} en particular, tendremos que saber que se trata de un sistema de gestión de aprendizaje (\textit{Learning Management System}). Estos sistemas se utilizan para administrar y gestionar las actividades de formación no presencial, permitiendo un trabajo asíncrono entre los participantes. Dentro de sus competencias encontramos la gestión de usuarios, gestión de recursos o material docente, actividades de formación, seguimiento en el proceso de aprendizaje de los participantes, realizar evaluaciones, foros de consulta, etc.~\cite{wiki:sistemaGestionAprendizaje}. Una vez conocido esto, indaguemos en por qué \textit{Moodle} forma parte de este apartado.

Aunque nuestra aplicación esté orientada a impartir una docencia online desde la \textit{API} de UBUVirtual, cabe mencionar \textit{Moodle} como una herramienta. Esto es debido a que ha sido instalada de manera local con el fin de no tener que depender un super usuario que nos proporcione los recursos necesarios para las pruebas pertienentes (asignaturas, cursos, etc.) que necesitamos en cada caso.

\textit{Moodle} es una herramienta escalable, personalizable, económica y segura a la par que flexible. Dentro del gran número de funcionalidades podemos destacar las siguientes:

\begin{itemize}
	\item Facilidad de uso
	\item Gestión de perfiles de usuario
	\item Facilidad de acceso
	\item Administración sencilla
	\item Realización de exámenes en línea
	\item Gestión de tareas
	\item Implementación de aulas virtuales
\end{itemize}

Moodle ofrece ciertas características de administración, dentro de las cuales entran los roles de usuario, que tienen un papel importante en nuestra \textit{API} (sección~\ref{sec:roles}). Los privilegios de cada uno de los roles son distintos y cada uno tiene una funcionalidades diferentes y es por esto que necesitábamos una instalación local, para poder probar cada uno de los roles~\cite{wiki:moodle}.

\section{Sublime Text}
Para la edición del los diferentes \textit{scripts} utilizaremos este editor de texto ya que, además de ser gratuito (aunque un tanto cargante con solicitar la compra de la versión de pago), es intuitivo y nos proporciona una interfaz cómoda para trabajar,además de una función de auto completar altamente útil.

\section{\TeX Studio}
Como se menciona en \textit{Wikipedia}: "\TeX{} studio es un editor de \LaTeX{} de código abierto y multiplataforma con una interfaz similar a \TeX{} maker". Esta herramienta es un IDE de \LaTeX{} que proporciona soporte de escritura incluyendo la corrección ortográfica, plegado de código y resaltado de sintaxis~\cite{wiki:texstudio}.


\section{JSONMate}
Hemos utilizado esta herramienta, la cual en realidad es una página web que nos ayuda a interpretar la información obtenida en formato \textit{JSON} y simplificarnos su vista~\cite{json:jsnomate}.

\section{PuTTY}
\textit{PuTTY} es un cliente \textit{SSH, Telnet, rlogin, y TCP raw} disponible para \textit{Windows} y en varias plataformas de \textit{Unix} (Versión \textit{Mac OS}). Hemos utilizado esta herramienta para acceder al servidor Arquimedes (sección \ref{sec:arquimedes}) desde nuestra máquina con \textit{Windows}, ya que desde \textit{Linux} realizamos las llamadas mediante el comando \texttt{ssh}~\cite{wiki:putty}.

\section{Servidor Arquimedes}\label{sec:arquimedes}
Se trata de un servidor proporcionado por la Universidad de Burgos para poder desplegar nuestra aplicación. Este servidor contiene una máquina con un sistema operativo \textit{Ubuntu} 16.04\footnote{\url{https://arquimedes.ubu.es/}}.

\section{Comparativa servidores para desplegar Flask}\label{sec:comparativa-servidores}
Dado que la Universidad de Burgos nos ha dotado con un servidor (ver sección ~\ref{sec:arquimedes}) tenemos que elegir la manera de desplegar nuestra \textit{API Flask}, para lo cual hemos realizado una comparativa con diversas herramientas para desplegarla con el fin de encontrar la mejor manera de hacerlo.

A continuación, mostraremos una tabla con las diferentes herramientas seleccionadas con el fin de elegir la que mejor se amolde a nuestro caso:

\begin{landscape}
	\begin{table}
		\centering
		\caption{Tabla comparativa servidores. ¿Por qué utilizar cada servidor?}
		\label{serverTable}
		\begin{tabular}{p{5cm} p{5cm} p{5cm} p{5cm}}
			\toprule
			\textbf{Apache} & \textbf{uWSGI} & \textbf{Stand-Alone (Gunicorn)} & \textbf{Stand-Alone (Twisted Web)} \\
			\midrule
			En el caso de tener experiencia utilizando Apache, así como que se tenga una dependencia del mismo, esto significará \textbf{estabilidad} en el entorno de producción de la aplicación, teniendo gran variedad de módulos estable sy completos. A su vez, es un software muy probado y fiable, teneindo gran variedad de información en la web. & Soporta aplicaciones Python por completo corriendo en WSGI, pudiendo sus componentes realizar muchas más funciones que correr la aplicación, con la correspondiente bajón en el uso de la memoria. Como \textbf{desventaja} hemos considerado que, como está actualmente en desarrollo, podría conllevar a un fallo que aún no se halla contemplado, teniendo a su vez una convención de nombres confusa. & \textbf{Gunicorn}: Si se desea extender de \textit{Apache} utilizando Python(siempre y cuando sea necesario) y programarlo para alguna tarea en concreto. Además, tiene la ventaja de ser sencillo de ejecutar si no se necesita extender de \textit{Apache} & \textbf{Twisted Web}: Si se desea extender de \textit{Apache} utilizando Python(siempre y cuando sea necesario ) siendo simple, estable y maduro. A su vez, puede soportar clientes virtuales.\\
			\bottomrule
		\end{tabular}
	\end{table}
\end{landscape}

Inicialmente escogimos \textit{Gunicorn} como método de despliegue de nuestra aplicación debido a su sencillez y compatibilidad con \textit{Apache} (compatible con extensiones y utilidades). A medida que avanzamos, nos dimos cuenta de que \textit{Gunicorn} no cumplía nuestras expectativas, así como una falta de información de configuración en nuestro caso. Por ello, finalmente nos decantamos por \textit{Apache} (mod\_wsgi) para el despliegue de nuestra aplicación.

\section{<<Librería>> Pyntcloud}
La palabra librería se encuentra entrecomillada ya que no es una biblioteca en sí misma, sino una serie de recursos que en nuestro caso han sido utilizados para la lectura de os archivos \textit{PLY}, tanto formato \textit{ASCII} como \textit{Binario} (ver sección \textbf{3.2 Formato PLY}~\cite{github:alberto-viewer} (Consultar apartado Conceptos Teóricos del proyecto predecesor al nuestro).)\footnote{\url{https://github.com/daavoo/pyntcloud/blob/master/pyntcloud/io/ply.py}}.

De esta manera, los modelos se leerán tanto en formato \textit{ASCII} como en \textit{Binario} almacenando los vértices y las caras por separado en dos \textit{DataFrames} diferentes. A su vez, diferencia entre los modelos con color a los modelos en escala de grises utilizando estructuras como las que se aprecia en las figuras~\ref{fig:dataframeColor} y~\ref{fig:dataframeNoColor}:
\imagen{dataframeColor}{DataFrame que contiene los vértices y colores.}{0.9}
\imagen{dataframeNoColor}{DataFrame que contiene únicamente los vértices.}{0.7}